% Copyright (C) 2011,2012 The ESPResSo project
%  
% This file is part of ESPResSo.
%   
% ESPResSo is free software: you can redistribute it and/or modify it
% under the terms of the GNU General Public License as published by the
% Free Software Foundation, either version 3 of the License, or (at your
% option) any later version.
%  
% ESPResSo is distributed in the hope that it will be useful, but
% WITHOUT ANY WARRANTY; without even the implied warranty of
% MERCHANTABILITY or FITNESS FOR A PARTICULAR PURPOSE.  See the GNU
% General Public License for more details.
%  
% You should have received a copy of the GNU General Public License
% along with this program.  If not, see <http://www.gnu.org/licenses/>.
%
\documentclass[
a4paper,                        % paper size
11pt,                           % font size
]{scrartcl}

\usepackage{hyperref}           % automatically creates links when
                                % using pdflatex, defines \url
\usepackage[numbers]{natbib}
\usepackage{xspace}

\newcommand{\es}{\mbox{\textsf{ESPResSo}}\xspace}
\newcommand{\ie}{\textit{i.e.}\xspace}
\newcommand{\eg}{\textit{e.g.}\xspace}
\newcommand{\etal}{\textit{et al.}\xspace}

\newenvironment{designrule}{\begin{quote}\itshape}{\end{quote}}

\begin{document}
\title{\es Python Interface Design}
\author{Olaf Lenz}
\date{\today}
\maketitle

\tableofcontents

\clearpage
\section{Design Criteria}

\subsection{Zen of Python}

\begin{verse}
Beautiful is better than ugly.\\
Explicit is better than implicit.\\
Simple is better than complex.\\
Complex is better than complicated.\\
Flat is better than nested.\\
Sparse is better than dense.\\
Readability counts.\\
Special cases aren't special enough to break the rules.\\
Although practicality beats purity.\\
Errors should never pass silently.\\
Unless explicitly silenced.\\
In the face of ambiguity, refuse the temptation to guess.\\
There should be one-- and preferably only one --obvious way to do it.\\
Although that way may not be obvious at first unless you're Dutch.\\
Now is better than never.\\
Although never is often better than \emph{right} now.\\
If the implementation is hard to explain, it's a bad idea.\\
If the implementation is easy to explain, it may be a good idea.\\
Namespaces are one honking great idea -- let's do more of those!
\end{verse}

This is the ``Zen of Python''\footnote{PEP 20:
  \url{http://www.python.org/dev/peps/pep-0020/}} a set of rules
embraced by the Python community.  We should heed them, so that
experienced Python users feel at home.

\subsection{Principle of Least Surprise}

\begin{designrule}
  Design the interface such that it works as the user expects, not as
  it would feel natural from the inner logic of the program.
\end{designrule}


This is the \emph{Principle of Least Surprise} or \emph{Principle of
  Least Astonishment} (POLA)
\footnote{\url{http://en.wikipedia.org/wiki/Principle_of_least_astonishment}}.

\subsection{Maximal Freedom in the Interface}

\begin{designrule}
  Allow for maximal possible freedom in the interface.
\end{designrule}

\begin{itemize}
\item Use \emph{runtime restrictions} rather than \emph{interface
    restrictions}.
\item You never know when a restriction might be lifted. Doing so is
  much easier when using runtime restrictions.
\item Don't restrict or because the inner logic of the program
  doesn't allow for more yet.
\item Do not force specific types where it is not necessary.
\end{itemize}

\paragraph{Example 1}

\begin{itemize}
\item Design 1
\begin{verbatim}
system.setElectrostatics(method="p3m", bjerrum=1.0)
\end{verbatim}

\item Design 2
\begin{verbatim}
p3m = P3M(bjerrum=1.0)
system.addInteraction(p3m)
\end{verbatim}

\item Design 1 does not allow to use several instances of P3M, while
  Design 2 does. If this should be restricted, Design 2 could throw an
  exception if the users tries to add more than one P3M.
\end{itemize}

\paragraph{Example 2}
There is no reason to allow only numerical particle ids in the
future. Instead, a particle could be identified by anything else, \eg
a string. In this case, the Python interface might provide a bridge
between the inner program logic and the user.

\subsection{Algorithms are not Physics}

\begin{designrule}
  Algorithms are not physics.
\end{designrule}

Algorithms can be used to model physics, but the same algorithm might
be able to model another aspect of physics. For example, P3M can be
used to model electrostatics as well as gravity.

This has a few implications:

\begin{itemize}
\item Don't restrict what a user can do just because you think ``it
  would be unphysical''.

  A user should have the freedom to employ algorithms wherever
  possible, even if you think it might be unphysical. You might simply
  not know what new method the user has thought of.

\item Choose class names and method names after algorithms, not intended
  physics.

  For example, a class that implements the P3M algorithm should be
  called \texttt{P3M}, not \texttt{Electrostatics}. On the one hand,
  P3M can also be used to simulate gravity, on the other hand, there
  might be other algorithms that implement electrostatics.
\end{itemize}

\subsection{Have the Future in Mind}

\begin{designrule}
  Don't model what is, model what should be.
\end{designrule}

We should try to create an interface that allows for future
development rather than an interface that models exactly what is
currently possible with \es.

For example, don't make variables global because the are currently
global in \es (\eg don't make the time step global).

\paragraph{Example}
\begin{verbatim}
>>> # unnecessarily global
>>> espresso.time_step = 0.01
>>> # should be a parameter of the integrator
>>> integrator = espresso.integrator.VelocityVerlet(time_step=0.01)
\end{verbatim}

\subsection{Specifity}

\begin{designrule}
  Design and name the classes and methods as generic as possible but
  as specific as necessary.
\end{designrule}

\begin{itemize}
\item Call the velocity verlet integrator \texttt{VelocityVerlet}, not
  \texttt{Integrator}. After all, there might be a different
  integrator at some time in the future.
\end{itemize}

\subsection{Keyword Arguments}

\begin{designrule}
  Prefer keyword arguments over positional arguments!
\end{designrule}

\clearpage
\section{Open Questions}

\subsection{Namespaces}

Should we employ namespaces, and to what extent?

\subsubsection{One Big Flat Namespace}

Put everything into module \texttt{espresso}.

\paragraph{Example}
\begin{verbatim}
>>> integrator = espresso.VelocityVerletIntegrator()
\end{verbatim}

\paragraph{Pros}
\begin{itemize}
\item "Flat is better than nested" (Zen of Python)
\item Everything needs to be defined in \texttt{espresso.py}
\end{itemize}

\paragraph{Cons}
\begin{itemize}
\item Hard to extend for other users
\end{itemize}

\subsubsection{Nested Namespaces}

Use nested namespaces, name the namespace after the usage of the
algorithms. Modules by other users can be put into namespace
\texttt{extensions}.

\paragraph{Example}
\begin{verbatim}
>>> integrator = espresso.integrator.VelocityVerlet()
>>> newinteraction = espresso.extensions.specialinteraction.SpecialInteraction()
\end{verbatim}

\paragraph{Pros}
\begin{itemize}
\item Logical
\item Models physical intuition
\item Good to extend
\end{itemize}

\clearpage
\section{Interface}

\subsection{Particles}

\begin{itemize}
\item A user should not keep references to all particles to allow for
  huge parallel systems, \ie it should not be necessary for a user to
  keep a list or dictionary of al particles!
\item Particles should be easily addressable (\eg via a numerical id)
\item Is it necessary to restrict the ids to numerical ids?
\item Should it be \emph{possible} to keep particle references as
  Python objects?
\end{itemize}

\subsection{Interactions}



\end{document}
