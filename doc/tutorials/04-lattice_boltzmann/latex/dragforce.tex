\section{Drag force on objects}
As a first test, we measure the drag force on different objects in a simulation
box. Under low Reynolds number conditions, an object with velocity $\vec{v}$
experiences a drag force $\vec{F}_\text{D}$ proportional to the velocity:
\begin{align*}
	\vec{F}_\text{D}=-\gamma\vec{v},
\end{align*}
where $\gamma$ is denoted the friction coefficient. In general $\gamma$ is a
tensor thus the drag force is generally not parallel to the velocity. For
spherical particles the drag force is given by Stokes' law:
\begin{align*}
	\vec{F}_\text{D}=-6\pi\eta a\vec{v},
\end{align*}
where $a$ is the radius of the sphere.

In this task you will measure the drag force on falling objects with LB and
\ES{}. In the sample script \texttt{lb\_stokes\_force.py} a spherical object at rest
is centered in a square channel. Bounce back boundary conditions are assumed on
the sphere. At the channel boundary the velocity is fixed by using appropriate
boundary conditions. Within a few hundred or thousand  integration steps a
steady state develops and the force on the sphere converges.

\subsection*{Radius dependence of the drag force}
Measure the drag force for three different input radii of the sphere. How good
is the agreement with Stokes' law? Calculate an effective radius from Stokes'
law and the drag force measured in the simulation. Is there a clear relation to
the input radius? Remember how the bounce back boundary condition work and how
good spheres can be represented by them.

\subsection*{Visualization of the flow field}
The script produces \texttt{vtk} files of the flow field. Visualize the flow field
with \texttt{paraview}. Open \texttt{paraview} by typing it on the command line. Make
sure you are in the folder where the files are located. So the agenda is:
\begin{itemize}
	\item Click in the menu \texttt{File}, \texttt{Open...}
	\item Choose the files with flow field \texttt{fluid...vtk}
	\item Click \texttt{Apply}
	\item Add a stream tracer filter \texttt{Filters}, \texttt{Alphabetical},
	        \texttt{Stream tracer}
	\item Change the seed type from \texttt{point source} to \texttt{high resolution
		line source}
	\item Click \texttt{Apply}
	\item Rotate the visualization box to see the stream lines.
	\item Use the play button in the bar below the menu bar to show the time
		evolution.
\end{itemize}

\subsection*{System size dependence}
Measure the drag force for a fixed radius but varying system size. Does the drag
force increase of decrease with the system size? Can you find a qualitative
explanation?
