

\section{The LB interface in \ES{}}
\ES{} features two virtually independent implementations of LB. One implementation
uses CPUs and one uses a GPU to perform the computational work. For this, we
provide two actor classes \texttt{LBFluid} and \texttt{LBFluid\_GPU} with the
module \texttt{espressomd.lb}, as well as the optional \texttt{LBBoundary} class
found in \texttt{espressomd.lbboundaries}.

The LB lattice is a cubic lattice, with a lattice constant \texttt{agrid} that
is the same in all spacial directions. The chosen box length must be an integer multiple
of \texttt{agrid}. The LB lattice is shifted by 0.5 \texttt{agrid} in all directions: the node
with integer coordinates $\left(0,0,0\right)$ is located at
$\left(0.5a,0.5a,0.5a\right)$.
The LB scheme and the MD scheme are not synchronized: In one
LB time step, several MD steps may be performed. This allows to speed
up the simulations and is adjusted with the parameter \texttt{tau}.
The LB parameter \texttt{tau} must be an integer multiple of the MD timestep.

Even if MD features are not used, the System parameters \texttt{cell\_system.skin} and
\texttt{time\_step} must be set. LB steps are performed 
in regular intervals, such that the timestep $\tau$ for LB is recovered. 

Important Notice: All commands of the LB interface use
MD units. This is convenient, as e.g. a particular 
viscosity can be set and the LB time step can be changed without
altering the viscosity. On the other hand this is a source
of a plethora of mistakes: The LBM is only reliable in a certain 
range of parameters (in LB units) and the unit conversion
may take some of them far out of this range. So note that you always
have to assure that you are not messing with that!

One brief example: a certain velocity may be 10 in MD units.
If the LB time step is 0.1 in MD units, and the lattice constant
is 1, then it corresponds to a velocity of $1\ \frac{a}{\tau}$ in LB units.
This is the maximum velocity of the discrete velocity set and therefore
causes numerical instabilities like negative populations.

\subsection*{The \texttt{LBFluid} class}
The \texttt{LBFluid} class provides an interface to the LB-Method in the \ES{}
core. When initializing an object, one can pass the aforementioned parameters
as keyword arguments. Parameters are given in MD units. The available keyword
arguments are:



\vspace{0,2cm}
\begin{tabular}{p{0.2\columnwidth}p{0.5\columnwidth}}
\texttt{dens} & The density of the fluid.\\
\texttt{agrid} & The lattice constant of the fluid. It is used to determine the number of LB nodes 
per direction from \texttt{box\_l}. {\em They have to be compatible.} \\
\texttt{visc} & The kinematic viscosity \\
  \texttt{tau} & The time step of LB. It has to be an integer multiple of
  \texttt{System.time\_step}. \\
\texttt{fric} & The friction coefficient $\gamma$ for the coupling scheme. \\
\texttt{ext\_force} & An external force applied to every node. This is given as
a list, tuple or array with three components.%\\
%\texttt{gamma\_odd} & Relaxation parameter for the odd kinetic modes. \\
%\texttt{gamma\_even} & Relaxation parameter for the even kinetic modes.
\end{tabular}\\ 
\vspace{0,2cm}

Using these arguments, one can initialize an LBFluid object. This object then
needs to be added to the systen's actor list. The code below provides a minimum
example.
\vspace{0,2cm}
\begin{pypresso}
  from espressomd import system, lb

  # initialize the System and set the necessary MD parameters for LB.
  s = system.System()
  s.box_l = [31, 41, 59]
  s.time_step = 0.01
  s.cell_system.skin = 0.4

  # Initialize and LBFluid with the minimum set of valid parameters.
  lbf = lb.LBFluid(agrid = 1, dens = 10, visc = .1, tau = 0.01)
  # Activate the LB by adding it to the System's actor list.
  s.actor.add(lbf)
\end{pypresso}
\vspace{0,2cm}
Note: The same applies for the class \texttt{LBFluid\_GPU}.

\subsection*{Sampling data from a node}
The \texttt{LBFluid} class also provides a set of methods which can be used to
sample data from the fluid nodes. For example 
\vspace{0,2cm}
\begin{pypresso}
  lbf[X,Y,Z].quantity
\end{pypresso}
\vspace{0,2cm}
returns the quantity of the node with $(X,Y,Z)$ coordinates. Note that the
indexing in every direction starts with 0. The possible properties are:
\vspace{0,8cm}
\begin{tabular}{p{0.2\columnwidth}p{0.5\columnwidth}}
  \texttt{velocity} & the fluid velocity (list of three floats) \\
  \texttt{pi} & the stress tensor (list of six floats: $\Pi_{xx}$, $\Pi_{xy}$, $\Pi_{yy}$, $\Pi_{xz}$,  $\Pi_{yz}$,  $\Pi_{zz}$) \\
  \texttt{pi\_neq} & the nonequilbrium part of the stress tensor, components as above. \\
  \texttt{population} & the 19 populations of the D3Q19 lattice.\\
  \texttt{boundary} & The boundary index.\\
  \texttt{density}\ & the local density. 
\end{tabular} \\
\vspace{0,8cm}

\subsection*{The \texttt{LBBoundary} class}
The \texttt{LBBoundary} class represents a boundary on the \texttt{LBFluid}
lattice. It is dependent on the classes of the module $\texttt{espressomd.shapes}$ as it
derives its geometry from them. For the initialization, the arguments
\texttt{shape} and \texttt{velocity} are supported. The \texttt{shape} argument
takes an object from the \texttt{shapes} module and the \texttt{velocity}
argument expects a list, tuple or array containing 3 floats. Setting the
\texttt{velocity} will results in a slip boundary condition.

Note that the boundaries are not constructed
through the periodic boundary. If, for example, one would set a sphere with
its center in one of the corner of the boxes, a sphere fragment will be
generated. To avoid this, make sure the sphere, or any other boundary, fits
inside the original box.

This part of the LB implementation is still experimental,
so please tell us about your experience with it. In general even the simple case of no-slip
boundary is still an important research topic in the lb community and in combination with
point particle coupling not much experience exists. This means: Do research on that topic, play
around with parameters and figure out what happens. 

Boundaries are initialized by passing a shape object to the \texttt{LBBoundary}
class. One way to initialize a wall is:
\begin{pypresso}
  from espressomd import lbboundaries
  wall = lbboundaries.LBBoundary(shape=shapes.Wall(normal=[1,0,0], dist=1), velocity = [0, 0, 0.01])
  s.lbboundaries.add(wall)
\end{pypresso}
Note that all used variables are inherited from previous examples. This will
create a wall a surface normal of $(1,0,0)$ at a distance of $1$ from the origin
of the coordinate system in direction of the normal vector. The wall exhibits a
slip boundary condition with a velocity of $(0,0,0.01)$. For the a no-slip
condition, leave out the velocity argument or set it so zero. Please refer to the
user guide for a complete list of constraints.

Currently only the so called \emph{link bounce back} method is implemented, where the effective
hydrodynamic boundary is located midway between two nodes. This is the simplest and yet a 
rather effective approach for boundary implementation. Using the shape objects
distance function, the nodes determine once during initialisation whether they
are boundary or fluid nodes.

