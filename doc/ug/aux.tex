\chapter{Auxilliary commands}
\label{chap:aux}

\section{Writing VTF files}
%\quickrefheading{Handling of VTF files}

There are two commands in \es{} that support writing files in the VMD
formats VTF, VSF and VCF.\footnote{A description of the format and a
  plugin to read the format in VMD is found in the subdirectory
  \texttt{vmdplugin/} of the \es{} source directory.} The commands can
be used to write the structure (\texttt{writevsf}) and coordinates
(\texttt{writevcf}) of the system to a single trajectory file (usually
with the extension \texttt{.vtf}), or to separate files (extensions
\texttt{.vsf} and \texttt{.vtf}).

\subsection{\texttt{writevsf}}

\tclcommand{writevsf}
{\var{file} [<short|verbose>] [<\var{radii}|auto>] [typedesc \var{typedesc}]}

Writes a structure block describing the system's structure to
\var{file}. The atom ids used in the file are identical to \es's
particle ids.  This makes it easy to write additional structure lines
to the file, e.g. to specify the \texttt{resname} of particle
compounds, like chains.  The output of this file can be used in a
standalone VSF file, or at the beginning of a trajectory VTF file that
contains a trajectory of a whole simulation.

\begin{tcloptions}
  \option{<short|verbose>}
  Specify, whether the output is in a human-readable, but somewhat
  longer format (\keyword{verbose}), or in a more compact form
  (\keyword{short}). The default is \keyword{verbose}.
  
  \option{<radius \var{radii}|auto>}
  Specify the VDW radii of the atoms. \var{radii} is either
  \keyword{auto}, or a Tcl-list describing the radii of the different
  particle types. When the keyword \keyword{auto} is used and a
  Lennard-Jones interaction between two particles of the given type is
  defined, the radius is set to be $\frac{\sigma_{LJ}}{2}$ plus the LJ
  shift.  Otherwise, the radius $0.5$ is substituted. The default is
  \keyword{auto}.
  
  Example: \verb!writevsf "show.tcl" radius {{0 2.0} {1 auto} {2 1.0}}!
  
  \option{typedesc \var{typedesc}}
  \var{typedesc} is a Tcl-list giving additional VTF atom-keywords to
  specify additional VMD characteristics of the atoms of the given type.
  If no description is given for a certain particle type, it defaults to
  \texttt{segid \textit{n}}, where \textit{n} is the type id.

  Example: \verb!writevsf "show.tcl" typedesc {{0 "segid colloid"} {1 "segid pe"}}!
\end{tcloptions}


\subsection{\texttt{writevcf}}
\tclcommand{writevcf}
{\var{file} [<short|verbose>] [<folded|absolute>]}

Writes a coordinate (or timestep) block that contains all coordinates
of the system's particles to \var{file}.

\begin{tcloptions}
  \option{<short|verbose>} Specify, whether the output is in a
  human-readable, but somewhat longer format (\keyword{verbose}), or
  in a more compact form (\keyword{short}). The default is
  \keyword{verbose}.

  \option{<folded|absolute>} Specify whether the particle positions
  are written in absolute coordinates (\keyword{absolute}) or folded
  into the central image of a periodic system (\keyword{folded}). The
  default is \keyword{absolute}.

  \option{<pids \var{pids}|all>} Specify the coordinates of which particles
  should be written. If \keyword{all} is used, all coordinates will be
  written (in the ordered timestep format). Otherwise, \var{pids} has
  to be a Tcl-list specifying the pids of the particles. The default
  is \keyword{all}.
  
  Example: \verb!pids {0 23 42}!

\end{tcloptions}


%%% Local Variables: 
%%% mode: latex
%%% TeX-master: "ug"
%%% End: 
