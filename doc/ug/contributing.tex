% Copyright (C) 2010,2012,2013,2014,2015,2016 The ESPResSo project
% Copyright (C) 2002,2003,2004,2005,2006,2007,2008,2009,2010 
%   Max-Planck-Institute for Polymer Research, Theory Group
%  
% This file is part of ESPResSo.
%   
% ESPResSo is free software: you can redistribute it and/or modify it
% under the terms of the GNU General Public License as published by the
% Free Software Foundation, either version 3 of the License, or (at your
% option) any later version.
%  
% ESPResSo is distributed in the hope that it will be useful, but
% WITHOUT ANY WARRANTY; without even the implied warranty of
% MERCHANTABILITY or FITNESS FOR A PARTICULAR PURPOSE.  See the GNU
% General Public License for more details.
%  
% You should have received a copy of the GNU General Public License
% along with this program.  If not, see <http://www.gnu.org/licenses/>.
%
\chapter{Getting involved}
\label{chap:devel}

Up to date information about the development of \es
can be found at the web page \url{http://espressomd.org}
As the important information can change in time, we will not describe
its contents in detail but rather request the reader to
go directly to the URL.
Among other things, one can find information about the following topics there:

\begin{itemize}
\item FAQ
\item Latest stable release of \es and older releases
\item Obtaining development version of \es
\item Archives of both developers' and users' mailing lists
\item Registering to \es mailing lists
\item Submitting a bug report
\end{itemize}

\section{Community support and mailing lists}

If you have any questions concerning \es which you cannot
resolve by yourself, you may post a message to
the mailing list. Instructions on how to register to the mailing
lists and post messages can be found on the homepage 
\url{http://espressomd.org}.
Before posting a question and waiting for someone to answer, 
it may be useful to search the mailing list archives or FAQ and 
see if you can get the answer immediately.
For several reasons it is recommended to send all questions 
to the mailing lists rather than to contact individual developers:
\begin{itemize}
  \item All registered users get your message and you have a higher 
  probability that it is answered soon.
  \item Your question and the answers are archived and the archives
  can be searched by others.
  \item The answer may be useful also to other registered users.
  \item There may not be a unique answer to your problem and it may 
  be useful to get suggestions from different people.
\end{itemize}

Please remember that this is a community mailing list. It is other
users and developers who are answering your questions. They do it
in their free time and are not paid for doing it.


\section{Contributing your own code}

If you are planning to make an extension to \es or 
already have a piece of your own code which could be useful 
to others, you are very welcome to contribute it to 
the community. 
Before you start making any changes to the code, you
should obtain the current development version of it.
For more information about how to obtain the
development version, refer to the homepage \url{http://espressomd.org}.

It is also generally a good idea to contact the mailing lists before
you start major coding projects. It might be that someone else is
already working on the problem or has a solution at hand.

\section{Developers' guide}
\label{sec:dg}

Besides the User guide, \es also contains a Developers' guide which is
a programmer documentation automatically built from comments in the
source code and using Doxygen.  It provides a cross-referenced
documentation of all functions and data structures available in \es
source code. It can be built by typing
\begin{code}
  make dg
\end{code}
in the build directory. Afterwards it can be found
in the subdirectory of the build directory: \texttt{doc/dg/html/index.html}.

A recent version of this guide can also be found on the \es{} homepage
\url{http://espressomd.org}.

\section{User's guide}

If, while reading this user guide, you notice any mistakes or badly
(if at all) described features or commands, you are very welcome to
contribute to the guide and have others benefit from your knowledge.

For this, you should also checkout the development version as
described on the homepage. As the user guide, like all \es{} code, is
always in flow and changes are made regularly, there are already many
paragraphs marked with a ``todo'' box. To turn on these boxes, edit
the main file \texttt{doc/ug/ug.tex} and adapt the inclusion of the
\LaTeX{} package \texttt{todonotes}.

You can then build the user guide by typing
\begin{code}
  make ug
\end{code}



%%% Local Variables: 
%%% mode: latex
%%% TeX-master: "ug"
%%% End: 
