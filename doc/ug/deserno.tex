% Copyright (C) 2010 The ESPResSo project
% Copyright (C) 2002,2003,2004,2005,2006,2007,2008,2009,2010 Max-Planck-Institute for Polymer Research, Theory Group, PO Box 3148, 55021 Mainz, Germany
%  
% This file is part of ESPResSo.
%   
% ESPResSo is free software: you can redistribute it and/or modify it
% under the terms of the GNU General Public License as published by the
% Free Software Foundation, either version 3 of the License, or (at your
% option) any later version.
%  
% ESPResSo is distributed in the hope that it will be useful, but
% WITHOUT ANY WARRANTY; without even the implied warranty of
% MERCHANTABILITY or FITNESS FOR A PARTICULAR PURPOSE.  See the GNU
% General Public License for more details.
%  
% You should have received a copy of the GNU General Public License
% along with this program.  If not, see <http://www.gnu.org/licenses/>.
%
\chapter{Conversion of Deserno files}
The following procedures are found in scripts/convertDeserno.tcl.

\begin{itemize}
 \item
\begin{code}
convertDeserno2MD <source_file> <destination_file>
\end{code}
converts the particle configuration stored in \var{source\_file} from
Deserno-format into blockfile-format, importing everything to \es{}
and writing it to \var{destination\_file}. The full particle
information, bonds, interactions, and parameters will be converted and
saved.  If \var{destination\_file} is "-1", the data is only loaded
into \es{} and nothing is written to disk.  If \var{destination\_file}
has the suffix \codebox{.gz}, the output-file will be compressed.  The
script uses some assumptions, e. g. on the
\var{particle\_type\_number}s of The part command for polymers,
counter-ions, or on sigma, shift, offset for Lennard-Jones-potentials
(The inter command; current defaults are 2.0, 0, 0, respectively);
these are all set by
\begin{code}
initConversion 
\end{code}
(which is automatically called by convertDeserno2MD) so have a look at
the sourcecode of \codebox{convertDeserno.tcl} in the
\codebox{scripts}-directory for a complete list of assumptions.
However, if for some reasons different values need to be set, it is
possible to bypass the initialization routine and/or override the
default values, e. g. by explicitly executing initConversion,
afterwards overwriting all variables which need to be re-set, and
manually invoking the main conversion script
\begin{code}
convertDeserno2MDmain <source_file> <destination_file> 
\end{code}
  to complete the process.
 \item
\begin{code}
convertMD2Deserno <source_file> <destination_file>
\end{code}
converts the particle configuration stored in the \es{}-blockfile
\var{source\_file} into a Deserno-compatible \var{destination\_file}.
If \var{source\_file} is "-1", the data is entirely taken from \es{}
without loading anything from disk.  If \var{source\_file} has the
suffix \codebox{.gz}, it is assumed to be compressed; otherwise it
will be treated as containing plain text.  Since Deserno stores much
more than \es{} does due to a centralized vs. a local storage policy,
it depends on correct values for the following properties, which
therefore should be contained in \var{source\_file}:
  \begin{enumerate}
  \item the \var{particle\_type\_number} used for polymers,
    counter-ions, and salt-molecules (defaults are: \codebox{set
      type\_P 0}, \codebox{set type\_CI 1}, and \codebox{set type\_S
      2}
  \item the \var{bond\_type\_number} used for FENE-interactions
    (default is: \codebox{set type\_FENE 0})
  \end{enumerate}
  As for convertDeserno2MD, the defaults are set upon initialization by
\begin{code}
initConversion 
\end{code}
(which is automatically called by convertMD2Deserno as well), but may
be overwritten the same way as explained for tcl\_convertDeserno2MD.
However, parameters stored in \var{source\_file} cannot (and will not)
be overwritten, because they were part of the system originally saved
and should not be altered initially.  Note, that some entries in a
Deserno-file cannot be determined at all, these are by default set to
\begin{code}
set prefix AA0000
set postfix 0
set seed -1
set startTime -1
set endTime -1
set integrationSteps -1
set saveResults -1
set saveConfig -1
set subbox_1D -1
set ip -1
set step -1 
\end{code}
but of course may be overwritten as well after calling initConversion
and before continuing with
\begin{code}
convertMD2DesernoMain <source_file> <destination_file>
\end{code}
the actual conversion process.  The Deserno-format assumes knowledge
of the topology, hence a respective analysis is conducted to identify
the type and structure of the polymer network. The script allows for
randomly stored polymer solutions and melts, no matter how they're
messed up; however, crosslinked networks need to be aligned to be
recognized correctly, i.e. they must be set up consecutively, such
that the first chain with \$MPC monomers corresponds to the first
\$MPC particles in [part], the 2nd one to the \$MPC following
particles, etc. etc.
\item It is now possible to save the whole state of \es{}, including
  all parameters and interactions. These scripts make use of that
  advantage by storing everything they find in the Deserno-file - but
  vice versa they also expect you to provide a blockfile containing
  all possible informations.
\end{itemize}
These conversion scripts have been tested with both polymer melts and
end-to-end-crosslinked networks in systems with or without
counterions. It should work with additional salt-molecules or neutral
networks as well, although that hasn't been tested yet - if you've
some of these systems in a Deserno-formated file, please submit them
for extensive analysis.


%%% Local Variables: 
%%% mode: latex
%%% TeX-master: "ug"
%%% End: 
