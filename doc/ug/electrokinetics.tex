% Copyright (C) 2013,2014,2015,2016 The ESPResSo project
%  
% This file is part of ESPResSo.
%   
% ESPResSo is free software: you can redistribute it and/or modify it
% under the terms of the GNU General Public License as published by the
% Free Software Foundation, either version 3 of the License, or (at your
% option) any later version.
%  
% ESPResSo is distributed in the hope that it will be useful, but
% WITHOUT ANY WARRANTY; without even the implied warranty of
% MERCHANTABILITY or FITNESS FOR A PARTICULAR PURPOSE.  See the GNU
% General Public License for more details.
%  
% You should have received a copy of the GNU General Public License
% along with this program.  If not, see <http://www.gnu.org/licenses/>.
%
\newcommand{\lb}{l_\mathrm{B}}
\newcommand{\kT}{k_\mathrm{B}T}

\chapter{\label{sec:electrokinetics}Electrokinetics}
\newescommand{electrokinetics}

The electrokinetics setup in \es{} allows for the description of
electro-hydrodynamic systems on the level of ion density distributions
coupled to a Lattice-Boltzmann (LB) fluid. The ion density
distributions may also interact with explicit charged particles, which
are interpolated on the LB grid.  In the following paragraph we
briefly explain the electrokinetic model implemented in \es, before
we come to the description of the interface.

If you are interested in using the electrokinetic implementation in
\es for scientific purposes, please contact G.~Rempfer before you
start your project.

\section{Electrokinetic Equations}

In the electrokinetics code we solve the following system of coupled continuity,
diffusion-advection, Poisson, and Navier-Stokes equations:
\begin{eqnarray}
\label{eq:ek-model-continuity} \frac{\partial n_k}{\partial t} & = & -\, \nabla \cdot \vec{j}_k \vphantom{\left(\frac{\partial}{\partial}\right)} ; \\
\label{eq:ek-model-fluxes} \vec{j}_{k} & = & -D_k \nabla n_k - \nu_k \, q_k n_k\, \nabla \Phi + n_k \vec{v}_{\mathrm{fl}} \vphantom{\left(\frac{\partial}{\partial}\right)} ; \\
\label{eq:ek-model-poisson} \Delta \Phi & = & -4 \pi \, \lb \, \kT \sum_k q_k n_k \vphantom{\left(\frac{\partial}{\partial}\right)}; \\
\nonumber \left(\frac{\partial \vec{v}_{\mathrm{fl}}}{\partial t} + \vec{v}_{\mathrm{fl}} \cdot \vec{\nabla} \vec{v}_{\mathrm{fl}} \right) \rho_\mathrm{fl} & = & -\kT \, \nabla \rho_\mathrm{fl} - q_k n_k \nabla \Phi \\
\label{eq:ek-model-velocity} & & +\, \eta \vec{\Delta} \vec{v}_{\mathrm{fl}} + (\eta / 3 + \eta_{\text{\,b}}) \nabla (\nabla \cdot \vec{v}_{\mathrm{fl}}) \vphantom{\left(\frac{\partial}{\partial}\right)} ; \\
\label{eq:ek-model-continuity-fl} \frac{\partial \rho_\mathrm{fl}}{\partial t} & = & -\,\nabla\cdot\left( \rho_\mathrm{fl} \vec{v}_{\mathrm{fl}} \right) \vphantom{\left(\frac{\partial}{\partial}\right)} , 
\end{eqnarray}
which define relations between the following observables
\begin{description}[itemsep=0cm,labelindent=1.5em,leftmargin=4.5em,style=nextline]
  \item[$n_k$] the number density of the particles of species $k$,
  \item[$\vec{j}_k$] the number density flux of the particles of species $k$,
  \item[$\Phi$] the electrostatic potential,
  \item[$\rho_{\mathrm{fl}}$] the mass density of the fluid,
  \item[$\vec{v}_{\mathrm{fl}}$] the advective velocity of the fluid,
\end{description}
and input parameters
\begin{description}[itemsep=0cm,labelindent=1.5em,leftmargin=4.5em,style=nextline]
  \item[$D_k$] the diffusion constant of species $k$,
  \item[$\nu_k$] the mobility of species $k$,
  \item[$q_k$] the charge of a single particle of species $k$,
  \item[$\lb$] the Bjerrum length,
  \item[$\kT$] the thermal energy given by the product of Boltzmann's constant
  $k_\text{B}$\\and the temperature $T$,
  \item[$\eta$] the dynamic viscosity of the fluid,
  \item[$\eta_{\text{\,b}}$] the bulk viscosity of the fluid.
\end{description}
The temperature $T$, and diffusion constants $D_k$ and mobilities $\nu_k$ of
individual species are linked through the Einstein-Smoluchowski relation $D_k /
\nu_k = \kT$. The system of equations described in
Eqs.~\eqref{eq:ek-model-continuity}-\eqref{eq:ek-model-continuity-fl}, combining 
diffusion-advection, electrostatics, and hydrodynamics is conventionally 
referred to as the \textit{Electrokinetic Equations}.

\todo{Complete in broad strokes the applicability of the
  electrokinetics model. Also mention the difference in temperatures
  between EK and LB species.}

The electrokinetic equations have the following properties:
\begin{itemize}
\item On the coarse time and length scale of the model, the dynamics
  of the particle species can be described in terms of smooth density
  distributions and potentials as opposed to the microscale where
  highly localized densities cause singularities in the potential.
	
  In most situations, this restricts the application of the model to
  species of monovalent ions, since ions of higher valency typically
  show strong condensation and correlation effects -- the localization
  of individual ions in local potential minima and the subsequent
  correlated motion with the charges causing this minima.
	
\item Only the entropy of an ideal gas and electrostatic interactions
  are accounted for. In particular, there is no excluded volume.
	
  This restricts the application of the model to monovalent ions and
  moderate charge densities. At higher valencies or densities,
  overcharging and layering effects can occur, which lead to
  non-monotonic charge densities and potentials, that can not be
  covered by a mean-field model such as Poisson-Boltzmann or this one.
	
  Even in salt free systems containing only counter ions, the
  counter-ion densities close to highly charged objects can be
  overestimated when neglecting excluded volume effects. Decades of
  the application of Poisson-Boltzmann theory to systems of
  electrolytic solutions, however, show that those conditions are
  fulfilled for monovalent salt ions (such as sodium chloride or
  potassium chloride) at experimentally realizable concentrations.
	
\item Electrodynamic and magnetic effects play no role. Electrolytic
  solutions fulfill those conditions as long as they don't contain
  magnetic particles.
	
\item The diffusion coefficient is a scalar, which means there can not
  be any cross-diffusion. Additionally, the diffusive behavior has
  been deduced using a formalism relying on the notion of a local
  equilibrium. The resulting diffusion equation, however, is known to
  be valid also far from equilibrium.
	
\item The temperature is constant throughout the system.
	
\item The density fluxes instantaneously relax to their local
  equilibrium values. Obviously one can not extract information about
  processes on length and time scales not covered by the model, such
  as dielectric spectra at frequencies, high enough that they
  correspond to times faster than the diffusive time scales of the
  charged species.
\end{itemize}

\section{Setup}

\subsection{\label{ssec:ek-init}Initialization}

\begin{essyntax}
  \require{1 or 2 or 3}{electrokinetics}
  \opt{agrid  \var{agrid}}
  \opt{lb_density  \var{lb\_density}}
  \opt{visc  \var{viscosity}}
  \opt{bulk_visc  \var{bulk\_viscosity}}
  \opt{friction   \var{gamma} } 
  \opt{gamma_odd  \var{gamma\_odd}}
  \opt{gamma_even  \var{gamma\_even}}
  \opt{T  \var{T}}
  \opt{bjerrum_length  \var{bjerrum\_length}}
  \opt{advection  \var{advection}}
  \opt{fluid-coupling  \var{fluid-coupling}}
  \opt{\require{4}{electrostatics_coupling}}
  \begin{features}
  \required[1]{ELECTROKINETICS}
  \required[2]{EK_BOUNDARIES}
  \required[3]{EK_REACTIONS}
  \required[4]{EK_ELECTROSTATIC_COUPLING}
  \end{features}
\end{essyntax}
The \lit{electrokinetics} command initializes the LB fluid with a given
set of parameters, and it is very similar to the \es{} Lattice-Boltzmann 
\lit{lbfluid} command in set-up. We therefore refer the reader to 
Chapter~\ref{sec:lb} for details on the implementation of LB in \es{} and 
describe only the major differences here. 

The first major difference with the LB implementation is that the 
electrokinetics set-up is a Graphics Processing Unit (GPU) only implementation. 
There is no Central Processing Unit (CPU) version, and at this time there are no
plans to make a CPU version available in the future. To use the electrokinetics
features it is therefore imperative that your computer contains a CUDA capable
GPU which is sufficiently modern. 

To set up a proper LB fluid using the \lit{electrokinetics} command one has to
specify at least the following options: \var{agrid}, \var{lb\_density}, \var{visc},
\var{friction}, \var{T}, and \var{bjerrum\_length}. The other options can be used to 
modify the behavior of the LB fluid. Note that the \lit{electrokinetics} command 
does not allow the user to set the time step parameter \lit{tau} as is the case for
the \lit{lbfluid} command, this parameter is instead taken directly from the input
of the \lit{setmd} \texttt{t\_step} command. The LB \emph{mass density} is set 
independently from the electrokinetic \emph{number densities}, since the LB fluid 
serves only as a medium through which hydrodynamic interactions are propagated, 
as will be explained further in the next paragraph. If no \var{lb\_density} is
specified, then our algorithm assumes \var{lb\_density} = 1.0. The two `new'
parameters are \var{T} the temperature at which the diffusive species are simulated
and \var{bjerrum\_length} the Bjerrum length associated with the electrostatic 
properties of the medium. See the above description of the electrokinetic 
equations for an explanation of the introduction of a temperature, which does 
not come in directly via a thermostat that produces thermal fluctuations.

\lit{advection} can be set to \emph{on} or \emph{off}. It controls whether there
should be an advective contribution to the diffusive species' fluxes. Default is
\emph{on}.

\lit{fluid-coupling} can be set to \emph{friction} or \emph{estatics}. This option
determines the force term acting on the fluid. The former specifies the force term
to be the sum of the species fluxes divided by their respective mobilities while the
latter simply uses the electrostatic force density acting on all species. Note that 
this switching is only possible for the linkcentered stencil. For all other stencils,
this choice is hardcoded. The default is \emph{friction}.

\lit{electrostatics_coupling} enables the action of the electrostatic potential
due to the electrokinetics species and charged boundaries on the MD particles.
The forces on the particles are calculated by interpolation from the electric
field which is in turn calculated from the potential via finite differences.
This only includes interactions between the species and boundaries and MD
particles, not between MD particles and MD particles. To get complete electrostatic
interactions a particles Coulomb method like Ewald or P3M has to be activate too.

\subsection{\label{ssec:ek-diff-species}Diffusive Species}

\begin{essyntax}
  \require{1 or 2 or 3}{electrokinetics}
  \var{species\_number}
  \opt{density  \var{density}}
  \opt{D  \var{D}}
  \opt{valency  \var{valency}}
  \opt{ext_force  \var{f_x} \var{f_y} \var{f_z}}
  \begin{features}
  \required[1]{ELECTROKINETICS}
  \required[2]{EK_BOUNDARIES}
  \required[3]{EK_REACTIONS}
  \end{features}
\end{essyntax}
The \lit{electrokinetics} command followed by an integer \var{species\_number}
(in the range 0 to 10) and several options can be used to initialize the 
diffusive species. Here the options specify: the number density
\var{density}, the diffusion coefficient \var{D}, the valency of the particles 
of that species \var{valency}, and an optional external (electric) force which
is applied to the diffusive species. As mentioned before, the LB density is 
completely decoupled from the electrokinetic densities. This has the advantage
that greater freedom can be achieved in matching the internal parameters to an
experimental system. Moreover, it is possible to choose parameters for which
the LB is more stable. The LB fluid must already be (partially) set up using the 
\lit{electrokinetics} \var{agrid} ... command, before the diffusive species can
be initialized. The variables \var{density}, \var{D}, and \var{valency} must be 
set to properly initialize the diffusive species; the \var{ext\_force} is
optional.

\subsection{\label{ssec:ek-boundaries}Boundaries}

\begin{essyntax}
  \require{1 or 2 or 3}{electrokinetics}
  \require{2}{boundary} 
  \opt{charge_density \var{charge\_density}}
  \opt{shape \var{shape\_args}}
  \begin{features}
  \required[1]{ELECTROKINETICS}
  \required[2]{EK_BOUNDARIES}
  \required[3]{EK_REACTIONS}
  \end{features}
\end{essyntax}
The \lit{boundary} command allows one to set up (internal or external) boundaries
for the electrokinetics algorithm in much the same way as the \lit{lbboundary} 
command is used for the LB fluid. The major difference with the LB command is 
given by the option \var{charge\_density}, with which a boundary can be endowed 
with a volume charge density. To create a surface charge density, a combination 
of two oppositely charged boundaries, one inside the other, can be used. 
However, care should be taken to maintain the surface charge density when the
value of \var{agrid} is changed. Currently, the following \var{shape}s are 
available: wall, sphere, cylinder, rhomboid, pore, stomatocyte, hollow_cone, and spherocylinder. We refer to 
the documentation of the \lit{lbboundary} command (Chapter~\ref{sec:lb}) for
information on the options \var{shape\_args} associated to these shapes. In
order to properly set up the boundaries, the \var{charge\_density} and relevant
\var{shape\_args} must be specified.

\section{\label{ssec:ek-output}Output}

\subsection{\label{ssec:ek-output-fields}Fields}

\begin{essyntax}
  \require{1 or 2 or 3}{electrokinetics}
  print 
  \require{1 or 2}{\var{property}}
  \opt{vtk}
  \var{filename}
  \begin{features}
  \required[1]{ELECTROKINETICS}
  \required[2]{EK_BOUNDARIES}
  \required[3]{EK_REACTIONS}
  \end{features}
\end{essyntax}
The print parameter of the \lit{electrokinetics} command enables simple 
visualization of simulation data. A property of the fluid field can be exported
into a file with name \var{filename} in one go. Currently, supported values of
the parameter \var{property} are: \var{density}, \var{velocity}, 
\var{potential}, and \var{boundary}, which give the LB fluid density, the LB 
fluid velocity, the electrostatic potential, and the location and type of the
boundaries, respectively. The boundaries can only be printed when the 
\texttt{EK_BOUNDARIES} is compiled in. The additional option \lit{vtk} can be
used to directly export in the vtk format. The vtk format is readable
by visualization software such as paraview\footnote{http://www.paraview.org/}
and mayavi2\footnote{http://code.enthought.com/projects/mayavi/}. If the 
\opt{vtk} option is not specified, a gnuplot readable data file will be exported.

\begin{essyntax}
  \require{1 or 2 or 3}{electrokinetics}
  \var{species\_number} 
  print 
  \var{property} 
  \opt{vtk}
  \var{filename}
  \begin{features}
  \required[1]{ELECTROKINETICS}
  \required[2]{EK_BOUNDARIES}
  \required[3]{EK_REACTIONS}
  \end{features}
\end{essyntax}
This print statement is similar to the above command. It enables the export of
diffusive species properties, namely: \var{density} and \var{flux}, which 
specify the number density and flux of species \var{species\_number},
respectively.

\subsection{\label{ssec:ek-local-quantities}Local Quantities}

\begin{essyntax}
  \require{1 or 2 or 3}{electrokinetics}
  node \var{x} \var{y} \var{z} 
  velocity
  \begin{features}
  \required[1]{ELECTROKINETICS}
  \required[2]{EK_BOUNDARIES}
  \required[3]{EK_REACTIONS}
  \end{features}
\end{essyntax}
The \lit{node} option of the \lit{electrokinetics} command allows one to output 
the value of a quantity on a single LB node. The node is addressed using three
integer values which run from 0 to \var{dim\_x}/\var{agrid},
\var{dim\_y}/\var{agrid}, and \var{dim\_z}/\var{agrid}, respectively. Thus far,
only the velocity of the LB fluid can be printed in the standard electrokinetics
implementation. For other quantities the \lit{lbnode} command may be used. 

\begin{essyntax}
  \require{1 or 2 or 3}{electrokinetics}
  \var{species\_number}
  node \var{x} \var{y} \var{z}
  density
  \begin{features}
  \required[1]{ELECTROKINETICS}
  \required[2]{EK_BOUNDARIES}
  \required[3]{EK_REACTIONS}
  \end{features}
\end{essyntax}
This command can be used to output the number density of the
\var{species\_number}-th diffusive species on a single LB node.

\section{\label{ssec:ek-checkpointing}Checkpointing}

\begin{essyntax}
  \variant{1}\require{1 or 2 or 3}{electrokinetics} checkpoint save \var{filename}
  \variant{1}\require{1 or 2 or 3}{electrokinetics} checkpoint load \var{filename}
  \begin{features}
  \required[1]{ELECTROKINETICS}
  \required[2]{EK_BOUNDARIES}
  \required[3]{EK_REACTIONS}
  \end{features}
\end{essyntax}
Variant \variant{1} writes the species density fields as well as all necessary 
LB fields into two files called \var{filename.ek} and \var{filename.lb}.
Variant \variant{2} reads these files back and restores the fields. All 
algorithm parameters must be set via the simulation script, as they are not 
part of the checkpointed data.

The format of the checkpoint is binary and no special care is taken with 
respect to the specific binary layout of the machine.

\section{Catalytic Reaction}

\subsection{Concept}

The electrokinetics solver implemented in \es{} can be used to
simulate a system, for which in addition to the electrokinetic
equations, there is a (local) catalytic reaction which converts one
species into another.

If you are interested in using this implementation in \es{} for
scientific purposes, please contact J.~de~Graaf before you start your
project.

Currently, a linear reaction is implemented which converts one species into two others, in
order to model the catalytic decomposition of hydrogen peroxide in the presence
of a platinum catalyst: $2 \mathrm{H}_{2}\mathrm{O}_{2} \rightarrow 
2 \mathrm{H}_{2}\mathrm{O} + \mathrm{O}_{2}$. The decomposition of 
$\mathrm{H}_{2}\mathrm{O}_{2}$ is in reality more complicated than the linear 
reaction introduced here, since it is assumed to proceed via several intermediate complexed-states, 
but our model can be thought of as modeling the rate-limiting step.
If we assume that there are three non-ionic species with number densities
$n_{k}$, where $n_{0} = [ \mathrm{H}_{2}\mathrm{O}_{2} ]$,
$n_{1} = [ \mathrm{H}_{2}\mathrm{O} ]$, and $n_{2} = [ \mathrm{O}_{2} ]$, 
then we can write the (electro)kinetic equations for this system as

\begin{eqnarray}
\label{eq:ek-reaction-continuity} \frac{\partial n_k}{\partial t} & = & -\, \nabla \cdot \vec{j}_k +\, f_{k} c n_{k} \vphantom{\left(\frac{\partial}{\partial}\right)} ; \\
\label{eq:ek-reaction-fluxes} \vec{j}_{k} & = & -D_k \nabla n_k + n_k \vec{v}_{\mathrm{fl}} \vphantom{\left(\frac{\partial}{\partial}\right)} ; \\
\nonumber \left(\frac{\partial \vec{v}_{\mathrm{fl}}}{\partial t} + \vec{v}_{\mathrm{fl}} \cdot \vec{\nabla} \vec{v}_{\mathrm{fl}} \right) \rho_\mathrm{fl} & = & -\kT \, \sum_{k} \nabla n_k   \\
\label{eq:ek-reaction-velocity} & & +\, \eta \vec{\Delta} \vec{v}_{\mathrm{fl}} + (\eta / 3 + \eta_{\text{\,b}}) \nabla (\nabla \cdot \vec{v}_{\mathrm{fl}}) \vphantom{\left(\frac{\partial}{\partial}\right)} ; \\
\label{eq:ek-reaction-continuity-fl} \frac{\partial \rho_\mathrm{fl}}{\partial t} & = & -\,\nabla\cdot\left( \rho_\mathrm{fl} \vec{v}_{\mathrm{fl}} \right) \vphantom{\left(\frac{\partial}{\partial}\right)} ,
\end{eqnarray}
which define relations between the following observables
\begin{description}[itemsep=0cm,labelindent=1.5em,leftmargin=4.5em,style=nextline]
  \item[$n_k$] the number density of the particles of species $k$,
  \item[$\vec{j}_k$] the number density flux of the particles of species $k$,
  \item[$\rho_{\mathrm{fl}}$] the mass density of the fluid,
  \item[$\vec{v}_{\mathrm{fl}}$] the advective velocity of the fluid,
\end{description}
and input parameters
\begin{description}[itemsep=0cm,labelindent=1.5em,leftmargin=4.5em,style=nextline]
  \item[$D_k$] the diffusion constant of species $k$,
  \item[$\kT$] the thermal energy given by the product of Boltzmann's constant
  $k_\text{B}$\\and the temperature $T$,
  \item[$\eta$] the dynamic viscosity of the fluid,
  \item[$\eta_{\text{\,b}}$] the bulk viscosity of the fluid,
  \item[$f_{k}$] the reaction constant $f_{0} \equiv -1$, $f_{1} = 1$ and $f_{2} = 0.5$ for the above reaction,
  \item[$c$] the reaction rate.
\end{description}
In this set of equations we have fully decoupled the number densities and the
fluid mass density. N.B. We have set the initial fluid mass density is not necessarily 
equal to the sum of the initial species number densities. This means that 
some care needs to be taken in the interpretation of the results obtained using
this feature. In particular, the solution of the Navier-Stokes equation exclusively
models the momentum transport through the (multicomponent) fluid, while the diffusive
properties of the individual chemical species are handled by Eqs.~\eqref{eq:ek-reaction-continuity}
and~\eqref{eq:ek-reaction-fluxes}.

It is important to note that to ensure mass conservation the reaction must satisfy:
\begin{equation}
\label{eq:ek-mass-balance} \sum_{k} f_{k} m_{k} = 0 ,
\end{equation}
where $m_{k}$ is the molecular mass of a reactive species. Unfortunately, the current
electrokinetic implementation does not conserve mass flux locally. That is to say, the
LB fluid is compressible and the sum of the fluxes of the three species is not equal to
zero in the frame co-moving with the advective fluid velocity. It is therefore debatable
whether it is necessary to impose Eq.~\eqref{eq:ek-mass-balance}, since the EK algorithm
itself does not conserve mass density. However, we strived to be as accurate as
possible and in future versions of the EK algorithm the lack of incompressibility will 
be addressed. 

The reaction is specified by the second term on the right-hand side of 
Eq.~\eqref{eq:ek-reaction-continuity}. It is important to note that this term
can be set locally, as opposed to the other terms in the equation system 
Eqs.~\eqref{eq:ek-reaction-continuity}-\eqref{eq:ek-reaction-continuity-fl}, in
our implementation, as will become clear in the following. This has the
advantage that catalytic surfaces may be modeled.

\subsection{\label{ssec:ek-reac-init}Initialization and Geometry Definition}

\begin{essyntax}
  \require{1 or 2 or 3}{electrokinetics}
  \require{3}{reaction}
  \opt{reactant_index \var{reactant\_index}}
  \opt{product0_index \var{product0\_index}}
  \opt{product1_index \var{product1\_index}}
  \opt{reactant_resrv_density \var{reactant\_resrv\_density}}
  \opt{product0_resrv_density \var{product0\_resrv\_density}}
  \opt{product1_resrv_density \var{product1\_resrv\_density}}
  \opt{reaction_rate \var{reaction\_rate}}
  \opt{mass_reactant \var{mass\_reactant}}
  \opt{mass_product0 \var{mass\_product0}}
  \opt{mass_product1 \var{mass\_product1}}
  \opt{reaction_fraction_pr_0 \var{reaction\_fraction\_pr\_0}}
  \opt{reaction_fraction_pr_1 \var{reaction\_fraction\_pr\_1}}
  \begin{features}
  \required[1]{ELECTROKINETICS}
  \required[2]{EK_BOUNDARIES}
  \required[3]{EK_REACTIONS}
  \end{features}
\end{essyntax}
The \lit{electrokinetics reaction} command is used to set up the catalytic 
reaction between three previously defined the diffusive species, of which the i
identifiers are given by \var{reactant\_index}, \var{product0\_index}, and 
\var{product1\_index}, respectively. In the 1:2 reaction, these fulfill the role
of the reactant and the two products, as indicated by the naming convention. For
each species a reservoir (number) density must be set, given by the variables
\var{reactant\_resrv\_density}, \var{product0\_resrv\_density}, and
\var{product1\_resrv\_density}, respectively. These reservoir densities
correspond to the initial number densities associated with the reactive species.
The reservoir densities, in tandem with reservoir nodes, see below, can be used 
to keep the reaction from depleting all the reactant in the simulation box. The 
\var{reaction\_rate} variable specifies the speed at which the reaction proceeds.
The three masses (typically given in the atomic weight equivalent) are used to 
determine the total mass flux provided by the reaction, as described above, and 
are also used to check whether the reaction ratios that are given satisfy the 
chemical requirement of mass conservation. Finally, the parameters
\var{reaction\_fraction\_pr\_0} and \var{reaction\_fraction\_pr\_1} specify
what fractions of the product are generated when a given quantity of reactant is
catalytically converted. To use a chemical reaction, all options for the 
\lit{electrokinetics reaction} command must be specified.

\begin{essyntax}
  \require{1 or 2 or 3}{electrokinetics}
  \require{3}{reaction}
  \require{3}{region}
  \opt{reaction_type \var{reaction\_type}}
  \opt{shape \var{shape\_args}}
  \begin{features}
  \required[1]{ELECTROKINETICS}
  \required[2]{EK_BOUNDARIES}
  \required[3]{EK_REACTIONS}
  \end{features}
\end{essyntax}
The \lit{region} option of the \lit{electrokinetics reaction} command allows one 
to set up regions in which the reaction takes place with the help of the 
constraints that are available to set up boundaries. The integer value 
\var{reaction\_type} can be used to select the reaction: 0 no reaction takes
place for this region, 1 the catalytic reaction takes place in this region, and 
2 the region functions as a reservoir, wherein the species densities are reset
to their initial (or reservoir) concentrations. The rest of the command follows 
the same format of the \lit{electrokinetics boundary} command. Currently, the 
following \var{shape}s are available: box, wall, sphere, cylinder, rhomboid, 
pore, stomatocyte, hollow_cone, and spherocylinder. The box shape is a \lit{region} specific command, which
can be used to set the entire simulation box to a specific reaction value. To
use the \lit{electrokinetics reaction region} command, one must first set up
a reaction, as described above. To successfully specify a region all the 
relevant arguments that go with the shape constraints must be provided.

\subsubsection{\label{sssec:ek-pdb-parse}Parsing PDB Files}

\begin{essyntax}
  \require{1 or 2 or 3}{electrokinetics}
  \require{2}{pdb-parse}
  \var{pdb\_filename}
  \var{itp\_filename}
  \begin{features}
  \required[1]{ELECTROKINETICS}
  \required[2]{EK_BOUNDARIES}
  \required[3]{EK_REACTIONS}
  \end{features}
\end{essyntax}
The \lit{electrokinetics pdb-parse} feature allows the user to parse simple PDB files, a file format introduced by the protein database to encode molecular structures. Together with a topology file (here \var{itp\_filename}) the structure gets interpolated to the \lit{electrokinetics} grid. For the input you will need to prepare a PDB file with a \lit{gromacs} force field to generate the topology file. Normally the PDB file extension is \lit{.pdb}, the topology file extension is \lit{.itp}. Obviously the PDB file is placed instead of \var{pdb\_filename} and the topology file instead of \var{itp\_filename}. \todo{At the moment this fails badly, if you try to parse incorrectly formatted files. This will be fixed in the future.}

\subsection{\label{ssec:ek-reac-output}Reaction-Specific Output}

\begin{essyntax}
  \require{1 or 2 or 3}{electrokinetics}
  print 
  \require{3}{\var{property}}
  \opt{vtk} 
  \var{filename}
  \begin{features}
  \required[1]{ELECTROKINETICS}
  \required[2]{EK_BOUNDARIES}
  \required[3]{EK_REACTIONS}
  \end{features}
\end{essyntax}
The print parameter of the \lit{electrokinetics} command can be used in
combination with the \texttt{EK\_REACTION} feature to give advanced output 
options. Currently, supported values of the parameter \var{property} are: 
\var{pressure} and \var{reaction\_tags}, which give the 
location and type of the reactive regions and the ideal-gas pressure coming from 
the diffusive species, respectively. To use this command a reaction must be set up.

%%% Local Variables:
%%% mode: latex
%%% TeX-master: "ug"
%%% End:
