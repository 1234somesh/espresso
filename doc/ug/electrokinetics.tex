\newcommand{\lb}{l_\mathrm{B}}
\newcommand{\kT}{k_\mathrm{B}T}

\chapter{\label{sec:electrokinetics}Electrokinetics}
\newescommand{electrokinetics}

The electrokinetics setup in \es{} allows for the description of electro-hydrodynamic
systems on the level of ion density distributions coupled to a Lattice-Boltzmann 
fluid interacting with explicit charged particles. In the following paragraph we 
briefly explain the electrokinetic model implemented in \es{}, before we come to
the description of the interface.

\section{Electrokinetic Equations}

In the electrokinetics code we solve the following system of coupled continuity,
diffusion-advection, Poisson, and Navier-Stokes equations:
\begin{eqnarray}
\label{eq:ek-model-continuity} \frac{\partial \rho_k}{\partial t} & = & -\, \nabla \cdot \vec{j}_k \vphantom{\left(\frac{\partial}{\partial}\right)}; \\
\label{eq:ek-model-fluxes} \vec{j}_{k} & = & -D_k \nabla \rho_k - \nu_k \, q_k \rho_k\, \nabla \Phi + \rho_k \vec{v} \vphantom{\left(\frac{\partial}{\partial}\right)}; \\
\label{eq:ek-model-poisson} \Delta \Phi & = & -4 \pi \, \lb \, \kT \sum_k q_k \rho_k \vphantom{\left(\frac{\partial}{\partial}\right)}; \\
\label{eq:ek-model-velocity} \left(\frac{\partial \vec{v}}{\partial t} + \vec{v} \cdot \vec{\nabla} \vec{v} \right) \rho_\text{fl} & = & -\kT \, \nabla \rho_\text{fl} - q_k \rho_k \nabla \Phi + \eta \vec{\Delta} \vec{v} + (\eta / 3 + \eta_{\text{\,b}}) \nabla (\nabla \cdot \vec{v}); \qquad \\
\label{eq:ek-model-continuity-fl} \frac{\partial \rho_\text{fl}}{\partial t} & = & -\,\nabla\cdot\left( \rho_\text{fl} \vec{v} \right) \vphantom{\left(\frac{\partial}{\partial}\right)}, 
\end{eqnarray}
which define relations between the following observables
\begin{description}[itemsep=0cm,labelindent=1.5em,leftmargin=4.5em,style=nextline]
  \item[$\rho_k$] the number density of the particles of species $k$,
  \item[$\vec{j}_k$] the number density flux of the particles of species $k$,
  \item[$\Phi$] the electrostatic potential,
  \item[$\vec{v}$] the advective velocity,
\end{description}
and input parameters
\begin{description}[itemsep=0cm,labelindent=1.5em,leftmargin=4.5em,style=nextline]
  \item[$D_k$] the diffusion constant of species $k$,
  \item[$\nu_k$] the mobility of species $k$,
  \item[$q_k$] the charge of a single particle of species $k$,
  \item[$\lb$] the Bjerrum-length,
  \item[$\kT$] the thermal energy given by the product of Boltzmann's constant
  $k_\text{B}$\\and the temperature $T$,
  \item[$\eta$] the dynamic viscosity of the fluid,
  \item[$\eta_{\text{\,b}}$] the bulk viscosity of the fluid.
\end{description}
The temperature $T$, and diffusion constants $D_k$ and mobilities $\nu_k$ of
individual species are linked through the Einstein-Smoluchowski relation $D_k /
\nu_k = \kT$. The system of equations described in
Eqs.~\eqref{eq:ek-model-continuity}-\eqref{eq:ek-model-continuity-fl}, combining 
diffusion-advection, electrostatics, and hydrodynamics is conventionally 
referred to as the \textit{Electrokinetic Equations}.

\textbf{TODO complete broad borders on the applicability of the model + difference in temperatures between EK and LB pieces.}

The electrokinetic equations 

\begin{itemize}
	\item On the coarse time and length scale of the model, the dynamics of the
	particle species can be described in terms of smooth density distributions and
	potentials as opposed to the microscale where highly localized densities cause
	singularities in the potential.
	
	In most situations, this restricts the application of the model to species of
	monovalent ions, since ions of higher valency typically show strong
	condensation and correlation effects -- the localization of individual ions in
	local potential minima and the subsequent correlated motion with the charges
	causing this minima.
	
	\item Only the entropy of an ideal gas and electrostatic interactions are
	accounted for. In particular, there is no excluded volume.
	
	This restricts the application of the model to monovalend ions and moderate
  charge densities. At higher valencies or densities, overcharging and layering
  effects can occur, which lead to non-monotonic charge densities and potentials,
  that can not be covered by a mean-field model such as Poisson-Boltzmann or this
  one.
	
	Even in salt free systems containing only counterions, the counterion
	densities close to highly charged objects can be overestimated when neglecting
	excluded volume effects. Decades of the application of Poisson-Boltzmann
	theory to systems of electrolytic solutions, however, show that those
	conditions are fulfilled for monovalent salt ions (such as sodium chloride or
	potassium chloride) at experimentally realizable concentrations.
	
	\item Electrodynamic and magnetic effects play no role. Electrolytic solutions
	fulfill those conditions as long as they don't contain magnetic particles.
	
	\item The diffusion coefficient is a scalar, which means there can not be any
	cross-diffusion. Additionally, the diffusive behavior has been deduced using
	a formalism relying on the notion of a local equilibrium. The resulting
	diffusion equation, however, is known to be valid also far from equilibrium.
	
	\item The temperature is constant throughout the system.
	
	\item The density fluxes instantaneously relax to their local equilibrium
	values. Obviously one can not extract information about processes on length
	and time scales not covered by the model, such as dielectric spectra at
	frequencies, high enough that they correspond to times faster than the
	diffusive time scales of the charged species.
\end{itemize}

\section{Setup}

\subsection{\label{ssec:ek-init}Initialization}

\begin{essyntax}
  electrokinetics
  \require{1 or 2 or 3}{\opt{agrid  \var{agrid}}}
  \require{1 or 2 or 3}{\opt{dens  \var{density} }}
  \require{1 or 2 or 3}{\opt{visc  \var{viscosity}}}
  \require{1 or 2 or 3}{\opt{bulk_visc  \var{bulk\_viscosity}}}
  \require{1 or 2 or 3}{\opt{friction   \var{gamma} } }
  \require{1 or 2 or 3}{\opt{gamma_odd  \var{gamma\_odd}}}
  \require{1 or 2 or 3}{\opt{gamma_even  \var{gamma\_even}}}
  \require{1 or 2 or 3}{\opt{T  \var{T}}}
  \require{1 or 2 or 3}{\opt{bjerrum_length  \var{bjerrum\_length}}}
  \begin{features}
  \required[1]{ELECTROKINETICS}
  \required[2]{EK_BOUNDARIES}
  \required[3]{EK_REACTIONS}
  \end{features}
\end{essyntax}
The \lit{electrokinetics} command initializes the LB fluid with a given
set of parameters, which in set-up is very similar to the Lattice-Boltzmann 
\lit{lbfluid} command. We therefore refer the reader to that chapter for any 
details and describe only the major differences here. 

The \lit{electrokinetics} command does not allow for a setup of the parameter 
\lit{tau} the time step, which is instead taken directly from the \lit{setmd}
\texttt{t\_step} command. Two additional parameters are however specified
\lit{T} the temperature at which the diffusive species are simulated and 
\lit{bjerrum\_length} the bjerrum length associated with the electrostatic 
properties of the medium.

\subsection{\label{ssec:ek-diff-species}Diffusive Species}

\begin{essyntax}
  electrokinetics \var{species\_number}
  \require{1 or 2 or 3}{\opt{density  \var{density}}}
  \require{1 or 2 or 3}{\opt{D  \var{D} }}
  \require{1 or 2 or 3}{\opt{valency  \var{valency}}}
  \require{1 or 2 or 3}{\opt{ext_force  \var{f_x} \var{f_y} \var{f_z}}}
  \begin{features}
  \required[1]{ELECTROKINETICS}
  \required[2]{EK_BOUNDARIES}
  \required[3]{EK_REACTIONS}
  \end{features}
\end{essyntax}
The \lit{electrokinetics} command with a specified integer \var{species\_number}
initializes the diffusive species, where the options specify: the number density
\var{density}, the diffusion coefficient \var{D}, the valency of the particles 
of that species \var{valency}, and an optional external (electric) force which is
applied to the diffusive species. 

\subsection{\label{ssec:ek-boundaries}Boundaries}

\begin{essyntax}
  electrokinetics boundary 
  \require{2}{\opt{charge_density  \var{charge\_density}}} \var{shape} \var{shape\_args}
  \begin{features}
  \required[1]{ELECTROKINETICS}
  \required[2]{EK_BOUNDARIES}
  \required[3]{EK_REACTIONS}
  \end{features}
\end{essyntax}

The \lit{boundary} command allows one to set up (internal or external) boundaries
for the electrokinetics algorithm in much the same way as the \lit{lbboundary} 
command is used. The major difference with the LB command is given by the inclusion
of the option \var{charge\_density}, with which a boundary can be endowed with a
volume charge density. To create a surface charge density, a combination of two
oppositely charged boundaries, one inside the other, can be used. However, care
should be taken to maintain the surface charge density when the value of 
\var{agrid} is changed. Currently, the following \var{shape}s are available:
wall, sphere, cylinder, rhomboid, pore, and stomatocyte. We refer to the 
documentation of the \lit{lbboundary} command for information on the options
associated to these shapes.

\section{\label{ssec:ek-output}Output}

\subsection{\label{ssec:ek-output-fields}Fields}

\begin{essyntax}
  electrokinetics print 
  \require{1 or 2}{\var{property}}
  \opt{vtk} \var{filename} [\var{filename}]
  \begin{features}
  \required[1]{ELECTROKINETICS}
  \required[2]{EK_BOUNDARIES}
  \required[3]{EK_REACTIONS}
  \end{features}
\end{essyntax}
The print parameter of the \lit{electrokinetics} command is a feature to simplify 
visualization. It allows for the export a property of the fluid field into a 
file with name \var{filename} in one go. Currently supported values of the 
parameter \var{property} are: \var{density}, \var{velocity}, \var{potential}, and
\var{boundary}, which give the LB fluid density, the LB fluid velocity, the 
electrostatic potential, and the location and type of the boundaries, respectively.
The boundaries can only be printed when the \texttt{EK_BOUNDARIES} is compiled in.
The additional option \lit{vtk} enables export in the vtk format which is readable
by visualization software such as paraview\footnote{http://www.paraview.org/} or 
mayavi2\footnote{http://code.enthought.com/projects/mayavi/}. If the \lit{vtk} 
option is not specified gnuplot readable data will be exported.

\begin{essyntax}
  electrokinetics \var{species\_number} print \var{property} \opt{vtk} \var{filename} [\var{filename}]
  \begin{features}
  \required[1]{ELECTROKINETICS}
  \required[2]{EK_BOUNDARIES}
  \required[3]{EK_REACTIONS}
  \end{features}
\end{essyntax}
This print statement allows for a similar export of the properties of the 
diffusive species, namely: \var{density} and \var{flux}, which specify the 
number density and species flux of species \var{species\_number}, respectively.

\subsection{\label{ssec:ek-local-quantities}Local Quantities}

\begin{essyntax}
  electrokinetics node \var{x} \var{y} \var{z} velocity
  \begin{features}
  \required[1]{ELECTROKINETICS}
  \required[2]{EK_BOUNDARIES}
  \required[3]{EK_REACTIONS}
  \end{features}
\end{essyntax}
The \lit{node} option of the \lit{electrokinetics} command allows one to output 
the value of a quantity on a single LB node, thus far only the velocity has been
implemented. For other quantities the \lit{lbnode} command may be used. 

\begin{essyntax}
  electrokinetics \var{species\_number} node \var{x} \var{y} \var{z} density
  \begin{features}
  \required[1]{ELECTROKINETICS}
  \required[2]{EK_BOUNDARIES}
  \required[3]{EK_REACTIONS}
  \end{features}
\end{essyntax}
This command can be used to output the number density of the \var{species\_number}
diffusive species on a single LB node.

\section{Catalytic Reaction}

\subsection{Concept}

\textbf{TODO: Describe the catalytic reaction and the `mass solution'.}

The electrokineticssolved implemented in \es{} can be used to simulate a system,
where in addition to the electrokinetic equations, there is a (local) catalytic
reaction which converts one species into another. This gives rise to a modified
set of reaction-diffusion-advection equations. 

\subsection{\label{ssec:ek-reac-init}Initialization and Geometry Definition}

\begin{essyntax}
  electrokinetics 
  \require{3}{reaction}
  \opt{reactant_index \var{reactant\_index}}
  \opt{product0_index \var{product0\_index}}
  \opt{product1_index \var{product1\_index}}
  \opt{reactant_resrv_density \var{reactant\_resrv\_density}}
  \opt{product0_resrv_density \var{product0\_resrv\_density}}
  \opt{product1_resrv_density \var{product1\_resrv\_density}}
  \opt{reaction_rate \var{reaction_rate}}
  \opt{mass_reactant \var{mass\_reactant}}
  \opt{mass_product0 \var{mass\_product0}}
  \opt{mass_product1 \var{mass\_product1}}
  \opt{reaction_fraction_pr_0 \var{reaction\_fraction\_pr\_0}}
  \opt{reaction_fraction_pr_1 \var{reaction\_fraction\_pr\_1}}
  \begin{features}
  \required[1]{ELECTROKINETICS}
  \required[2]{EK_BOUNDARIES}
  \required[3]{EK_REACTIONS}
  \end{features}
\end{essyntax}
The \lit{electrokinetics reaction} command is used to set up the catalytic reaction
between three previously defined the diffusive species, of which the identifiers 
are given by \var{reactant\_index}, \var{product0\_index}, and \var{product1\_index}
respectively. In the 1:2 reaction, these fulfill the role of the reactant and 
the two products, as indicated by the naming convention. For each species a 
reservoir (number) density must be set, given by the variables
\var{reactant\_resrv\_density}, \var{product0\_resrv\_density}, and
\var{product1\_resrv\_density}, respectively. These reservoir densities, in tandem
with reservoir nodes, see below, can be used to keep the reaction from depleting
all the reactant in the simulation box. The \var{reaction_rate} variable specifies
the speed at which the reaction proceeds. The three masses (typically given in 
the atomic weigth equivalent) are used to determine the total mass flux provided
by the reaction, and are used to check whether the reaction that is given satisfies
the chemical requirement of mass conservation. Finally, the parameters
\var{reaction\_fraction\_pr\_0} and \var{reaction\_fraction\_pr\_1} specify
what fractions of the product are generated when a given quantity of reactant is
catalytically converted.

\begin{essyntax}
  electrokinetics 
  \require{3}{reaction}
  region
  \opt{reaction_type  \var{reaction\_type}}
  \var{shape} \var{shape\_args}
  \begin{features}
  \required[1]{ELECTROKINETICS}
  \required[2]{EK_BOUNDARIES}
  \required[3]{EK_REACTIONS}
  \end{features}
\end{essyntax}
The \lit{region} option of the \lit{electrokinetics reaction} command allows one 
to set up regions in which the reaction takes place with the help of the constraints
that are available to set up boundaries. The integer value \var{reaction\_type}
can be used to select the reaction: 0 no reaction, 1 reaction, and 2 reservoir.
the rest of the command follows the same format of the \lit{electrokinetics boundary}
command. Currently, the following \var{shape}s are available: box,
wall, sphere, cylinder, rhomboid, pore, and stomatocyte. The box shape is a 
\lit{region} specific command, to set the entire box to a specific reaction value.

\subsection{\label{ssec:ek-reac-output}Reaction-Specific Output}

\begin{essyntax}
  electrokinetics print 
  \require{3}{\var{property}}
  \opt{vtk} \var{filename} [\var{filename}]
  \begin{features}
  \required[1]{ELECTROKINETICS}
  \required[2]{EK_BOUNDARIES}
  \required[3]{EK_REACTIONS}
  \end{features}
\end{essyntax}
The print parameter of the \lit{electrokinetics} command can be used in combination
with the \texttt{EK\_REACTION} feature to give advanced output options. Currently
supported values of the parameter \var{property} are: \var{pressure}, 
\var{reaction\_tags}, and \var{mass\_flux}, which give the the location and type of 
the reactive regions, the ideal-gas pressure coming from the diffusive species and
the total mass flux of the reactive species, respectively.

\begin{essyntax}
  electrokinetics node \var{x} \var{y} \var{z} 
  \require{3}{mass_flux}
  \begin{features}
  \required[1]{ELECTROKINETICS}
  \required[2]{EK_BOUNDARIES}
  \required[3]{EK_REACTIONS}
  \end{features}
\end{essyntax}
In combination with the \texttt{EK\_REACTION} feature the \lit{node} option of the
\lit{electrokinetics} command allows one to output the value mass flux on a single 
LB node.

%%% Local Variables:
%%% mode: latex
%%% TeX-master: "ug"
%%% End:
