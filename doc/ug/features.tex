% Copyright (C) 2010,2012,2013,2014 The ESPResSo project
% Copyright (C) 2002,2003,2004,2005,2006,2007,2008,2009,2010 
%   Max-Planck-Institute for Polymer Research, Theory Group
%  
% This file is part of ESPResSo.
%   
% ESPResSo is free software: you can redistribute it and/or modify it
% under the terms of the GNU General Public License as published by the
% Free Software Foundation, either version 3 of the License, or (at your
% option) any later version.
%  
% ESPResSo is distributed in the hope that it will be useful, but
% WITHOUT ANY WARRANTY; without even the implied warranty of
% MERCHANTABILITY or FITNESS FOR A PARTICULAR PURPOSE.  See the GNU
% General Public License for more details.
%  
% You should have received a copy of the GNU General Public License
% along with this program.  If not, see <http://www.gnu.org/licenses/>.
%
\chapter{Features}
\label{chap:features}

\index{features|mainindex}

This chapter describes the features that can be activated in \es. Even
if possible, it is not recommended to activate all features, because
this will negatively effect \es's performance.

Features can be activated in the configuration header
\texttt{myconfig.h} (see section \vref{sec:myconfig}). Too activate
\texttt{FEATURE}, add the following line to the header file:
\begin{verbatim}
#define FEATURE
\end{verbatim}

\section{General features}

\todo{The list contains all features, but there are tons of docs missing!}

\begin{itemize}
\item \newfeature{PARTIAL\_PERIODIC} By default, all coordinates in \es{}
  are periodic. With \texttt{PARTIAL\_PERIODIC} turned on, the \es{}
  global variable \texttt{periodic} (see section \vref{tcl:setmd})
  controls the periodicity of the individual coordinates. Note that
  this slows the integrator down by around $10-30\%$.
\item \newfeature{ELECTROSTATICS} This switches on the various
  electrostatics algorithms, such as P3M. See section
  \vref{sec:inter-electrostatics} for details on these algorithms.
\item \newfeature{DIPOLES} This activates the dipole-moment property of
  particles; 
  In addition, the various magnetostatics algorithms, such as P3M are switched on.  See
  section \vref{sec:inter-electrostatics} for details on these
  algorithms.  
\item \newfeature{ROTATION} Switch on rotational degrees of freedom
  for the particles, as well as the corresponding quaternion
  integrator.  See section \vref{ssec:particleproperties} for
  details. Note, that when the feature is activated, every particle
  has three additional degrees of freedom, which for example means
  that the kinetic energy changes at constant temperature is twice as
  large.
\item \newfeature{ROTATION\_PER\_PARTICLE} Allows to set whether a
  particle has rotational degrees of freedom.
\item \newfeature{LANGEVIN\_PER\_PARTICLE} Allows to choose the
  Langevin temperature and friction coefficient per particle.
\item \newfeature{ROTATIONAL\_INERTIA}
\item \newfeature{EXTERNAL\_FORCES} Allows to define an arbitrary
  constant force for each particle individually. Also allows to fix
  individual coordinates of particles, \eg keep them at a fixed
  position or within a plane.
\item \newfeature{CONSTRAINTS} Turns on various spatial constraints such
  as spherical compartments or walls. This constraints interact with
  the particles through regular short ranged potentials such as the
  Lennard--Jones potential. See section \vref{tcl:constraint} for
  possible constraint forms.
\item \newfeature{TUNABLE\_SLIP} Switch on tunable slip conditions for
  planar wall boundary conditions. See section \vref{sec:tunableSlip}
  for details.
\item \newfeature{MASS} Allows particles to have individual masses. Note
  that some analysis procedures have not yet been adapted to take the
  masses into account correctly.
\item \newfeature{EXCLUSIONS} Allows to exclude specific short ranged
  interactions within molecules.
\item \newfeature{COMFORCE} Allows to pull apart groups of particles
\item \newfeature{COMFIXED} Allows to fix the center of mass of all particles of a certain type.
\item \newfeature{MOLFORCES} \todo{Docs missing}
\item \newfeature{BOND\_CONSTRAINT} Turns on the RATTLE integrator which
  allows for fixed lengths bonds between particles. \todo{How to use it?}
\item \newfeature{VIRTUAL\_SITES\_COM} Virtual sites are particles,
  the position and velocity of which is not obtained by integrating
  equations of motion. Rather, they are placed using the position (and
  orientation) of other particles. The feature \lit{VIRTUAL_SITES_COM}
  allows to place a virtual particle into the center of mass of a set
  of other particles. See section \ref{sec:virtual} for details.
\item \newfeature{VIRTUAL\_SITES\_RELATIVE} Virtual sites are
  particles, the position and velocity of which is not obtained by
  integrating equations of motion. Rather, they are placed using the
  position (and orientation) of other particles. The feature
  \lit{VIRTUAL_SITES_RELATIVE} allows for rigid arrangements of
  particles. See section \ref{sec:virtual} for details.
\item \newfeature{VIRTUAL\_SITES\_NO\_VELOCITY}
\item \newfeature{VIRTUAL\_SITES\_THERMOSTAT}
\item \newfeature{THERMOSTAT\_IGNORE\_NON\_VIRTUAL}
\item \newfeature{BOND\_VIRTUAL}
\item \newfeature{MODES}
\item \newfeature{ADRESS}
\item \newfeature{METADYNAMICS}
\item \newfeature{LANGEVIN_PER_PARTICLE} Allows to define the temperature and
  friction coefficient for individual particles. See 
  \ref{ssec:particleproperties} for details.
\item \newfeature{CATALYTIC_REACTIONS} Allows the user to define three particle types to be
  reactant, catalyzer, and product. Reactants get converted into products in the vicinity of a catalyst according to a used-defined reaction rate constant. It is also possible to set up a chemical equilibrium reaction between the reactants and products, with another rate constant. See section \ref{sec:Reactions} for details.
\item \newfeature{OVERLAPPED}
\item \newfeature{COLLISION\_DETECTION} Allows particles to be bound oo collision. See section \ref{sec:collision}
\item \newfeature{OLD\_RW\_VERSION} This switches back to the old,
  \emph{wrong} random walk code of the polymer. Only use this if you
  rely on the old behaviour and \emph{know what you are doing}.
\end{itemize}

In addition, there are switches that enable additional features in the
integrator or thermostat:
\begin{itemize}
\item \newfeature{NEMD} Enables the non-equilbrium (shear) MD support
  (see section \vref{sec:NEMD}).
\item \newfeature{NPT} Enables an on--the--fly NPT integration scheme
  (see section \vref{ssec:NPTthermostat}).
\item \newfeature{DPD} Enables the dissipative particle dynamics
  thermostat (see section \vref{sec:DPD}).
\item \newfeature{TRANS\_DPD} Enables the transversal dissipative
  particle dynamics thermostat (see section \vref{sec:transDPD}).
\item \newfeature{INTER\_DPD} Enables the dissipative
  particle dynamics thermostat implemented as an interaction,
  allowing to choose different parameters between different particle
  types (see section \vref{sec:interDPD}).
\item \newfeature{INTER\_RF} \todo{Documentation!}
\item \newfeature{DPD\_MASS\_RED} Enables masses in DPD using reduced,
  dimensionless mass units.
\item \newfeature{DPD\_MASS\_LIN} Enables masses in DPD using absolute
  mass units.
\item \newfeature{LB} Enables the lattice-Boltzmann fluid code (see
  section \vref{sec:lb}).
\item \newfeature{LB_GPU} Enables the lattice-Boltzmann fluid code support for GPU (see
  section \vref{sec:lb}).
\item \newfeature{SHANCHEN} Enables the Shan Chen bicomponent fluid code on the GPU (see
  section \vref{sec:lb}).
\item \newfeature{LB_ELECTROHYDRODYNAMICS} Enables the implicit
  calculation of electro-hydrodynamics for charged particles and salt
  ions in an electric field.
\end{itemize}

\section{Interactions}

The following switches turn on various short ranged interactions (see section
\vref{sec:inter-nb}):
\begin{itemize}
\item \newfeature{TABULATED} Enable support for user--defined
  interactions.
\item \newfeature{LENNARD\_JONES} Enable the Lennard--Jones potential.
\item \newfeature{LENNARD\_JONES\_GENERIC} Enable the generic
  Lennard--Jones potential with configurable exponents and individual
  prefactors for the two terms.
\item \newfeature{LJCOS} Enable the Lennard--Jones potential with a cosine--tail.
\item \newfeature{LJCOS2} Same as LJCOS, but using a slightly different way of
  smoothing the connection to 0.
\item \newfeature{LJ\_ANGLE} Enable the directional Lennard--Jones potential.
\item \newfeature{GAY\_BERNE}
\item \newfeature{HERTZIAN}
\item \newfeature{MOL\_CUT}
\item \newfeature{NO\_INTRA\_NB}
\item \newfeature{MORSE} Enable the Morse potential.
\item \newfeature{BUCKINGHAM} Enable the Buckingham potential.
\item \newfeature{SOFT\_SPHERE} Enable the soft sphere potential.
\item \newfeature{SMOOTH\_STEP} Enable the smooth step potential, a
  step potential with two length scales.
\item \newfeature{BMHTF_NACL} Enable the Born-Meyer-Huggins-Tosi-Fumi potential,
  which can be used to model salt melts.
\end{itemize}

Some of the short range interactions have additional features:
\begin{itemize}
\item \newfeature{LJ\_WARN\_WHEN\_CLOSE} This adds an additional check to
  the Lennard--Jones potentials that prints a warning if particles come
  too close so that the simulation becomes unphysical.
\item \newfeature{OLD_DIHEDRAL} Switch the interface of the dihedral potential
  to its old, less flexible form. Use this for older scripts that are not yet
  adapted to the new interface of the dihedral potential.
\end{itemize}

If you want to use bond-angle potentials (see section
\vref{sec:angle}), you need the followig features.
\begin{itemize}
\item \newfeature{BOND\_ANGLE}
\item \newfeature{BOND\_ANGLEDIST}
\item \newfeature{BOND\_ENDANGLEDIST}
\end{itemize}
\todo{BOND\_ANGLEDIST and BOND\_ENDANGLEDIST are completely undocumented.}

\section{Debug messages}

Finally, there are a number of flags for debugging. The most important
one are
\begin{itemize}
\item \newfeature{ADDITIONAL\_CHECKS} Enables numerous additional checks
  which can detect inconsistencies especially in the cell systems.
  This checks are however too slow to be enabled in production runs.
\item \newfeature{MEM\_DEBUG} Enables an internal memory allocation
  checking system. This produces output for each allocation and
  freeing of a memory chunk, and therefore allows to track down memory
  leaks. This works by internally replacing \texttt{malloc},
  \texttt{realloc} and \texttt{free}.
\end{itemize}

The following flags control the debug output of various sections of
Espresso. You will however understand the output very often only by
looking directly at the code.
\begin{itemize}
\item \newfeature{COMM\_DEBUG} Output from the asynchronous communication code.
\item \newfeature{EVENT\_DEBUG} Notifications for event calls, i.~e. the
  \texttt{on\_?} functions in \texttt{initialize.c}. Useful if some
  module does not correctly respond to changes of e.~g.  global
  variables.
\item \newfeature{INTEG\_DEBUG} Integrator output.
\item \newfeature{CELL\_DEBUG} Cellsystem output.
\item \newfeature{GHOST\_DEBUG} Cellsystem output specific to the
  handling of ghost cells and the ghost cell communication.
\item \newfeature{GHOST\_FORCE\_DEBUG}
\item \newfeature{VERLET\_DEBUG} Debugging of the Verlet list code of the
  domain decomposition cell system.
\item \newfeature{LATTICE\_DEBUG} Universal lattice structure debugging.
\item \newfeature{HALO\_DEBUG}
\item \newfeature{GRID\_DEBUG}
\item \newfeature{PARTICLE\_DEBUG} Output from the particle handling code.
\item \newfeature{P3M\_DEBUG}
\item \newfeature{ESR\_DEBUG} debugging of P$^3$Ms real space part.
\item \newfeature{ESK\_DEBUG} debugging of P$^3$Ms $k$--space part.
\item \newfeature{EWALD\_DEBUG}
\item \newfeature{FFT\_DEBUG} Output from the unified FFT code.
\item \newfeature{MAGGS\_DEBUG}
\item \newfeature{RANDOM\_DEBUG}
\item \newfeature{FORCE\_DEBUG} Output from the force calculation loops.
\item \newfeature{PTENSOR\_DEBUG} Output from the pressure tensor calculation loops.
\item \newfeature{THERMO\_DEBUG} Output from the thermostats.
\item \newfeature{LJ\_DEBUG} Output from the Lennard--Jones code.
\item \newfeature{MORSE\_DEBUG} Output from the Morse code.
\item \newfeature{FENE\_DEBUG}
\item \newfeature{ONEPART\_DEBUG} Define to a number of a particle to
  obtain output on the forces calculated for this particle.
\item \newfeature{STAT\_DEBUG}
\item \newfeature{POLY\_DEBUG}
\item \newfeature{MOLFORCES\_DEBUG}
\item \newfeature{LB\_DEBUG} Output from the lattice--Boltzmann code.
\item \newfeature{VIRTUAL\_SITES\_DEBUG}
\item \newfeature{ASYNC\_BARRIER} Introduce a barrier after each
  asynchronous command completion. Helps in detection of mismatching
  communication.
\item \newfeature{FORCE\_CORE} Causes \es{} to try to provoke a core dump
  when exiting unexpectedly.
\item \newfeature{MPI\_CORE} Causes \es{} to try this even with MPI errors.
\end{itemize}

%%% Local Variables: 
%%% mode: latex
%%% TeX-master: "ug"
%%% End: 
