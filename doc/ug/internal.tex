% Copyright (C) 2010,2012,2013,2014 The ESPResSo project
% Copyright (C) 2002,2003,2004,2005,2006,2007,2008,2009,2010 
%   Max-Planck-Institute for Polymer Research, Theory Group
%  
% This file is part of ESPResSo.
%   
% ESPResSo is free software: you can redistribute it and/or modify it
% under the terms of the GNU General Public License as published by the
% Free Software Foundation, either version 3 of the License, or (at your
% option) any later version.
%  
% ESPResSo is distributed in the hope that it will be useful, but
% WITHOUT ANY WARRANTY; without even the implied warranty of
% MERCHANTABILITY or FITNESS FOR A PARTICULAR PURPOSE.  See the GNU
% General Public License for more details.
%  
% You should have received a copy of the GNU General Public License
% along with this program.  If not, see <http://www.gnu.org/licenses/>.
%
\chapter{Under the hood}
\label{chap:underhood}

\begin{itemize}
\item Implementation issues that are interesting for the user
\item Main loop in pseudo code (for comparison)
\end{itemize}

\section{Internal particle organization}
\label{sec:internal-particle-organization}

Since basically all major parts of the main MD integration have to
access the particle data, efficient access to the particle data is
crucial for a fast MD code. Therefore the particle data needs some
more elaborate organisation, which will be presented here. A particle
itself is represented by a structure (Particle) consisting of several
substructures (e. g. ParticlePosition, ParticleForce or
ParticleProperties), which in turn represent basic physical properties
such as position, force or charge. The particles are organised in one
or more particle lists on each node, called Cell cells. The cells are
arranged by several possible systems, the cellsystems as described
above. A cell system defines a way the particles are stored in \es{},
i. e. how they are distributed onto the processor nodes and how they
are organised on each of them. Moreover a cell system also defines
procedures to efficiently calculate the force, energy and pressure for
the short ranged interactions, since these can be heavily optimised
depending on the cell system. For example, the domain decomposition
cellsystem allows an order N interactions evaluation.

Technically, a cell is organised as a dynamically growing array, not
as a list. This ensures that the data of all particles in a cell is
stored contiguously in the memory. The particle data is accessed
transparently through a set of methods common to all cell systems,
which allocate the cells, add new particles, retrieve particle
information and are responsible for communicating the particle data
between the nodes. Therefore most portions of the code can access the
particle data safely without direct knowledge of the currently used
cell system. Only the force, energy and pressure loops are implemented
separately for each cell model as explained above.

The domain decomposition or link cell algorithm is implemented in
\es{} such that the cells equal the \es{} cells, i. e. each cell is a
separate particle list. For an example let us assume that the
simulation box has size $20\times 20\times 20$ and that we assign 2
processors to the simulation. Then each processor is responsible for
the particles inside a $10\times 20\times 20$ box. If the maximal
interaction range is 1.2, the minimal possible cell size is 1.25 for 8
cells along the first coordinate, allowing for a small skin of 0.05.
If one chooses only 6 boxes in the first coordinate, the skin depth
increases to 0.467. In this example we assume that the number of cells
in the first coordinate was chosen to be 6 and that the cells are
cubic. \es{} would then organise the cells on each node in a $6\times
12\times 12$ cell grid embedded at the centre of a $8\times 14 \times
14$ grid. The additional cells around the cells containing the
particles represent the ghost shell in which the information of the
ghost particles from the neighbouring nodes is stored. Therefore the
particle information stored on each node resides in 1568 particle
lists of which 864 cells contain particles assigned to the node, the
rest contain information of particles from other nodes.a

Classically, the link cell algorithm is implemented differently.
Instead of having separate particle lists for each cell, there is only
one particle list per node, and a the cells actually only contain
pointers into this particle list. This has the advantage that when
particles are moved from one cell to another on the same processor,
only the pointers have to be updated, which is much less data (4 rsp.
8 bytes) than the full particle structure (around 192 bytes, depending
on the features compiled in). The data storage scheme of \es{} however
requires to always move the full particle data. Nevertheless, from our
experience, the second approach is 2-3 times faster than the classical
one.

To understand this, one has to know a little bit about the
architecture of modern computers. Most modern processors have a clock
frequency above 1GHz and are able to execute nearly one instruction
per clock tick. In contrast to this, the memory runs at a clock speed
around 200MHz. Modern double data rate (DDR) RAM transfers up to
3.2GB/s at this clock speed (at each edge of the clock signal 8 bytes
are transferred). But in addition to the data transfer speed, DDR RAM
has some latency for fetching the data, which can be up to 50ns in the
worst case. Memory is organised internally in pages or rows of
typically 8KB size. The full $2\times 200$ MHz data rate can only be
achieved if the access is within the same memory page (page hit),
otherwise some latency has to be added (page miss). The actual latency
depends on some other aspects of the memory organisation which will
not be discussed here, but the penalty is at least 10ns, resulting in
an effective memory transfer rate of only 800MB/s. To remedy this,
modern processors have a small amount of low latency memory directly
attached to the processor, the cache.

The processor cache is organised in different levels. The level 1 (L1)
cache is built directly into the processor core, has no latency and
delivers the data immediately on demand, but has only a small size of
around 128KB. This is important since modern processors can issue
several simple operations such as additions simultaneously. The L2
cache is larger, typically around 1MB, but is located outside the
processor core and delivers data at the processor clock rate or some
fraction of it.

In a typical implementation of the link cell scheme the order of the
particles is fairly random, determined e. g. by the order in which the
particles are set up or have been communicated across the processor
boundaries. The force loop therefore accesses the particle array in
arbitrary order, resulting in a lot of unfavourable page misses. In
the memory organisation of \es{}, the particles are accessed in a
virtually linear order. Because the force calculation goes through the
cells in a linear fashion, all accesses to a single cell occur close
in time, for the force calculation of the cell itself as well as for
its neighbours. Using the domain decomposition cell scheme, two cell
layers have to be kept in the processor cache. For 10000 particles and
a typical cell grid size of 20, these two cell layers consume roughly
200 KBytes, which nearly fits into the L2 cache. Therefore every cell
has to be read from the main memory only once per force calculation.



%%% Local Variables: 
%%% mode: latex
%%% TeX-master: "ug"
%%% End: 
