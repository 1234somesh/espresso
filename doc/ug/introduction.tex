\chapter{Introduction}
\label{chap:intro}

(new)

\begin{itemize}
\item \es{} is a generic soft matter simulation packages
\item for molecular dynamics simulations in soft matter research
\item focussed on coarse-grained models
\item employs modern algorithms (Lattice-Boltzmann, DPD, P3M, \ldots)
\item written in C for maximal portability
\item Tcl-controlled
\item parallelized
\end{itemize}

\section{Guiding principles}
\label{sec:ideas}

(from paper: 2.1 Goals and principles)

\es
\begin{itemize}
\item does \emph{not} do the physics for you!
\item requires you to understand what you do (can not be used as a
  black box)
\item gives you maximal freedom (flexibility)
\item is extensible
\item integrates system setup, simulation and analysis, as this can't
  be strictly separated in soft matter simulations
\item has no predefined units
\item sets as few defaults as possible
\end{itemize}

\section{Algorithms contained in \es}

The following algorithms are implemented in \es{}:

\begin{itemize}
\item ensembles: NVE, NVT, NpT
\item charged systems:
  \begin{itemize}
  \item P3M for fully periodic systems
  \item ELC and MMM-family of algorithms for charged systems with
    non-periodic boundary conditions
  \item Maggs algorithm 
  \end{itemize}
\item Hydrodynamics:
  \begin{itemize}
  \item DPD (as a thermostat)
  \item Lattice-Boltzmann
  \end{itemize}
\end{itemize}

\section{Basic program structure}
\label{sec:structure}

(from paper: 2.2 Basic program structure)

\begin{itemize}
\item Control level: \texttt{Tcl}
\item ``Kernel'' written in \texttt{C}
\item This user's guide will focus on the control level
\end{itemize}

\section{On units}
\label{sec:units}
\index{units}
\index{physical units}
(new)

\begin{itemize}
\item Reduced units
\item comparison to ``real units''
\item three examples on different length scales
  \begin{itemize}
  \item some atomistic model?
  \item coarse-grained model (\eg lipid bilayer)
  \item billards?
  \end{itemize}
\end{itemize}

\section{Requirements}
\label{sec:requirements}
\index{requirements}

The following libraries and tools are required to be able to compile
and use \es:

\begin{description}
\item[Tcl/Tk] \index{Tcl/Tk} \es{} requires the Toolkit Command
  Language Tcl/Tk \footnote{\url{http://www.tcl.tk/}} in the version
  8.3 or later.  Some example scripts will only work with Tcl 8.4. You
  do not only need the interpreter, but also the header files and
  libraries.  Depending on the operating system, these may come in
  separate development packages. If you want to use a graphical user
  interface (GUI) for your simulation scripts, you will also need Tk.
  
\item[FFTW] \index{FFTW} In addition, \es{} needs the FFTW library
  \footnote{\url{http://www.fftw.org/}} for Fourier transforms.
  ESPResSo can work with both the 2.1.x and 3.0.x series. Again, the
  header files are required.
  
\item[MPI] \index{MPI} Finally, if you want to use \es{} in parallel,
  you need a working MPI environment (version 1.2). Currently, the
  following MPI implementations are supported:
  \begin{itemize}
  \item LAM/MPI is the preferred variant
  \item MPICH, which seems to be considerably slower than LAM/MPI in
    our benchmarks.
  \item On AIX systems, \es{} can also use the native POE parallel
    environment.
  \item On DEC/Compaq/HP OSF/Tru64, \es{} can also use the native
    dmpirun MPI environment.
  \end{itemize}
\end{description}


\section{Syntax description}
\label{sec:syntax}

Throughout the user's guide, formal definitions of the syntax of
several Tcl- and shell-commands can be found. The following
conventions are used in these decriptions:
\begin{itemize}
\item Keywords and literals of the command that have to be typed
  exactly as given are written in \lit{typewriter} font.
\item If the command has variable arguments, they are set in
  \var{italic font}. The description following the syntax definition
  should contain a detailed explanation of the argument and its
  type.
\item \texttt{[\var{argument}]} specifies, that \var{argument} is
  optional, \ie{} it can be omitted.
\item \texttt{<\var{alt1}|\var{alt2}>} specifies, that one of the
  alternatives \var{alt1} or \var{alt2} can be used.
\end{itemize}

\minisec{Example}
\begin{code}
writevsf \var{file} [<short|verbose>] [<\var{radii}|auto>] [typedesc \var{typedesc}]
\end{code}


%%% Local Variables: 
%%% mode: latex
%%% TeX-master: "ug"
%%% End: 
