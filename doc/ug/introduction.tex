% Copyright (C) 2010,2011,2012,2013,2014,2015,2016 The ESPResSo project
% Copyright (C) 2002,2003,2004,2005,2006,2007,2008,2009,2010 
%    Max-Planck-Institute for Polymer Research, Theory Group
%  
% This file is part of ESPResSo.
%   
% ESPResSo is free software: you can redistribute it and/or modify it
% under the terms of the GNU General Public License as published by the
% Free Software Foundation, either version 3 of the License, or (at your
% option) any later version.
%  
% ESPResSo is distributed in the hope that it will be useful, but
% WITHOUT ANY WARRANTY; without even the implied warranty of
% MERCHANTABILITY or FITNESS FOR A PARTICULAR PURPOSE.  See the GNU
% General Public License for more details.
%  
% You should have received a copy of the GNU General Public License
% along with this program.  If not, see <http://www.gnu.org/licenses/>.
%
\chapter{Introduction}
\label{chap:intro}

\es is a simulation package designed to perform Molecular Dynamics (MD)
and Monte Carlo (MC) simulations.  It is meant to be a universal tool for 
simulations of a variety of soft matter systems. 
\es features a broad range of interaction potentials which opens up
possibilities for performing simulations using models with 
different levels of coarse-graining. It also includes modern and
efficient algorithms for treatment of electrostatics (P3M, MMM-type 
algorithms, Maggs algorithm, \ldots), hydrodynamic interactions
(DPD, Lattice-Boltzmann), and magnetic interactions. It is 
designed to exploit the capabilities of parallel computational
environments. The program is being continuously extended to keep
the pace with current developments both in the algorithms and software.

The kernel of \es is written in C with computational efficiency in mind.
Interaction between the user and the simulation engine is provided
via a Tcl scripting interface. This enables setup of arbitrarily
complex systems which users might want to simulate in future, 
as well as modifying simulation parameters during runtime.

%\begin{itemize}
%\item \es{} is a generic soft matter simulation packages
%\item for molecular dynamics simulations in soft matter research
%\item focussed on coarse-grained models
%\item employs modern algorithms (Lattice-Boltzmann, DPD, P3M, \ldots)
%\item written in C for maximal portability
%\item Tcl-controlled
%\item parallelized
%\end{itemize}

\section{Guiding principles}
\label{sec:ideas}

\es is a tool for performing computer simulation and this user guide
describes how to use this tool. However, it should be borne in mind
that being able to operate a tool is not sufficient to obtain 
physically meaningful results. It is always the responsibility of the user
to understand the principles behind the model, simulation
and analysis methods he is using. \es will \emph{not} do that for you!

It is expected that the users of \es and readers of this user guide
have a thorough understanding of simulation methods and algorithms
they are planning to use. They should have passed a basic course on
molecular simulations or read one of the renown textbooks, \eg
\cite{frenkel02b}.  It is not necessary to understand everything that
is contained in \es, but it is inevitable to understand all methods
that you want to use.  Using the program as a black box without
proper understanding of the background will most probably result in
wasted user and computer time with no useful output.

To enable future extensions, the functionality of the program is kept
as general as possible.  It is modularized, so that extensions to some
parts of the program (\eg implementing a new potential) can be done by
modifying or adding only few files, leaving most of the code
untouched.

To facilitate the understanding and the extensibility, much emphasis
is put on readability of the code.  Hard-coded assembler loops are
generally avoided in hope that the overhead in computer time will be
more than compensated for by saving much of the user time while trying
to understand what the code is supposed to do.

Hand-in-hand with the extensibility and readability of the code comes
the flexibility of the whole program.  On the one hand, it is provided
by the generalized functionality of its parts, avoiding highly
specialized functions.  An example can be the implementation of the
Generic Lennard-Jones potential described in
section~\ref{sec:GenLennardJones} where the user can change all
available parameters. Where possible, default values are avoided,
providing the user with the possibility of choice.  \es cannot be
aware whether your particles are representing atoms or billiard balls, so
it cannot check if the chosen parameters make sense and it is the
user's responsibility to make sure they do.

On the other hand, flexibility of \es stems from the employment of a scripting language  at the steering level. Apart from the ability to modify the simulation
and system parameters at runtime, many simple tasks which are not
computationally critical can be implemented at this level, without
even touching the C-kernel.  For example, simple problem-specific
analysis routines can be implemented in this way and made interact
with the simulation core.  Another example of the program's
flexibility is the possibility to integrate system setup, simulation and
analysis in one single control script. \es provides commands to create
particles and set up interactions between them.  Capping of forces
helps prevent system blow-up when initially some particles are placed
on top of each other. Using the Tcl interface, one can simulate the
randomly set-up system with capped forces, interactively check whether
it is safe to remove the cap and switch on the full interactions and
then perform the actual productive simulation.

\section{Available simulation methods}

\es provides a number of useful methods. The following table shows the
various methods as well as their status. The table distinguishes
between the state of the development of a certain feature and the
state of its use. We distinguish between five levels:
\begin{description}
\item[Core] means that the method is part of the core of \es, and that
  it is extensively developed and used by many people.
\item[Good] means that the method is developed and used by independent
  people from different groups.
\item[Group] means that the method is developed and used in one group.
\item[Single] means that the method is developed and used by one
  person only.
\item[None] means that the method is developed and used by nobody.
\end{description}
If you believe that the status of a certain method is wrong, please
report so to the developers.

\newpage
\definecolor{supportgrey}{rgb}{0.9,0.9,0.9}
\rowcolors{1}{supportgrey}{}
\begin{longtable}{|l|l|l|}
  \hline
  \textbf{Feature} & \textbf{Development Status} & \textbf{Usage Status}\\\hline%
  \multicolumn{3}{|c|}{\textbf{Integrators, Thermostats, Barostats}} \\
  Velocity-Verlet Integrator          & Core   & Core   \\
  Langevin Thermostat                 & Core   & Core   \\
  GHMC Thermostat                     & Single & Single \\
  DPD Thermostat                      & None   & Good   \\
  Isotropic NPT                       & None   & Single \\
  NEMD                                & None   & Group  \\
  Quarternion Integrator              & None   & Good   \\
  \multicolumn{3}{|c|}{\textbf{Interactions}} \\
  Short-range Interactions            & Core   & Core   \\
  Directional Lennard-Jones           & Single & Single \\
  Gay-Berne Interaction               & None   & Single \\
  Constraints                         & Core   & Core   \\
  Relative Virtual Sites              & Good   & Good   \\
  Center-of-mass Virtual Sites        & None   & Good   \\
  RATTLE Rigid Bonds                  & None   & Group  \\
  \multicolumn{3}{|c|}{\textbf{Coulomb Interaction}} \\
  P3M                                 & Core   & Core   \\
  P3M on GPU                          & Single & Single \\
  Dipolar P3M                         & Group  & Good   \\
  Ewald on GPU                        & Single & Single \\
  MMM1D                               & Single & Good   \\
  MMM2D                               & Single & Good   \\
  MMM1D on GPU                        & Single & Single \\
  ELC                                 & Good   & Good   \\
  MEMD                                & Single & Group  \\
  ICC*                                & Group  & Group  \\
  \multicolumn{3}{|c|}{\textbf{Hydrodynamic Interaction}} \\
  Lattice-Boltzmann                   & Core   & Core   \\
  Lattice-Boltzmann on GPU            & Group  & Core   \\
  DPD                                 & None   & Good   \\
  Shan-Chen Multicomponent Fluid      & Group  & Group  \\
  Tunable Slip Boundary               & Single & Single \\
  Stokesian Dynamics                  & Single & Single \\
  \multicolumn{3}{|c|}{\textbf{Analysis}} \\
  uwerr                               & None   & Good   \\
  \multicolumn{3}{|c|}{\textbf{Input/Output}} \\
  Blockfiles                          & Core   & Core   \\
  VTF output                          & Core   & Core   \\
  VTK output                          & Group  & Group  \\
  PDB output                          & Good   & Good   \\
  Online visulation with VMD          & Good   & Good   \\
  \multicolumn{3}{|c|}{\textbf{Miscellaneous}} \\
  Grand canonical feature             & Single & Single \\
  Metadynamics                        & Single & Single \\
  Parallel Tempering                  & Single & Single \\
  Electrokinetics                     & Group  & Group  \\
  Object-in-fluid                     & Group  & Group  \\
  Collision Detection                 & Group  & Group  \\
  Catalytic Reactions                 & Single & Single \\
  mbtools package                     & Group  & Group  \\
  \hline
\end{longtable}

\section{Basic program structure}
\label{sec:structure}

As already mentioned, \es consists of two components.
The simulation engine is written in C and C++ for the sake
of computational efficiency. The steering or control
level is interfaced to the kernel via an interpreter 
of the Tcl and Python scripting languages.
Please be aware that the Tcl interface is going to be removed soon and new simulations should use Python as scripting language.

The kernel performs all computationally demanding tasks. Before all,
integration of Newton's equations of motion, including calculation of
energies and forces. It also takes care of internal organization of
data, storing the data about particles, communication between
different processors or cells of the cell-system. The kernel is
modularized so that basic functions are accessed via a set of
well-defined lean interfaces, hiding the details of the complex
numerical algorithms.

The scripting interface (Python or Tcl) are used to setup the system (particles, boundary conditions, interactions, ...), control the simulation, run analysis, and store and load results.
The user has at hand the full reliability and functionality of the scripting language. For instance, it is possible to use the SciPy package for analysis and PyPlot for plotting.
With a certain overhead in efficiency, it can also be
used to reject/accept new configurations in combined MD/MC schemes. In
principle, any parameter which is accessible from the scripting level can be
changed at any moment of runtime.  In this way methods like
thermodynamic integration become readily accessible.

The focus of the user guide is documenting the scripting interface, its
behavior and use in the simulation. It only describes certain
technical details of implementation which are necessary for
understanding how the script interface works.  Technical documentation of the
code and program structure is contained in the Developers' guide (see
section~\ref{sec:dg}).

\section{On units}
\label{sec:units}
\index{units}
\index{length unit}
\index{time unit}
\index{energy unit}
\index{physical units}

What is probably one of the most confusing subjects for beginners of
\es is, that \es does not predefine any units.  While most MD programs
specify a set of units, like, for example, that all lengths are
measured in \AA ngstr\"om or nanometers, times are measured in nano-
or picoseconds and energies are measured in $\mathrm{kJ/mol}$, \es
does not do so.

Instead, the length-, time- and energy scales can be freely chosen by
the user. Once these three scales are fixed, all remaining units are
derived from these three basic choices.

The probably most important choice is the length scale.
A length of $1.0$ can mean a nanometer, an \AA ngstr\"om,
or a kilometer - depending on the physical system, that the user has
in mind when he writes his \es-script. When creating particles that are
intended to represent a specific type
of atoms, one will probably use a length scale of \AA ngstr\"om.  This
would mean, that \eg the parameter $\sigma$ of the Lennard-Jones
interaction between two atoms would be set to twice the van-der-Waals
radius of the atom in \AA ngstr\"om.  Alternatively, one could set
$\sigma$ to $2.0$ and measure all lengths in multiples of the
van-der-Waals radius. When simulation colloidal particles,
which are usually of micrometer size, one will choose their diameter
(or radius) as basic length scale, which is much larger than the 
\AA ngstr\"om scale used in atomistic simulations.
 
The second choice to be made is the energy scale.  One can
for example choose to set the Lennard-Jones parameter $\epsilon$ to
the energy in $\mathrm{kJ/mol}$.  Then all energies will be
measured in that unit.  Alternatively, one can choose to set it to
$1.0$ and measure everything in multiples of the van-der-Waals binding
energy of the respective particles.

The final choice is the time (or mass) scale. By default, \es{} uses a
reduced mass of 1, so that the mass unit is simply the mass of all
particles. Combined with the energy and length scale, this is sufficient
to derive the resulting time scale:
$$
[\mathrm{time}] = [\mathrm{length}]\sqrt{\frac{[\mathrm{mass}]}{[\mathrm{energy}]}}.
$$
This means, that if you measure lengths in \AA ngstr\"om, energies
in $k_B T$ at 300\,K and masses in 39.95u, then your time scale is
$\AA \sqrt{39.95u / k_B T} = 0.40\,\mathrm{ps}$.

On the other hand, if you want a particular time scale, then the mass
scale can be derived from the time, energy and length scales as
$$
[\mathrm{mass}] = [\mathrm{energy}]\frac{[\mathrm{time}]^2}{[\mathrm{length}]^2}.
$$
By activating the feature MASSES, you can specify particle masses
in the chosen unit system.

A special note is due regarding the temperature, which is coupled to
the energy scale by Boltzmann's constant. However, since \es{} does not
enforce a particular unit system, we also don't know the numerical value
of the Boltzmann constant in the current unit system. Therefore, when
specifying the temperature of a thermostat, you actually do not define
the temperature, but the value of the thermal energy $k_B T$ in the
current unit system. For example, if you measure energy in units of
$\mathrm{kJ/mol}$ and your real temperature should be 300\,K, then you need to
set the thermostat's effective temperature to
$k_B 300\, K \mathrm{mol / kJ} = 2.494$.

As long as one remains within the same unit system throughout the
whole \es-script, there should be no problems.

\section{Requirements}
\label{sec:requirements}
\index{Installation requirements}

The following libraries and tools are required to be able to compile
and use \es:

\begin{description}
\item[Tcl/Tk] \index{Tcl/Tk} \es{} requires the Toolkit Command
  Language Tcl/Tk \footnote{\url{http://www.tcl.tk/}} in the version
  8.3 or later.  Some example scripts will only work with Tcl 8.4. You
  do not only need the interpreter, but also the header files and
  libraries.  Depending on the operating system, these may come in
  separate development packages. If you want to use a graphical user
  interface (GUI) for your simulation scripts, you will also need Tk.
  
\item[FFTW] \index{FFTW} For some algorithms (\eg P$^3$M), \es needs
  the FFTW library version 3 or later
  \footnote{\url{http://www.fftw.org/}} for Fourier transforms.
  Again, the header files are required.
  
\item[MPI] \index{MPI} Finally, if you want to use \es in parallel,
  you need a working MPI environment (that implements the MPI standard
  version 1.2).
\end{description}

\subsection{Installing Requirements on Ubuntu 16.04 LTS}
\label{sec:installing-requirements}
\index{Installation requirements!Ubuntu 16.04 LTS}

To make ESPResSo run on Ubuntu 16.04 LTS, its dependencies can be installed with:
\begin{code}
  sudo apt install build-essential cmake cython python-numpy tcl-dev tk-dev libboost-all-dev openmpi-common
\end{code}
Optionally the ccmake utility can be installed for easier configuration:
\begin{code}
  sudo apt install cmake-curses-gui
\end{code}

\subsection{Installing Requirements on Mac OS X}
\label{sec:installing-requirements-mac}
\index{Installation requirements!Mac OS X}

To make ESPResSo run on Mac OS X 10.9 or higher, its dependencies can be installed using MacPorts.
First, download the installer package appropriate for your Mac OS X version from \url{https://www.macports.org/install.php} and install it. Then, run the following commands:
\begin{code}
  sudo xcode-select --install
  sudo xcodebuild -license accept
  port selfupdate
  port install cmake python27 python27-cython python27-numpy tcl tk openmpi-default fftw-3 +openmpi boost +openmpi +python27
  port select --set cython cython27
  port select --set python python27
  port select --set mpi openmpi-mp-fortran
\end{code}

\section{Tcl: Syntax description}
\label{sec:syntax}


Throughout the user's guide, formal definitions of the syntax of
several Tcl-commands can be found. The following conventions are used
in these descriptions:
\begin{itemize}
\item Different \emph{variants} of a command are labeled \variant{1},
  \variant{2}, \ldots
\item Keywords and literals of the command that have to be typed
  exactly as given are written in \lit{typewriter} font.
\item If the command has variable arguments, they are set in
  \var{italic font}. The description following the syntax definition
  should contain a detailed explanation of the argument and its
  type.
\item \texttt{\alt{\var{alt1} \asep \var{alt2}}} specifies, that one
  of the alternatives \var{alt1} or \var{alt2} can be used.
\item \texttt{\opt{\var{argument}}} specifies, that the argument
  \var{argument} is optional, \ie it can be omitted.
\item When an optional argument or a whole command is marked by a
  superscript label (\fmark{1}), this denotes that the argument can
  only be used, when the corresponding feature (see appendix
  \vref{chap:features}) specified in ``Required features'' is
  activated.
\end{itemize}


\minisec{Example}

\renewcommand{\variant}[1]{\par\rawvariant{#1}}
\begin{essyntaxbox}
  \variant{1} 
  constraint wall normal \var{n_x} \var{n_y} \var{n_z} 
  dist \var{d} type \var{id}
  
  \variant{2}
  constraint sphere center \var{c_x} \var{c_y} \var{c_z} 
  radius \var{rad} direction \var{direction} type \var{id} 
  
  \require{1}{%
    \variant{3}
    constraint rod center \var{c_x} \var{c_y} 
    lambda \var{lambda}
  } 
  
  \require{2,3}{%
    \variant{4}
    constraint ext_magn_field \var{f_x} \var{f_y} \var{f_z} 
  }

  \begin{features}
    \required{CONSTRAINTS}
    \required[1]{ELECTROSTATICS}
    \required[2]{ROTATION}
    \required[3]{DIPOLES}
  \end{features}

\end{essyntaxbox}
\renewcommand{\variant}[1]{\rawvariant{#1}}

%%% Local Variables: 
%%% mode: latex
%%% TeX-master: "ug"
%%% End: 
