% Copyright (C) 2010 The ESPResSo project
% Copyright (C) 2002,2003,2004,2005,2006,2007,2008,2009,2010 Max-Planck-Institute for Polymer Research, Theory Group, PO Box 3148, 55021 Mainz, Germany
%  
% This file is part of ESPResSo.
%   
% ESPResSo is free software: you can redistribute it and/or modify it
% under the terms of the GNU General Public License as published by the
% Free Software Foundation, either version 3 of the License, or (at your
% option) any later version.
%  
% ESPResSo is distributed in the hope that it will be useful, but
% WITHOUT ANY WARRANTY; without even the implied warranty of
% MERCHANTABILITY or FITNESS FOR A PARTICULAR PURPOSE.  See the GNU
% General Public License for more details.
%  
% You should have received a copy of the GNU General Public License
% along with this program.  If not, see <http://www.gnu.org/licenses/>.
%
\chapter{Introduction}
\label{chap:intro}

\todo{Make the following lists full text.}

Just a testing change2.

\begin{itemize}
\item \es{} is a generic soft matter simulation packages
\item for molecular dynamics simulations in soft matter research
\item focussed on coarse-grained models
\item employs modern algorithms (Lattice-Boltzmann, DPD, P3M, \ldots)
\item written in C for maximal portability
\item Tcl-controlled
\item parallelized
\end{itemize}

\section{Guiding principles}
\label{sec:ideas}

(from paper: 2.1 Goals and principles)

\es
\begin{itemize}
\item does \emph{not} do the physics for you!
\item requires you to understand what you do (can not be used as a
  black box)
\item gives you maximal freedom (flexibility)
\item is extensible
\item integrates system setup, simulation and analysis, as this can't
  be strictly separated in soft matter simulations
\item has no predefined units
\item sets as few defaults as possible
\end{itemize}

\section{Algorithms contained in \es}

The following algorithms are implemented in \es{}:

\begin{itemize}
\item ensembles: NVE, NVT, NpT
\item charged systems:
  \begin{itemize}
  \item P3M for fully periodic systems
  \item ELC and MMM-family of algorithms for charged systems with
    non-periodic boundary conditions
  \item Maggs algorithm 
  \end{itemize}
\item Hydrodynamics:
  \begin{itemize}
  \item DPD (as a thermostat)
  \item Lattice-Boltzmann
  \end{itemize}
\end{itemize}

\section{Basic program structure}
\label{sec:structure}

(from paper: 2.2 Basic program structure)

\begin{itemize}
\item Control level: \texttt{Tcl}
\item ``Kernel'' written in \texttt{C}
\item This user's guide will focus on the control level
\end{itemize}

\section{On units}
\label{sec:units}
\index{units}
\index{length unit}
\index{time unit}
\index{energy unit}
\index{physical units}

What is probably one of the most confusing subjects for beginners of
\es is, that \es does not predefine any units.  While most MD programs
specify a set of units, like, for example, that all lengths are
measured in \AA ngstr\"om or nanometers, times are measured in nano- or
picoseconds and energies are measured in $\frac{kJ}{\mathrm{mol}}$,
\es does not do so.

Instead, the length-, time- and energy scales can be freely chosen by
the user.  A length of $1.0$ can mean a nanometer, an \AA ngstr\"om,
or a kilometer - depending on the physical system, that the user has
in mind when he writes his \es-script.  The user can choose the unit
system that suits the system best.

When creating particles that are intended to represent a specific type
of atoms, one will probably use a length scale of \AA ngstr\"om.  This
would mean, that \eg the parameter $\sigma$ of the Lennard-Jones
interaction between two atoms would be set to twice the van-der-Waals
radius of the atom in \AA ngstr\"om.  Alternatively, one could set
$\sigma$ to $2.0$ and measure all lengths in multiples of the
van-der-Waals radius.

The second choice to be made is the energy (and time-) scale.  One can
for example choose to set the Lennard-Jones parameter $\epsilon$ to
the energy in $\frac{kJ}{\mathrm{mol}}$.  Then all energies will be
measured in that unit.  Alternatively, one can choose to set it to
$1.0$ and measure everything in multiples of the van-der-Waals binding
energy.

As long as one remains within the same unit system throughout the
whole \es-script, there should be no problems.

\section{Requirements}
\label{sec:requirements}
\index{requirements}

The following libraries and tools are required to be able to compile
and use \es:

\begin{description}
\item[Tcl/Tk] \index{Tcl/Tk} \es{} requires the Toolkit Command
  Language Tcl/Tk \footnote{\url{http://www.tcl.tk/}} in the version
  8.3 or later.  Some example scripts will only work with Tcl 8.4. You
  do not only need the interpreter, but also the header files and
  libraries.  Depending on the operating system, these may come in
  separate development packages. If you want to use a graphical user
  interface (GUI) for your simulation scripts, you will also need Tk.
  
\item[FFTW] \index{FFTW} In addition, \es{} needs the FFTW library
  \footnote{\url{http://www.fftw.org/}} for Fourier transforms.
  ESPResSo can work with both the 2.1.x and 3.0.x series. Again, the
  header files are required.
  
\item[MPI] \index{MPI} Finally, if you want to use \es{} in parallel,
  you need a working MPI environment (version 1.2). Currently, the
  following MPI implementations are supported:
  \begin{itemize}
  \item LAM/MPI is the preferred variant
  \item MPICH, which seems to be considerably slower than LAM/MPI in
    our benchmarks.
  \item On AIX systems, \es{} can also use the native POE parallel
    environment.
  \item On DEC/Compaq/HP OSF/Tru64, \es{} can also use the native
    dmpirun MPI environment.
  \end{itemize}
\end{description}


\section{Syntax description}
\label{sec:syntax}


Throughout the user's guide, formal definitions of the syntax of
several Tcl-commands can be found. The following conventions are used
in these decriptions:
\begin{itemize}
\item Different \emph{variants} of a command are labelled \variant{1},
  \variant{2}, \ldots
\item Keywords and literals of the command that have to be typed
  exactly as given are written in \lit{typewriter} font.
\item If the command has variable arguments, they are set in
  \var{italic font}. The description following the syntax definition
  should contain a detailed explanation of the argument and its
  type.
\item \texttt{\alt{\var{alt1} \asep \var{alt2}}} specifies, that one
  of the alternatives \var{alt1} or \var{alt2} can be used.
\item \texttt{\opt{\var{argument}}} specifies, that the arugment
  \var{argument} is optional, \ie{} it can be omitted.
\item When an optional argument or a whole command is marked by a
  superscript label (\fmark{1}), this denotes that the argument can
  only be used, when the corresponding feature (see appendix
  \vref{chap:features}) specified in ``Required features'' is
  activated.
\end{itemize}


\minisec{Example}

\renewcommand{\variant}[1]{\par\rawvariant{#1}}
\begin{essyntaxbox}
  \variant{1} 
  constraint wall normal \var{n_x} \var{n_y} \var{n_z} 
  dist \var{d} type \var{id}
  
  \variant{2}
  constraint sphere center \var{c_x} \var{c_y} \var{c_z} 
  radius \var{rad} direction \var{direction} type \var{id} 
  
  \require{1}{%
    \variant{3}
    constraint rod center \var{c_x} \var{c_y} 
    lambda \var{lambda}
  } 
  
  \require{2,3}{%
    \variant{4}
    constraint ext_magn_field \var{f_x} \var{f_y} \var{f_z} 
  }

  \begin{features}
    \required{CONSTRAINTS}
    \required[1]{ELECTROSTATICS}
    \required[2]{ROTATION}
    \required[3]{DIPOLES}
  \end{features}

\end{essyntaxbox}
\renewcommand{\variant}[1]{\rawvariant{#1}}

%%% Local Variables: 
%%% mode: latex
%%% TeX-master: "ug"
%%% End: 
