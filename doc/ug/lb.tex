% Copyright (C) 2010,2011,2012,2013,2014 The ESPResSo project
%  
% This file is part of ESPResSo.
%   
% ESPResSo is free software: you can redistribute it and/or modify it
% under the terms of the GNU General Public License as published by the
% Free Software Foundation, either version 3 of the License, or (at your
% option) any later version.
%  
% ESPResSo is distributed in the hope that it will be useful, but
% WITHOUT ANY WARRANTY; without even the implied warranty of
% MERCHANTABILITY or FITNESS FOR A PARTICULAR PURPOSE.  See the GNU
% General Public License for more details.
%  
% You should have received a copy of the GNU General Public License
% along with this program.  If not, see <http://www.gnu.org/licenses/>.
%

\chapter{Lattice-Boltzmann}
\label{sec:lb}
\newescommand{lb}

For an implicit treatment of a solvent, \es allows to couple the
molecular dynamics simulation to a Lattice-Boltzmann fluid. The Lattice-
Boltzmann-Method (LBM) is a fast, lattice based method that, in its
``pure'' form, allows to calculate fluid flow in different boundary conditions
of arbitrarily complex geometries. Coupled to molecular dynamics, it allows for 
the computationally efficient inclusion of hydrodynamic interactions into the 
simulation. The implementation of boundary conditions for the LBM is a difficult
 task, a lot of research is still being conducted on this topic. The focus of 
the \es implementation of the LBM is, of course, the coupling to MD and 
therefore available geometries and boundary conditions are somewhat limited 
in comparison to ``pure'' codes. 

Here we restrict the documentation to the interface. For a more detailed
description of the method, please refer to the literature.

\section{Setting up a LB fluid}

\begin{citebox}
  Please cite~\citewbibkey{espresso2} if you use the LB fluid and
  \citewbibkey{lbgpu} if you use the GPU implementation.
\end{citebox}


\begin{essyntax}
  lbfluid
  \require{2}{\opt{gpu}}
  \require{1 or 2}{\opt{agrid  \var{agrid}}}
  \require{1 or 2 or 3}{\opt{dens  \var{density} }}
  \require{1 or 2 or 3}{\opt{visc  \var{viscosity}}}
  \require{1 or 2}{\opt{tau  \var{lb\_timestep}}}
  \require{1 or 2 or 3}{\opt{bulk_viscosity  \var{bulk\_viscosity}}}
  \require{1 or 2 or 3}{\opt{ext_force  \var{f_x} \var{f_y} \var{f_z}}}
  \require{1 or 2 or 3}{\opt{friction   \var{gamma} } }
  \require{2}{\opt{couple   \var{2pt/3pt} } }
  \require{1 or 2 or 3}{\opt{gamma_odd  \var{gamma\_odd}}}
  \require{1 or 2 or 3}{\opt{gamma_even  \var{gamma\_even}}}
  \require{3}{\opt{mobility} \var{mobilities}  }
  \require{3}{\opt{sc\_coupling} \var{coupling\_constants}  }
  \begin{features}
  \required[1]{LB}
  \required[2]{LB_GPU}
  \required[3]{SHANCHEN}
  \end{features}
\end{essyntax}
The \lit{lbfluid} command initializes the fluid with a given
set of parameters. It is also possible to change parameters on the
fly, but this will only rarely be done in practice. Before being able
to use the LBM, it is necessary to set up a box of a desired size. The
parameter \lit{agrid} is used to set the lattice constant of the
fluid, so the size of the box in every direction must be a multiple of
\var{agrid}.

In \es the LB scheme and the MD scheme are not synchronized: In one LB
time step typically several MD steps are performed. This allows to
speed up the simulations and is adjusted with the parameter \var{tau},
the LB timestep.
The parameters \var{dens} and \var{visc} set up the density and
(kinematic) viscosity of the LB fluid in (usual) MD units.  Internally the LB
implementation works with a different set of units: all lengths are
expressed in \var{agrid}, all times in \var{tau} and so on.  Therefore
changing \var{agrid} and \var{tau}, might change the behaviour of the
LB fluid, \eg at boundaries, due to characteristics of the LBM
itself. It should also be noted that the LB nodes are located at 0.5, 1.5, 2.5, etc (in terms of \var{agrid}).
This has important implications for the location of hydrodynamic boundaries which are generally considered
to be halfway between two nodes to first order.
Currently it is not possible to precisely give a parameter set
where reliable results are expected, but we are currently performing a
study on that. Therefore the LBM should \emph{not be used as a black
  box}, but only after a careful check of all parameters that were
applied. 

The parameter \lit{ext_force} allows to apply an external body force
density that is homogeneous over the fluid. It is again to be given in
MD units.  The parameter \lit{bulk_viscosity} allows to tune the bulk
viscosity of the fluid and is given in MD units. In the limit of low
Mach (often also low Reynolds) number the results should be
independent of the bulk viscosity up to a scaling factor. 
It is however known that the values of the viscosity does 
affect the quality of the implemented link-bounce-back method.
\lit{gamma_odd} and
\lit{gamma_even} are the relaxation parameters for the kinetic
modes. Due to their somewhat obscure nature they are to be given
directly in LB units.

Before running a simulation at least the following parameters must be
set up: \lit{agrid}, \lit {dens}, \lit{visc}, \lit{tau},
\lit{friction}. For the other parameters, the following are taken:
\var{bulk\_viscosity}=0, \var{gamma\_odd}=0, \var{gamma\_even}=0,
\var{ f_x} = \var{ f_y} = \var{ f_z} = 0.

If the feature \lit{SHANCHEN} is activated, the Lattice Boltzmann
code (so far GPU version only) is extended to a two-component
Shan-Chen (SC) method.  The \lit{lbfluid} command requires in this case
to supply two values, for the respective fluid components, to each
of the options \lit{dens}, \lit{visc}, \lit{bulk_viscosity}, \lit{friction},
\lit{gamma_odd} and \lit{gamma_even}, when they are used, otherwise
they are set to the default values. The three elements of the
coupling matrix can be supplied with the option \lit{sc_coupling},
and the mobility coefficient can be
specified with the option \lit{mobility}. By default no copuling is activated, and the relaxation parameter associated to the mobility is zero, corresponding to an infinite value for \lit{mobility}. Additional details are
given in \ref{sec:shanchen} and \ref{sec:scmd-coupling}.

\begin{essyntax}
  lbfluid print_interpolated_velocity \var{x} \var{y} \var{z}
\end{essyntax}
This variant returns the velocity at point in countinous space. 
This can make it easier to calculate flow profiles independent of
the lattice constant.

\begin{essyntax}
  lbfluid save_ascii_checkpoint \var{filename}\\
  lbfluid save_binary_checkpoint \var{filename}\\
  lbfluid load_ascii_checkpoint \var{filename}\\
  lbfluid load_binary_checkpoint \var{filename}
\end{essyntax}
The first two save commands save all of the LB fluid nodes' populations to \var{filename} in ascii or binary format respectively.
The two load commands load the populations from \var{filename}.  This is  useful for restarting a simulation either on the same
machine or a different machine.  Some care should be taken when using the binary format as the format of doubles can depend
on both the computer being used as well as the compiler. One thing that one needs to be aware of is that loading the checkpoint also requires the used to reuse the old forces. This is necessary
since the coupling force between the paricles and the fluid has already been applied to the fluid. Failing to reuse the old forces breaks momentum conservation, which is in general a problem. It is
particularly problematic for bulk simulations as the system as a whole acquires a drift of the center of mass, causing errors in the calculation of velocities and diffusion coefficients.
The correct way to restart
an LB simulation is to first load in the particles with the correct forces, and use ``integrate {\it steps} reuse_forces'' upon the first
call to integrate. This causes the old forces to be reused and thus conserves momentum.

\section{LB as a thermostat}
\begin{essyntax}
  thermostat \require{1 or 2 or 3}{lb} \var{T}
  \begin{features}
  \required[1]{LB}
  \required[2]{LB_GPU}
  \required[3]{SHANCHEN}
  \end{features}
\end{essyntax}
The LBM implementation in \es uses Ahlrichs and D\"unweg's point coupling method
to couple MD particles the LB fluid. This coupling consists
in a frictional force and a random force:
\begin{equation*}
  \vec{F} = -\gamma \left(\vec{v}-\vec{u}\right) + \vec{F}_R.
\end{equation*}
The momentum acquired by the particles is then transferred back to the fluid using a linear interpolation scheme, to preserve total momentum.
In the GPU implementation the force can alternatively be interpolated using a three point scheme which couples the particles to the nearest 27 LB nodes.
This can be called using ``lbfluid \lit{couple} 3pt'' and is described in D\"{u}nweg and Ladd by equation 301\cite{duenweg08a}. Note that the three point coupling scheme is incompatible with the Shan Chen Lattice Boltmann.
The frictional force tends to decrease the relative velocity
between the fluid and the particle whereas the random forces are chosen
so large that the average kinetic energy per particle corresponds to
the given temperature, according to a fluctuation dissipation theorem.
No other thermostatting mechanism is necessary then. Please switch off any other thermostat 
 before starting the LB thermostatting mechanism.

The LBM implementation provides a fully thermalized LB fluid, \ie all
nonconserved modes, including the pressure tensor, fluctuate correctly
according to the given temperature and the relaxation parameters. All
fluctuations can be switched off by setting the temperature to 0.

Regarind the unit of the temperature, please refer to
Section~\ref{sec:units}.

\section{The Shan Chen bicomponent fluid\label{sec:shanchen}}
\begin{citebox}
  Please cite~\citewbibkey{sega13c} if you use the Shan Chen implementation described below.
\end{citebox}


The Lattice Boltzmann variant of Shan and Chan\cite{shan93a} is widely
used as it is simple and yet very effective in reproducing the most
important traits of multicomponent or multiphase fluids. The version
of the Shan-Chen method implemented in \es is an extension to
bi-component fluids of the multi-relaxation-times Lattice Boltzmann
with fluctuations applied to all modes, that is already present in
\es. It features, in addition, coupling with particles\cite{sega13c}
and component-dependent particle interactions (see sections
\ref{sec:scmd-coupling} and\ref{sec:scmd-affinity}).

The Shan-Chen fluid is set up using the \lit{lbfluid} command,
supplying two values (one per component) to the \lit{dens} option. Optionally, two values can be set for each of the usual transport coefficients  (shear and bulk viscosity), and for the ghost modes. It
is possible to set a relaxation time also for the momentum modes,
since they are not conserved quantities in the Shan-Chen method,
by using the option \lit{mobility}. The mobility transport coefficient
expresses the propensity of the two components to mutually diffuse,
and, differently from other transport coefficients, only one value
is needed, as it carachterizes the mixture as a whole. When thermal
fluctuations are switched on, a random noise is added, in addition,
also to the momentum modes. Differently from the other modes, a
correlated noise is added to the momentum ones, in order to preserve
the \emph{total} momentum. 


The fluctuating hydrodynamic equations that are simulated using the
Shan-Chen approach are
\begin{equation}\label{eq:shanchen-NS}
\rho \left(\frac{\partial }{\partial  t} {\vec {u}} + ({\vec {u}}\cdot {\vec {\nabla}})  {\vec {u}} \right)=-{\vec {\nabla}} p+{\vec {\nabla}} \cdot ({\vec {\Pi}}+\hat{{\vec {\sigma}}})+\sum_{\zeta} {\vec {g}}_{\zeta},
\end{equation}
\begin{equation}\label{eq:shanchen-cont}
\frac{\partial }{\partial  t} \rho_{\zeta}+{\vec {\nabla}} \cdot (\rho_{\zeta} {\vec {u}}) = {\vec {\nabla}} \cdot  ({\vec {D}}_{\zeta}+\hat{{\vec {\xi}}}_{\zeta}),
\end{equation}
\begin{equation}\label{eq:shanchen-globalcont}
\partial_t \rho+{\vec {\nabla}} \cdot (\rho {\vec {u}}) = 0,
\end{equation}
where the index $\zeta=1,2$ specifies the component, $\vec{u}$ is
the fluid (baricentric) velocity, $\rho=\sum_\zeta\rho_\zeta$ is
the total density, and $p=\sum_{\zeta} p_{\zeta}=\sum_{\zeta} c_s^2
\rho_{\zeta}$ is the internal pressure of the mixture ($c_s$ being
the sound speed). Two fluctuating terms $\hat{{\vec{\sigma}}}$ and
$\hat{{\vec{\xi}}}_{\zeta}$ are associated, respectivelu, to the
diffusive current ${\vec{D}}_{\zeta}$ and to the viscous stress
tensor ${\vec{\Pi}}$.

The coupling between the fluid components
is realized by the force
\begin{equation}
 \vec{g}_{\zeta}(\vec{r}) =  - \rho_{\zeta}(\vec{r})
 \sum_{\vec{r}'}\sum_{\zeta'}  g_{\zeta \zeta'} \rho_{\zeta'}
 (\vec{r}') (\vec{r}'-\vec{r}),
\end{equation} that acts on the component $\zeta$ at node position
$\vec{r}$, and depends on the densities on the neighboring nodes
located at $\vec{r}'$.  The width of the interfacial regions between
two components, that can be obtained with the Shan-Chen method is
usually 5-10 lattice units.  The coupling matrix $g_{\zeta \zeta'}$
is in general symmetric, so in the present implementation only three
real values need to be specified with the option \lit{sc_coupling}.
The \lit{lbfluid} command sets the density of the two components
to the values specified by the option \lit{dens}, and these can be
modified with the \lit{lbnode} command. Note that the number of
active fluid components can be accessed through the global variable
\lit{lb_components}.


\section{SC as a thermostat\label{sec:scmd-coupling}}
The coupling of particle dynamics to the Shan-Chen fluid has been
conceived as an extension of the Ahlrichs and D\"unweg's point
coupling, with the force acting on a particle given by
\begin{equation}
  \vec{F} = -\frac{\sum_\zeta \gamma_\zeta \rho_\zeta(\vec{r})}{\sum_\zeta \rho_\zeta(\vec{r}_\zeta)} \left(\vec{v}-\vec{u}\right) + \vec{F}_R + \vec{F}^{ps},
\end{equation}
where $\zeta$ identifies the component, $\rho_\zeta(\vec{r})$ is a
linear interpolation of the component density on the nodes surrounding
the particle, $\gamma_\zeta$ is the component-dependent friction
coefficient, $\vec{F}_R$ is the usual random force, and
\begin{equation}
\vec{F}^{\mathrm{ps}}= -  \sum_{\zeta} \kappa_{\zeta} \nabla \rho_{\zeta}(\vec{r}).
\end{equation}
This is an effective solvation force, that can drive the particle
towards density maxima or minima of each component, depending on the
sign of the constant $\kappa_\zeta$. Note that by setting the coupling
constant to the same negative value for both components will, in
absence of other forces, push the particle to the interfacial region.

In addition to the solvation force acting on particles, another one
that acts on the fluid components is present, representing the
solvation force of particles on the fluid.
\begin{equation}
  \vec{F}_{\zeta}^{\mathrm{fs}}(\vec{r}) = -\lambda_{\zeta} \rho_{\zeta}(\vec{r}) \sum_i \sum_{\vec{r}'} \Theta \left[\frac{(\vec{r}_i-\vec{r})}{|\vec{r}_i-\vec{r}|} \cdot \frac{(\vec{r}'-\vec{r})}{|\vec{r}'-\vec{r}|} \right] \frac{\vec{r}'-\vec{r}}{|\vec{r}'-\vec{r}|^2},
\end{equation}
where $\Theta(x)=1$ if $0<x<1$, and 0 otherwise, the sum over lattice
nodes is performed on the neighboring sites of $\vec{r}$ and the index
$i$ runs over all particles. Note that a dependence on the particle
index $i$ is assumed for $\kappa_\zeta$ and $\lambda_\zeta$.  This
force has the effect of raising or lowering (depending on the sign of
the coupling constant $\lambda_\zeta$) the density in the eight nodes
around a particle.  The particle property \lit{solvation}
(Chap.~\ref{chap:part}) sets the coupling constants
$\lambda_A$,$\kappa_A$,$\lambda_B$ and $\kappa_B$, where $A$ and $B$
denote the first and second fluid component, respectively.  A complete
description of the copuling scheme can be found in \cite{sega13c}.

\section{SC component-dependent interactions between particles}
\label{sec:scmd-affinity}

Often particle properties depend on the type of solvent in which they
are. For example, a polymer chain swells in a good solvent, and
collapses in a bad one. One of the possible ways to model the good or
bad solvent condition in coarse-grained models is to employ a WCA or a
LJ (attractive) potential, respectively. If one wants to model the two
components of the SC fluid as good/bad solvent, it is possible to do
it using the \lit{affinity} argument of the \lit{inter} command. This
non-bonded interaction type acts as a modifier to other
interactions. So far only the Lennard-Jones interaction is changed by
the \lit{affinity}, so that it switches in a continuous way (after the
potential minimum) from the full interaction to the WCA one. For more
information see \ref{sec:LennardJones} and \ref{sec:affinity}.

\section{Reading and setting single lattice nodes}
\begin{essyntax}
  lbnode \var{x} \var{y} \var{z} \alt{print \asep set} \var{args}
  \begin{features}
  \required[1]{LB}
  \required[2]{LB_GPU}
  \required[3]{SHANCHEN}
  \end{features}
\end{essyntax}
The \lit{lbnode} command allows to inspect (\lit{print}) and modify
(\lit{set}) single LB nodes.  Note that the indexing in every
direction starts with 0.  For both commands you have to specify what
quantity should be printed
or modified. Print allows the following arguments: \\
\begin{tabular}{p{0.2\columnwidth}p{0.5\columnwidth}}
  \lit{rho}\ & the density (one scalar$^{1,2}$ or two scalars$^3$). \\
  \lit{u} & the fluid velocity (three floats: $u_x$, $u_y$, $u_z$) \\
  \lit{pi} & the fluid velocity (six floats: $\Pi_{xx}$, $\Pi_{xy}$, $\Pi_{yy}$, $\Pi_{xz}$,  $\Pi_{yz}$,  $\Pi_{zz}$) \\
  \lit{pi_neq} & the nonequilbrium part of the pressure tensor, components as above. \\
  \lit{pop} & the 19 population (check the order from the source code please). \\
  \lit{boundary} & the flag indicating whether the node is a fluid node ($\lit{boundary}=0$) or a boundary node ($\lit{boundary}\ne 0$). Does not support \lit{set}. Refer to the \lit{lbboundary} command for this functionality.\\
$^1$\lit{LB} or \lit{LB_GPU}; $^2$\lit{SHANCHEN}
\end{tabular}

\noindent{}Example:
The line
\begin{tclcode}
puts [ lbnode 0 0 0 print u ]
\end{tclcode}
prints the fluid velocity in node 0 0 0 to the screen.  The command
\lit{set} allows to change the density or fluid velocity in a single
node. Setting the other quantities can easily be implemented.
Example:
\begin{tclcode}
puts [ lbnode 0 0 0 set u 0.01 0. 0.]
\end{tclcode}

\section{Removing total fluid momentum}
\label{sec:remove-momentum}
\begin{essyntax}
  lbfluid remove_momentum
  \begin{features}
  \required[1]{LB}
  \required[2]{LB_GPU}
  \required[3]{SHANCHEN}
  \end{features}
\end{essyntax}

In some cases, such as free energy profile calculations, it might be useful to prevent interface motion. This can be achieved using the command \lit{lbfluid remove_momentum}, that removes the total momentum of the fluid.


\section{Visualization}
\label{ssec:LBvisualization}
\begin{essyntax}
  lbfluid print \opt{vtk} \var{property} \var{filename} [\var{filename}]
\end{essyntax}
The print parameter of the \lit{lbfluid} command is a feature to simplify visualization. It allows for the export of the whole fluid field data into a 
file with name \var{filename} at once. Currently supported values for the 
parameter \var{property} are boundary and velocity when using \lit{LB} or \lit{LB_GPU} and density and velocity when using \lit{SHANCHEN}. The additional option 
\lit{vtk} enables export in the vtk format which is readable by visualization software such as paraview\footnote{http://www.paraview.org/} or mayavi2\footnote{http://code.enthought.com/projects/mayavi/}. Otherwise gnuplot readable data will be 
exported. If you plan to use paraview for visualization, note that also the 
particle positions can be exportet in the VTK format \ref{sec:writevtk}.
If the \lit{SHANCHEN} bicomponent fluid is used, two filenames have to be supplied when exporting the density field, to save both components.

\section{Setting up boundary conditions}
\begin{essyntax}
  \variant{1} lbboundary \var{shape} \var{shape\_args} \opt{velocity
    \var{vx} \var{vy} \var{vz}}
  \variant{2} lbboundary force \opt{\var{n_\mathrm{boundary}}}
  \begin{features}
  \required{LB_BOUNDARIES}
  \end{features}
\end{essyntax}

If nothing else is specified, periodic boundary conditions are assumed
for the LB fluid. Variant \variant{1} allows to set up other (internal
or external) boundaries.

The \lit{lbboundary} command syntax is very close to the
\lit{constraint} syntax, as usually one wants the hydrodynamic
boundary conditions to be shaped similarily to the MD
boundaries. Currently the shapes mentioned above are available and
their syntax exactly follows the syntax of the constraint command. For
example
\begin{tclcode}
  lbboundary wall dist 1.5 normal 1. 0. 0. 
\end{tclcode}
creates a planar boundary condition at distance 1.5 from the origin of
the coordinate system where the half space $x>1.5$ is treated as
normal LB fluid, and the other half space is filled with boundary
nodes.

Intersecting boundaries are in principle possible but must be treated
with care. 
In the current, only partly satisfactory, all nodes that are within at least
one boundary are treated as boundary nodes. Improving this is nontrivial, 
and suggestions are very welcome.

Currently, only the so called ``link-bounce-back'' algorithm for wall
nodes is available. This creates a boundary that is located
approximately midway between the lattice nodes, so in the above
example this corresponds indeed to a boundary at $x=1.5$. Note that
the location of the boundary is unfortunately not entirely independent of the
viscosity. This can \eg be seen when using the sample script
\lit{poiseuille.tcl} with a high viscosity.

The bounce back boundary conditions allow to set velocity at a boundary to a nonzero
value. This allows to create shear flow and boundaries moving relative to 
each other. This could be a fixed sphere in a channel moving at a finite speed -- 
corresponding to the galilei-transform of a moving sphere in a fixed channel.
The velocity boundary conditions are implemented according to \cite{succi01a}
eq.~12.58. Using this implementation as a blueprint for the boundary treatment 
an implementation of the Ladd-Coupling should be relatively straightforward.

Variant \variant{2} prints out the force on boundary number
\lit{n_boundary}.

\section{Choosing between the GPU and CPU implementations}
\begin{essyntax}
  \variant{1} lbfluid cpu
  \variant{2} lbfluid gpu
  \begin{features}
    \required[1]{LB}
    \required[2]{LB_GPU}
  \end{features}
\end{essyntax}

A very recent development is an implementation of the LBM for NVIDIA
GPUs using the CUDA framework.  On CUDA-supporting machines this can
be activated by configuring with \lit{configure
  --with-cuda=/path/to/cuda} and activating the feature \lit{LB_GPU}.
Within the \es-Tcl-script, the \lit{lbfluid} command can be used to
choose between the CPU and GPU implementations of the
Lattice-Boltzmann algorithm, for further information on CUDA support
see section~\ref{sec:cuda}.

Variant \variant{1} is the default and turns on the standard CPU
implementation of the Lattice-Boltzmann fluid, while variant
\variant{2} turns on the GPU implementation, implying that all
following LB-related commands are executed on the GPU.

Currently only a subset of the CPU commands are available for the GPU
implementation.  For boundary conditions analogous to the CPU
implementation, the feature \lit{LB_BOUNDARIES_GPU} has to be
activated.


\section{Electrohydrodynamics}

\begin{essyntax}
  setmd mu_E \var{\mu E_x} \var{\mu E_y} \var{\mu E_z}
  \begin{features}
    \required{LB}
    \required{LB_ELECTROHYDRODYNAMICS}
  \end{features}
\end{essyntax}

If the feature \lit{LB_ELECTROHYDRODYNAMICS} is activated, the
(non-GPU) Lattice Boltzmann Code can be used to implicitely model
surrounding salt ions in an external electric field by having the
charged particles create flow.

For that to work, you need to set the electrophoretic mobility
(multiplied by the external $E$-field) $\mu E$ in all 3 dimensions for
your system. The three given parameters are float values and should,
for a meaningful system, be less than $1.0$.

For more information on this method and how it works, read the
publication~\cite{hickey10a}.

%%% Local Variables:
%%% mode: latex
%%% TeX-master: "ug"
%%% End:
