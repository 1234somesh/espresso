\chapter{Setting up particles}
\label{chap:part}

\section{\texttt{part}: Creating single particles}
\newescommand{part}

\subsection{Defining particle properties}

\begin{essyntax}
  part
  \var{pid}
  \opt{pos \var{x} \var{y} \var{z}}
  \opt{type \var{typeid}}
  \opt{v \var{vx} \var{vy} \var{vz}}
  \opt{f \var{fx} \var{fy} \var{fz}}
  \opt{bond \var{bondid} \var{pid2} \dots}
  \require{1}{\opt{q \var{charge}}}
  \require{2}{\opt{quat \var{q1} \var{q2} \var{q3} \var{q4}}}
  \require{2}{\opt{omega \var{x} \var{y} \var{z}}}
  \require{2}{\opt{torque \var{x} \var{y} \var{z}}}\\
  \require{3}{\opt{\opt{un}fix \var{x} \var{y} \var{z}}}
  \require{3}{\opt{ext_force \var{x} \var{y} \var{z}}}
  \require{4}{\opt{exclude \var{pid2}\dots}}
  \require{4}{\opt{exclude delete \var{pid2}\dots}}
  \require{5}{\opt{mass \var{mass}}}
  \require{6}{\opt{dipm \var{moment}}}
  \require{6}{\opt{dip \var{dx} \var{dy} \var{dz}}}
  \begin{features}
    \required[1]{ELECTROSTATICS} 
    \required[2]{ROTATION}
    \required[3]{EXTERNAL_FORCES}
    \required[4]{EXCLUSION}
    \required[5]{MASS}
    \required[6]{DIPOLES}
  \end{features}
\end{essyntax}

This command modifies particle data, namely position, type (monomer,
ion, \dots), charge, velocity, force and bonds. Multiple properties can
be changed at once. If you add a new particle the position has to be
set first because of the spatial decomposition.

\begin{arguments}
\item[\var{pid}]
\item[\opt{pos \var{x} \var{y} \var{z}}] Sets the position of this
  particle to $(x,y,z)$.
\item[\opt{type \var{typeid}}] Restrictions:
  $\var{typeid} \geq 0$.\\ The
  \var{typeid} is used in the \keyword{inter} command
  (see section \vref{tcl:inter}) to define the parameters of the non
  bonded interactions between different kinds of particles.
\item[\opt{v \var{vx} \var{vy} \var{vz}}] Sets the velocity of
this particle to $(vx,vy,vz)$. The velocity remains variable and will be changed
during integration.
\item[\opt{f \var{fx} \var{fy} \var{fz}}] Set the force acting on this particle
to $(fx,fy,fz)$. The force remains variable and will be changed during integration.
\item[\opt{bond \var{bondid} \var{pid2}\dots}]
  Restrictions: \var{bondid} $\geq 0$; \var{pid2} must
  be an existing particle.  The \var{bondid} is used for
  the inter command to define bonded interactions.
\item[bond delete] Will delete all bonds attached to this particle.
\item[\opt{q \var{charge}}] Sets the charge of this praticle to $q$.
\item[\opt{quat \var{q1} \var{q2} \var{q3} \var{q4}}] \todo{Docs required (JC?).} 
  \item[\opt{omega \var{x} \var{y} \var{z}}] \todo{Docs
  required (JC?).} 
  \item[\opt{torque \var{x} \var{y} \var{z}}] \todo{Docs
  required (JC?).}
\item[\opt{fix \var{x} \var{y} \var{z}}] Fixes the particle in space.
  By supplying a set of 3 integers as arguments it is possible to fix
  motion in \var{x}, \var{y}, or \var{z} coordinates independently. For
  example \var{fix 0 0 1} will fix motion only in z. Note that
  \var{fix} without arguments is equivalent to \var{fix 1 1 1}.
\item[\opt{ext_force \var{x} \var{y} \var{z}}]
  An additional external force is applied to the particle.
\item[\opt{unfix}] Release any external influence from the particle.
\item[\opt{exclude \var{pid2}\dots+}] Restrictions:
  \var{pid2} must be an existing particle.  Between the
  current particle an the exclusion partner(s), no nonbonded
  interactions are calculated. Note that unlike bonds, exclusions are
  stored with both partners.  Therefore this command adds the defined
  exclusions to both partners.
\item[\opt{exclude delete \var{pid2}\dots}] Searches for the
  given exclusion and deletes it. Again deletes the exclusion with
  both partners.
  \item[\opt{mass \var{mass}}] Sets the mass of this particle to $mass$. If not
  set, all particles have a mass of 1 in reduced units.
  \item[\opt{dipm \var{moment}}] Sets the dipol moment of this particle to $moment$.
  \item[\opt{dip \var{dx} \var{dy} \var{dz}}] Sets the orientation of the
  dipol axis to $(dx,dy,dz)$.
 
  \end{arguments}
\subsection{Getting particle properties}

\begin{essyntax}
  \variant{1}
  part \var{pid} print
  \optlong{\alt{id \asep pos \asep type \asep folded_position \asep type \asep
      q \asep v \asep f \asep fix \asep ext_force \asep bond \asep
      \mbox{connections \opt{\var{range}}}}}\dots
  \variant{2} part
\end{essyntax}

Variant \variant{1} will return a list of the specified properties of
particle \var{pid}, or all properties, if no keyword is
specified.  Variant \variant{2} will return a list of all properties
of all particles.

\minisec{Example}
\begin{code}
part 40 print id pos q bonds
\end{code}
will return a list like
\begin{tclcode}
40 8.849 1.8172 1.4677 1.0 {}
\end{tclcode}
This routine is primarily intended for effective use in Tcl scripts.

When the keyword \keyword{connection} is specified, it returns the
connectivity of the particle up to \var{range} (defaults to 1). For
particle 5 in a linear chain the result up to \var{range} = 3 would
look like:
\begin{tclcode}
{ { 4 } { 6 } } { { 4 3 } { 6 7 } } { {4 3 2 } { 6 7 8 } } 
\end{tclcode}
The function is useful when you want to create bonded interactions to
all other particles a certain particle is connected to. Note that this
output can not be used as input to the part command. Check results if
you use them in ring structures.

If none of the options is specified, it returns all properties of the
particle, if it exists, in the form
\begin{tclcode}
  0 pos 2.1 6.4 3.1 type 0 q -1.0 v 0.0 0.0 0.0 f 0.0 0.0 0.0
  bonds { {0 480} {0 368} ... } 
\end{tclcode}
which may be used as an input to this function later on. The first
integer is the particle number.

Variant \variant{2} returns the properties of all stored particles in
a tcl-list with the same format as specified above:
\begin{tclcode}
{0 pos 2.1 6.4 3.1 type 0 q -1.0 v 0.0 0.0 0.0 f 0.0 0.0 0.0
 bonds{{0 480}{0 368}...}} 
{1 pos 1.0 2.0 3.0 type 0 q 1.0 v 0.0 0.0 0.0 f 0.0 0.0 0.0
 bonds{{0 340}{0 83}...}} 
{2...{{...}...}}
{3...{{...}...}}
...
\end{tclcode}

\subsection{Deleting  particles}
\label{tcl:part:delete}

\begin{essyntax}
  \variant{1} part \var{pid} delete
  \variant{2} part deleteall
\end{essyntax}

In variant \variant{1}, the particle \var{pid} is deleted
and all bonds referencing it.  Variant \variant{2} will delete all
particles currently present in the simulation. Variant \variant{3}
will delete all currently defined exclusions.

\subsection{Exclusions}

\begin{essyntax}
  \variant{1} part auto_exclusions \opt{\var{range}}
  \variant{2} part delete_exclusions
\end{essyntax}

Variant \variant{1} will create exclusions for all particles pairs
connected by not more than \var{range} bonds (\var{range} defaults to
2). This is typically used in atomistic simulations, where nearest and
next nearest neighbour interactions along the chain have to be omitted
since they are included in the bonding potentials. For example, if the
system contains particles $0$ \dots $100$, where particle $n$ is
bonded to particle $n-1$ for $1 \leq n \leq 100$, then it will result
in the exclusions:
\begin{itemize}
  \item particle 1 does not interact with particles 2 and 3
  \item particle 2 does not interact with particles 1, 3 and 4
  \item particle 3 does not interact with particles 1, 2, 4 and 5
  \item ...
\end{itemize}

Variant \variant{2} deletes all exclusions currently present in the
system.

\section{Creating groups of particle}

\subsection{\texttt{polymer}: Setting up polymer chains}

\newescommand{polymer}
\begin{essyntax}
  polymer 
  \var{num\_polymers} \var{monomers\_per\_chain}
  \var{bond\_length}\\
  \opt{start \var{pid}} 
  \opt{pos \var{x} \var{y} \var{z}}
  \opt{mode \alt{RW \asep SAW \asep PSAW} 
    \opt{\var{shield} \opt{\var{try_\mathrm{max}}}}} 
  \require{1}{\opt{charge \var{q}}} 
  \require{1}{\opt{distance \var{d_\mathrm{charged}}}}
  \opt{types \var{typeid_\mathrm{neutral}}
    \opt{\var{typeid_\mathrm{charged}}}} 
  \opt{bond \var{bondid}} 
  \opt{angle \var{\phi} \opt{\var{\theta} \opt{\var{x} \var{y}
        \var{z}}}}
  \require{2}{\opt{constraints}}
  \begin{features}
    \required[1]{ELECTROSTATICS}
    \required[2]{CONSTRAINTS}
  \end{features}
\end{essyntax}

This command will create \var{num\_polymers} polymer or
polyelectrolyte chains with \var{monomers\_per\_chain} monomers per
chain. The length of the bond between two adjacent monomers will be
set up to be \var{bond\_length}.

\begin{arguments}
\item[\var{num\_polymers}] Sets the number of polymer chains.
\item[\var{monomers\_per\_chain}] Sets the number of monomers per
  chain.
\item[\var{bond\_length}] Sets the initial distance between two adjacent
  monomers. The distance during the course of the simulation depends on the
  applied potentials. For fixed bond length please refer to the SHAKE algorithm.
  \todo{Link to rattle/shake.}
\item[\opt{start \var{pid}}] Sets the particle number of the
  start monomer to be used with the \keyword{part} command. This
  defaults to 0.

\item[\opt{pos \var{x} \var{y} \var{z}}] Sets the position of the
  first monomer in the chain to \var{x}, \var{y}, \var{z} (defaults to
  a randomly chosen value)
  
\item[\opt{mode \alt{RW  \asep  PSAW  \asep  SAW} \opt{\var{shield}
      \opt{\var{try_\mathrm{max}}}}}]
  Selects the setup mode:
  \begin{description}
  \item[\keyword{RW} (Random walk)] The monomers are
    randomly placed by a random walk with a steps size of
    \var{bond_length}.
  \item[\keyword{PSAW} (Pruned self-avoiding walk)] The position of a
    monomer is randomly chosen in a distance of \var{bond_length} to
    the previous monomer. If the position is closer to another
    particle than \var{shield}, the attempt is repeated up to
    \var{try_\mathrm{max}} times. Note, that this is not a real self-avoiding
    random walk, as the particle distribution is not the same. If you
    want a real self-avoiding walk, use the \keyword{SAW} mode.
    However, \keyword{PSAW} is several orders of magnitude faster than
    \keyword{SAW}, especially for long chains.
  \item[\keyword{SAW} (Self-avoiding random walk)] The positions of
    the monomers are chosen as in the plain random walk. However, if
    this results in a chain that has a monomer that is closer to
    another particle than \var{shield}, a new attempt of setting up
    the whole chain is done, up to \var{max_try} times.
  \end{description}
  The default for the mode is \keyword{RW}, the default for the
  \var{shield} is $1.0$, and the default for \var{try_\mathrm{max}} is
  $30000$, which is usually enough for \keyword{PSAW}. Depending on
  the length of the chain, for the \keyword{SAW} mode,
  \var{try_\mathrm{max}} has to be increased by several orders of
  magnitude.
\item[\opt{charge \var{valency}}] Sets the valency of the charged
  monomers.  If the valency of the charged polymers \var{valency} is
  smaller than $10^{-10}$, the charge is assumed to be zero, and the
  types are set to $\var{typeid_\mathrm{charged}} =
  \var{typeid_\mathrm{neutral}}$. If charge is not set, it defaults to
  0.0.
\item[\opt{distance \var{d_\mathrm{charged}}}] Sets the stride
  between the indices of two charged monomers. This defaults defaults
  to 1, meaning that all monomers in the chain are charged.
\item[\opt{types \var{typeid_\mathrm{neutral}}
    \var{typeid_\mathrm{charged}}}] Sets the type ids of the neutral
    and charged monomer types to be used with the \keyword{part}
    command. If only \var{typeid_\mathrm{neutral}} is defined,
    \var{typeid_\mathrm{charged}} defaults to $1$. If the option is
    omitted, both monomer types default to $0$.
  \item[\opt{bond \var{bondid}}] Sets the type number of the bonded
    interaction to be set between the monomers. This defaults to $0$.
    Any bonded interaction, no matter how many bonding-partners
    needed, is stored with the second particle in this bond.
    \todo{Link to bonded interactions}
  \item[\opt{angle \var{\phi} [\var{\theta} [\var{x} \var{y}
      \var{z}]]}] Allows for setting up helices or planar polymers:
    \var{\phi} and \var{theta} are the angles between adjacent bonds.
    \var{x}, \var{y} and \var{z} set the position of the second
    monomer of the first chain.
  \item[\opt{constraints}] If this option is specified, the particle setup-up
  tries to obey previously defined constraints (see section \vref{sec:constraint}).
\end{arguments}

\subsection{\texttt{counterions}: Set up counterions}
\newescommand{counterions}
\begin{essyntax}
  counterions
  \var{N} 
  \opt{start \var{pid}} 
  \opt{mode \alt{SAW \asep RW} \opt{\var{shield} \opt{\var{try_\mathrm{max}} }}} 
  \require{1}{\opt{charge \var{val}}}
  \opt{type \var{typeid}}
  \begin{features}
    \required[1]{ELECTROSTATICS}
  \end{features}
\end{essyntax}
This command will create \var{N} counterions in the simulation box.
\begin{arguments}
\item[\opt{start \var{pid}}] Sets the particle id of the first
  counterion.  It defaults to the current number of particles, \ie
  counterions are placed after all previously defined particles.
\item[\opt{mode \alt{SAW \asep RW} \opt{\var{shield}
      \opt{\var{try_\mathrm{max}} }}}] Specifies the setup method to
  place the counterions. It defaults to \keyword{SAW}. See the
  \keyword{polymer} command for a detailed description.
\item[\opt{charge \var{val}}] Specifies the charge of the counterions.
  If not set, it defaults to $-1.0$.
\item[\opt{type \var{typeid}}] Specifies the particle type of the
  counterions. It defaults to $2$.
\end{arguments}

\smallskip
\subsection{\texttt{salt}: Set up salt ions}
\newescommand{salt}
\begin{essyntax}
  salt 
  \var{N_+} \var{N_-} 
  \opt{start \var{pid}} 
  \opt{mode \alt{SAW \asep RW} \opt{\var{shield} \opt{\var{try_\mathrm{max}}}}}
  \require{1}{\opt{charges \var{val_+} \opt{\var{val_-}}}} 
  \opt{types \var{typeid_+} \opt{\var{typeid_-}}}
  \opt{rad \var{r}}
  \begin{features}
    \required[1]{ELECTROSTATICS}
  \end{features}
\end{essyntax}

Create \var{N_+} positively and \var{N_-} negatively charged salt ions
of charge \var{val_+} and \var{val_-} within the simulation box.
\begin{arguments}
\item[\opt{start \var{pid}}] Sets the particle id of the first
  (positively charged) salt ion. It defaults to the current number of
  particles.
\item[\opt{mode \alt{SAW \asep RW} \opt{\var{shield}
      \opt{\var{try_\mathrm{max}} }}}] Specifies the setup method to
  place the counterions. It defaults to \keyword{SAW}. See the
  \keyword{polymer} command for a detailed description.
\item[\opt{charge \var{val_+} \opt{\var{val_-}}}] Sets the charge of
  the positive salt ions to \var{val_+} and the one of the negatively
  charged salt ions to \var{val_-}. If not set, the values default to
  $1.0$ and $-1.0$, respectively.
\item[\opt{type \var{typeid_+} \opt{\var{typeid_-}}}] Specifies the
  particle type of the salt ions. It defaults to $3$ respectively $4$.
\item[\opt{rad \var{r}}] The salt ions are only placed in a
  sphere with radius \var{r} around the origin.
\end{arguments}


\subsection{\texttt{diamond}: Setting up diamond polymer networks}
\newescommand{diamond}
\begin{essyntax}
  diamond 
  \var{a} \var{bond\_length} \var{monomers\_per\_chain} 
  \opt{counterions \var{N_\mathrm{CI}}}\\ 
  \require{1}{\opt{charges \var{val_\mathrm{node}}
      \var{val_\mathrm{monomer}} \var{val_\mathrm{CI}}}}
  \require{1}{\opt{distance \var{d_\mathrm{charged}}}}
  \opt{nonet}
  \begin{features}
    \required[1]{ELECTROSTATICS}
  \end{features}
\end{essyntax}

Creates a diamond-shaped polymer network with 8 tetra-functional nodes
connected by $2*8$ polymer chains of length \var{MPC} in a unit cell of length
\var{a}. For inter-particle bonds interaction $0$ is taken which must be a
two-particle bond. 
\todo{A picture would be helpful.}
\todo{Which typeids are used for the different particles?}

\begin{arguments}
\item[\var{a}] Determines the size of the of the unit cell.
\item[\var{bond\_length}] Specifies the bond length of the polymer
  chains connecting the 8 tetra-functional nodes.
\item[\var{monomers\_per\_chain}] Sets the number of chain monomers
  between the functional nodes.
\item[\opt{counterions \var{N_\mathrm{CI}}}] Adds \var{N_\mathrm{CI}}
  counterions to the system.
\item[\opt{charges \var{val_\mathrm{node}} \var{val_\mathrm{monomer}}
    \var{val_\mathrm{CI}}}] Sets the charge of the nodes to
  \var{val_\mathrm{node}}, the charge of the connecting monomers to
  \var{val_\mathrm{monomer}}, and the charge of the counterions to
  \var{val_\mathrm{CI}}.
\item[\opt{distance \var{d_\mathrm{charged}}}] Specifies the distance
  between charged monomers along the interconnecting chains. If
  $\var{d_\mathrm{charged}} > 1$ the remaining chain monomers are
  uncharged.
  \item[\opt{nonet}] \todo{Define what \var{nonet} does.}
\end{arguments}


\subsection{\texttt{icosaeder}: Setting up an icosaeder}
\newescommand{icosaeder}
\begin{essyntax}
  icosaeder 
  \var{a} \var{monomers\_per\_chain} 
  \opt{counterions \var{N_\mathrm{CI}}} 
  \require{1}{\opt{charges \var{val_\mathrm{monomers}} \var{val_\mathrm{CI}}}}
  \require{1}{\opt{distance \var{d_\mathrm{charged}}}}
  \begin{features}
    \required[1]{ELECTROSTATICS}
  \end{features}
\end{essyntax}

Creates a modified icosaeder to model a fullerene (or soccer ball).
The edges are modeled by polymer chains connected at the corners of
the icosaeder. For inter-particle bonds interaction $0$ is taken which
must be a two-particle bond.  \todo{A picture would be helpful}

\begin{arguments}
\item[\var{a}] Defines the size of the icosaeder.
\item[\var{monomers\_per\_chain}] Specifies the number of chain monomers along one edge.
\item[\opt{counterions \var{N_\mathrm{CI}}}] Specifies the number of
  counterions to be placed into the system.
\item[\opt{charges \var{val_\mathrm{monomers}} \var{val_\mathrm{CI}}}]
  Set the charges of the monomers to \var{val_\mathrm{monomers}} and
  the charges of the counterions to \var{val_\mathrm{CI}}.
\item[\opt{distance \var{d_\mathrm{charged}}}] Specifies the distance
  between two charged monomer along the edge. If
  $\var{d_\mathrm{charged}} > 1$ the remaining monomers are uncharged.
\end{arguments}

\subsection{\texttt{crosslink}: Cross-linking polymers}
\newescommand{crosslink}
\begin{essyntax}
  crosslink 
  \var{num\_polymer} \var{monomers\_per\_chain} 
  \opt{start \var{pid}} 
  \opt{catch \var{r_\mathrm{catch}}}
  \opt{distLink \var{link\_dist}} 
  \opt{distChain \var{chain\_dist}} 
  \opt{FENE \var{bondid}} 
  \opt{trials \var{try_\mathrm{max}}} 
\end{essyntax}

Attempts to end-crosslink the current configuration of
\var{num\_polymer} equally long polymers with
\var{monomers\_per\_chain} monomers each, returning how many ends are
successfully connected.

\begin{arguments}
\item[\opt{start \var{pid}}] \var{pid} specifies the first monomer of
  the chains to be linked. It has to be specified if the polymers do
  not start at id 0.
\item[\opt{catch \var{r_catch}}] Set the radius around each monomer
  which is searched for possible new monomers to connect to.
  \var{r_\mathrm{catch}} defaults to $1.9$.
\item[\opt{distLink \var{link\_dist}}] The minimal distance of two
  interconnecting links. It defaults to $2$.
\item[\opt{distChain \var{chain\_dist}}] The minimal distance for an
  interconnection along the same chain. It defaults to $0$. If set to
  \var{monomers\_per\_chain}, no interchain connections are created.
\item[\opt{FENE \var{bondid}}] Sets the bond type for the connections
  to \var{bondid}.
\item[\opt{trials \var{try_\mathrm{max}}}] If not specified,
  \var{try_\mathrm{max}} defaults to $30000$.
\end{arguments}

\section{\texttt{constraint}: Setting up constraints}\label{sec:constraint}
\newescommand{constraint}

\begin{essyntax}
  \variant{1} 
  constraint wall normal \var{n_x} \var{n_y} \var{n_z} 
  dist \var{d} type \var{id}
  
  \variant{2}
  constraint sphere center \var{c_x} \var{c_y} \var{c_z} 
  radius \var{rad} direction \var{direction} type \var{id} 
  
  \variant{3}
  constraint cylinder center \var{c_x} \var{c_y} \var{c_z} 
  axis \var{n_x} \var{n_y} \var{n_z} 
  radius \var{rad} length \var{length} 
  direction \var{direction} 
  type \var{id} 
  
  \variant{4}
  constraint maze nsphere \var{n} 
  dim \var{d} sphrad \var{r_s} cylrad \var{r_c}
  type \var{id}
  
  \variant{5}  
  constraint pore center \var{c_x} \var{c_y} \var{c_z} 
  axis \var{n_x} \var{n_y} \var{n_z} 
  radius \var{rad} length \var{length} 
  type \var{id} 
  
  \require{1}{%
    \variant{6}
    constraint rod center \var{c_x} \var{c_y} 
    lambda \var{lambda}
  } 
  
  \require{1}{%
    \variant{7}
    constraint plate height \var{h}
    sigma \var{sigma} 
  }
  
  \require{2,3}{%
    \variant{8}
    constraint ext_magn_field \var{f_x} \var{f_y} \var{f_z} 
  }

  \begin{features}
    \required{CONSTRAINTS}
    \required[1]{ELECTROSTATICS}
    \required[2]{ROTATION}
    \required[3]{DIPOLES}
  \end{features}
\end{essyntax}

\todo{Does this command really work only with the LJ potential, or
  with any short-ranged potential?}
The \codebox{constraint} command offers a variety of surfaces that can be
defined to interact with desired particles. Variants \variant{1} to \variant{5}
create interactions via a
Lennard-Jones potential. \[4 \epsilon \left(\left(\frac{\sigma}{r}\right)^{12} -
  \left(\frac{\sigma}{r}\right)^6 + shift\right)\] with r being the
distance of the center of the particle to the surface. The constraints are identified like a particle via its
type for the lennard-jones force calculation. 
After a type is defined for each constraint one has
to define
the interaction of all different particle types with the constraint using
the \codebox{inter} command.

Variants \variant{6} and \variant{7} create interactions based on electrostatic
interactions. The corresponding force acts in direction of the normal vector of the
surface and applies to all charged particles.

Variant \variant{8} does not define a surface but is based on magnetic
dipolar interaction with an external magnetic field. It applies to all particles
with a dipol moment.

\textbf{Note that constraints are not saved to checkpoints and that they have to
be reset upon restarting a simulation.}

The resulting surface in variant \variant{1} is a plane defined by the
normal vector \var{n_x} \var{n_y} \var{n_z} and the distance
\var{d} from the origin. The force acts in direction of the normal. 

The resulting surface in variant
\variant{2} is a sphere with center \var{c_x} \var{c_y} \var{c_z} and radius
\var{rad}. The \var{direction} determines the force direction, -1 or
\opt{inside} for inward and +1 or \opt{outside} for outward. 

The resulting surface
in variant \variant{3} is a cylinder with center \var{c_x} \var{c_y}
\var{c_z} and radius \var{rad}. The \var{length} parameter is \textbf{half} 
of the cylinder length. The \var{axis} is a
vector along the cylinder axis, which is normalized in the program.
The \var{direction} is defined the same way as for the spherical
constraint. 

The resulting surface in variant \variant{4} is \var{n}
spheres of radius \var{r_s} along each dimension, connected by
cylinders of radius \var{r_c}. The spheres have simple cubic
symmetry. The spheres are distributed evenly by dividing the
\var{box_l} by \var{n}.  Dimension of the maze can be controlled by
\var{d}: 0 for one dimensional, 1 for two dimensional and 2 for three
dimensional maze.

Variant \variant{5} sets up a cylindrical pore similar to variant \variant{3} 
with a center
\var{c_x}
\var{c_y}
\var{c_z} and radius \var{rad}. The \var{length} parameter is \textbf{half} 
of the cylinder length. The \var{axis} is a
vector along the cylinder axis, which is normalized in the program.
\todo{Is this command obsolete? Cylinder?}

Variant \variant{6} specifies an electrostatic interaction between the charged
particles in the system to an infinitely long rod with a
line charge of \var{lambda} which is alinge along the z-axis and centered at
\var{c_x} and \var{c_y}.

Variant \variant{7} specifies the electrostatic interactinos between the charged
particles in the system and an inifinitely large plate in the x-y-plane at
height \var{h}. The plate carries a charge density of \var{sigma}.
  
Variant \variant{8} specifies the dipolar coupling of particles with a dipolar
moment to an external field \var{f_x} \var{f_y} \var{f_z}. 

\subsection{Deleting a constraint}
\begin{essyntax}
  constraint delete \opt{\var{num}} 
\end{essyntax}

This command will delete constraints. If \var{num} is specified only this
constraint will deleted, otherwise all constraints will be removed from the
system. 

\subsection{Getting the force on a constraint}
\begin{essyntax}
constraint force \var{n} 
\end{essyntax}
Returns the force acting on the \var{n}th constraint.


\subsection{Getting the currently defined constraints}
\begin{essyntax}
constraint  \opt{\var{num}} 
\end{essyntax}
Prints out all constraint information. If \var{num} is specified only this
constraint is displayed, otherwise all constraints will be printed.

%%% Local Variables: 
%%% mode: latex
%%% TeX-master: "ug"
%%% End: 
