\chapter{Setting up particles}
\label{chap:part}

\section{\texttt{part}: Creating single particles}
\eslabel{part}

\subsection{Defining particle properties}

\begin{essyntax}[PART]
  part
  \var{particle_number}
  \opt{pos \var{x} \var{y} \var{z}}
  \opt{type \var{particle_type_number}}
  \requiresfeature{\opt{q \var{charge}}}{ELECTROSTATICS}
  \opt{v \var{x_value} \var{y_value} \var{z_value}}
  \opt{f \var{x_value} \var{y_value} \var{z_value}}
  \opt{quat \var{q1} \var{q2} \var{q3} \var{q4}}
  \opt{omega \var{x_value} \var{y_value} \var{z_value}}
  \opt{torque \var{x_value} \var{y_value} \var{z_value}}
  \opt{\opt{un}fix \var{x} \var{y} \var{z}}
  \opt{ext_force \var{x_value} \var{y_value} \var{z_value}}
  \opt{bond \var{bond_type_number}}
  \opt{exclude \var{exclusion_partner}\ldots}
\end{essyntax}

This command modifies particle data, namely position, type (monomer,
ion, \dots), charge, velocity, force and bonds. Multiple properties can
be changed at once. If you add a new particle the position has to be
set first because of the spatial decomposition.

\begin{arguments}
\item[\var{particle_number}]
\item[\opt{pos \var{x} \var{y} \var{z}}] Sets the position of this
  particle to $(x,y,z)$.
\item[\opt{type \var{particle_type_number}}] Restrictions:
  \var{particle_type_type_number} $\geq 0$.\\  The
  \var{particle_type_number} is used in the inter command to define
  the parameters of the non bonded interactions between different
  kinds of particles.
\item[\opt{q \var{charge}}]
\item[\opt{v \var{x_value} \var{y_value} \var{z_value}}]
\item[\opt{f \var{x_value} \var{y_value} \var{z_value}}] 
\item[\opt{quat \var{q1} \var{q2} \var{q3} \var{q4}} 
  \opt{omega \var{x_value} \var{y_value} \var{z_value}} 
  \opt{torque \var{x_value} \var{y_value} \var{z_value}}]
  Require the feature ROTATION.
\item[\opt{fix \var{x} \var{y} \var{z}}] Fixes the particle in space.
  By supplying a set of 3 integers as arguments it is possible to fix
  motion in \var{x}, \var{y}, or \var{z} coordinates independetly. For
  example \var{fix 0 0 1} will fix motion only in z. Note that
  \var{fix} without arguments is equivalent to \var{fix 1 1 1} (Needs
  compiled flag EXTERNAL_FORCES in config.h).
\item[\opt{ext_force \var{x_value} \var{y_value} \var{z_value}}]
  An additional external force is applied to the particle (Needs
  compiled flag EXTERNAL_FORCES in config.h).
\item[\opt{unfix}] Release any external influence from the particle
  (Needs compiled flag EXTERNAL_FORCES in config.h).
\item[\opt{bond \var{bond_type_number} \var{partner}+}]
  Restrictions: \var{bond_type_number} $\geq 0$; \var{partner} must
  be an existing particle.  The \var{bond_type_number} is used for
  the inter command to define bonded interactions.
\item[bond delete] Will delete all bonds attached to this particle.
\item[\opt{exclude \var{exclusion_partner}+}] Restrictions:
  \var{exclusion_partner} must be an existing particle.  Between the
  current particle an the exclusion partner(s), no nonbonded
  interactions are calculated (Needs compiled flag EXTERNAL_FORCES in
  config.h). Note that unlike bonds, exclusions are stored with both
  partners.  Therefore this command adds the defined exclusions to
  both partners.
\item[\opt{exclude delete \var{exclusion_partner}+}] Searches for the
  given exclusion and deletes it. Again deletes the exclusion with
  both partners.
\end{arguments}

\subsection{Getting particle properties}

\begin{essyntax}
  \variant{1}part \var{particle_number} print\\
  \alt{id \asep pos \asep type \asep folded_position \asep type \asep
    q \asep v \asep f \asep fix \asep ext_force \asep bond \asep
    \mbox{connections \opt{\var{range}}}}+
  \variant{2}part
\end{essyntax}

Variant \variant{1} will return a list of the specified properties of
particle \var{particle_number}, or all properties, if no keyword is
specified.  Variant \variant{2} will return a list of all properties
of all particles.

\minisec{Example}
\begin{code}
part 40 print id pos q bonds
\end{code}
will return a list like
\begin{tclcode}
40 8.849 1.8172 1.4677 1.0 {}
\end{tclcode}
This routine is primarily intended for effective use in Tcl scripts.

When the keyword \keyword{connection} is specified, it returns the
connectivity of the particle up to \var{range} (defaults to 1). For
particle 5 in a linear chain the result up to \var{range} = 3 would
look like:
\begin{tclcode}
{ { 4 } { 6 } } { { 4 3 } { 6 7 } } { {4 3 2 } { 6 7 8 } } 
\end{tclcode}
The function is useful when you want to create bonded interactions to
all other particles a certain particle is connected to. Note that this
output can not be used as input to the part command. Check results if
you use them in ring structures.

If none of the options is specified, it returns all properties of the
particle, if it exists, in the form
\begin{tclcode}
  0 pos 2.1 6.4 3.1 type 0 q -1.0 v 0.0 0.0 0.0 f 0.0 0.0 0.0
  bonds { {0 480} {0 368} ... } 
\end{tclcode}
which may be used as an input to this function later on. The first
integer is the particle number.

Variant \variant{2} returns the properties of all stored particles in
a tcl-list with the same format as specified above:
\begin{tclcode}
{0 pos 2.1 6.4 3.1 type 0 q -1.0 v 0.0 0.0 0.0 f 0.0 0.0 0.0
 bonds{{0 480}{0 368}...}} 
{1 pos 1.0 2.0 3.0 type 0 q 1.0 v 0.0 0.0 0.0 f 0.0 0.0 0.0
 bonds{{0 340}{0 83}...}} 
{2...{{...}...}}
{3...{{...}...}}
...
\end{tclcode}

\subsection{Deleting  particles}
%\label{tcl:part:delete}

\begin{essyntax}
  \variant{1} part \var{particle_number} delete
  \variant{2} part deleteall
\end{essyntax}

In variant \variant{1}, the particle \var{particle_number} is deleted
and all bonds referencing it.  Variant \variant{2} will delete all
particles currently present in the simulation. Variant \variant{3}
will delete all currently defined exclusions.

\subsection{Exclusions}

\begin{essyntax}
  \variant{1} part auto_exclusions \opt{\var{range}}
  \variant{2} part delete_exclusions
\end{essyntax}

Variant \variant{1} will create exclusions for all particles pairs
connected by not more than \var{range} bonds (\var{range} defaults to
2). This is typically used in atomistic simulations, where nearest and
next nearest neigbor interactions along the chain have to be omitted
since they are included in the bonding potentials. For example, if the
system contains particles $0$ \dots $100$, where particle $n$ is
bonded to particle $n-1$ for $1 \leq n \leq 100$, then it will result
in the exclusions:
\begin{itemize}
  \item particle 1 does not interact with particles 2 and 3
  \item particle 2 does not interact with particles 1, 3 and 4
  \item particle 3 does not interact with particles 1, 2, 4 and 5
  \item ...
\end{itemize}

Variant \variant{2} deletes all exclusions currently present in the
system.

\section{Creating groups of particle}

\subsection{\texttt{polymer}: Setting up polymer chains}

\eslabel{polymer}
\begin{essyntax}
  \variant{1}
  polymer 
  \var{num_polymers} \var{monomers_per_chain}
  \var{bond_length}\\
  \opt{start \var{part_id}} 
  \opt{pos \var{x} \var{y} \var{z}}
  \opt{mode \alt{RW \asep SAW \asep PSAW} 
    \opt{\var{shield} \opt{\var{max_try}}}} 
  \opt{charge \var{val_charged_monomer}} 
  \opt{distance \var{dist_charged_monomer}}\\ 
  \opt{types \var{type_neutral_monomer}
    \opt{\var{type_charged_monomer}}} 
  \opt{bond \var{type_bond}} 
  \opt{angle \var{phi} \opt{\var{theta} \opt{\var{x} \var{y}
        \var{z}}}}

  \variant{2}
  polymer
\end{essyntax}

This command will create \var{num_polymers} polymer or
polyelectrolyte chains with \var{monomers_per_chain} monomers per
chain. The length of the bond between two adjacent monomers will be
set up to be \var{bond_length}.

\begin{arguments}
\item[\var{num_polymers}] Sets the number of polymer chains.
\item[\var{monomers_per_chain}] Sets the number of monomers per
  chain.
\item[\var{bond_length}] Sets the distance between two adjacent
  monomers.
\item[\opt{start \var{part_id}}] Sets the particle number of the
  start monomer to be used with the \keyword{part} command. This
  defaults to 0.

\item[\opt{pos \var{x} \var{y} \var{z}}] Sets the position of the
  first monomer in the chain to \var{x}, \var{y}, \var{z} (defaults to
  a randomly chosen value)
  
\item[\opt{mode \alt{RW  \asep  PSAW  \asep  SAW} \opt{\var{shield}
      \opt{\var{max_try}}}}]
  Selects the setup mode:
  \begin{description}
  \item[\keyword{RW} (Random walk)] The monomers are
    randomly placed by a random walk with a steps size of
    \var{bond_length}.
  \item[\keyword{PSAW} (Pruned self-avoiding walk)] The position of a
    monomer is randomly chosen in a distance of \var{bond_length} to
    the previous monomer. If the position is closer to another
    particle than \var{shield}, the attempt is repeated up to
    \var{try_max} times. Note, that this is not a real self-avoiding
    random walk, as the particle distribution is not the same. If you
    want a real self-avoiding walk, use the \keyword{SAW} mode.
    However, \keyword{PSAW} is several orders of magnitude faster than
    \keyword{SAW}, especially for long chains.
  \item[\keyword{SAW} (Self-avoiding random walk)] The positions of
    the monomers are chosen as in the plain random walk. However, if
    this results in a chain that has a monomer that is closer to
    another particle than \var{shield}, a new attempt of setting up
    the whole chain is done, up to \var{max_try} times.
  \end{description}
  The default for the mode is \keyword{RW}, the default for the
  \var{shield} is $1.0$, and the default for \var{max_try} is
  $30000$, which is usually enough for \keyword{PSAW}. Depending on
  the length of the chain, for the \keyword{SAW} mode, \var{max_try}
  has to be increased by several orders of magnitude.
\item[\opt{charge \var{val_charged_monomer}}] Sets the valency of
  the charged monomers.  If the valency of the charged polymers
  \var{val_charged_monomer} is smaller than $10^{-10}$, the charge
  is assumed to be zero, and the types are set to
  \var{type_charged_monomer} = \var{type_neutral_monomer}. This
  defaults to 0.0.

\item[\opt{distance \var{dist_charged_monomer}}] Sets the stride
  between the indices of two charged monomers. This defaults defaults
  to 1, meaning that all monomers in the chain are charged.
  
\item[\opt{types \var{type_neutral_monomer}
    \var{type_charged_monomer}}] Sets the type numbers of the
  neutral and charged monomer types to be used with the \keyword{part}
  command. If \var{type_neutral_monomer} is defined,
  \var{type_charged_monomer} defaults to 1. If the option is
  omitted, both monomer types default to 0.
  
\item[\opt{bond \var{type_bond}}] Sets the type number of the bonded
  interaction to be set between the monomers. This defaults to 0.
  
\item[\opt{angle \var{phi} [\var{theta} [\var{x} \var{y} \var{z}]]}]
  Allows for setting up helices or planar polymers: \var{phi} sets
  the angle $\phi$ and \var{theta} sets the angle $\theta$ between
  adjacent bonds. \var{x}, \var{y} and \var{z} set the position of the
  second monomer of the first chain.
\end{arguments}

\subsection{\texttt{counterions}: Set up counterions}
\eslabel{counterions}
\begin{essyntax}
  counterions
  \var{N_CI} 
  \opt{\var{shield} \opt{\var{max_try} }} 
  \opt{start \var{part_id}} 
  \opt{mode \alt{SAW \asep RW}} 
  \opt{charge \var{val_CI}} 
  \opt{type \var{type_CI}}
\end{essyntax}
Create \var{N_CI} counterions in the simulation box.
\todo{Docs required.}

\smallskip
\subsection{\texttt{salt}: Set up salt ions}
\eslabel{salt}
\begin{essyntax}
  salt 
  \var{N_pS} \var{N_nS} 
  \opt{start \var{part_id}} 
  \opt{mode \alt{SAW \asep RW} \opt{\var{shield} \opt{\var{max_try}}}}
  \opt{charges \var{val_pS} \opt{\var{val_nS}}} 
  \opt{types \var{type_pS} \opt{\var{type_nS}}}
\end{essyntax}
Create \var{N_pS} positively and \var{N_nS} negatively charged salt
ions of charge \var{val_pS} and \var{val_nS} within the simulation
box.

\subsection{\texttt{diamond}: Setting up diamond polymer networks}
\eslabel{diamond}
\begin{essyntax}
  diamond 
  \var{a} \var{bond_length} \var{MPC} 
  \opt{counterions \var{N_CI}} 
  \opt{charges \var{val_nodes} \var{val_cM} \var{val_CI}} 
  \opt{distance \var{cM_dist}} 
  \opt{nonet}
\end{essyntax}
\todo{Docs missing.}
Creates a diamond-shaped polymer network with 8 tetra-functional nodes
connected by $2*8$ polymer chains of length \var{MPC}.

\subsection{\texttt{icosahedron}: Setting up an icosahedron}
\eslabel{icosahedron}
\begin{essyntax}
  icosahedron 
  \var{a} \var{MPC} 
  \opt{counterions \var{N_CI}} 
  \opt{charges \var{val_cM} \var{val_CI}} 
  \opt{distance \var{cM_dist}}
\end{essyntax}
\todo{Docs missing.}
Creates a modified icosahedron to model a fulleren (or soccer ball).

\subsection{\texttt{crosslink}: Cross-linking polymers}
\eslabel{crosslink}
\begin{essyntax}
  crosslink 
  \var{N_P} \var{MPC} 
  \opt{start \var{part_id}} 
  \opt{catch \var{r_catch}}
  \opt{distLink \var{link_dist}} 
  \opt{distChain \var{chain_dist}} 
  \opt{FENE \var{type_FENE}} 
  \opt{trials \var{max_try}} 
\end{essyntax}
\todo{Docs missing.}
Attempts to end-crosslink the current configuration of \var{N_P}
equally long polymers with \var{MPC} monomers each, returning how many
ends are successfully connected. 

\section{\texttt{constraint}: Setting up constraints}
\eslabel{constraint}

\begin{essyntax}
  \variant{1} 
  constraint wall normal \var{n_x} \var{n_y} \var{n_z} 
  dist \var{d} type \var{id}

  \variant{2}
  constraint sphere center \var{c_x} \var{c_y} \var{c_z} 
  radius \var{rad} direction \var{direction} type \var{id} 
  
  \variant{3}
  constraint cylinder center \var{c_x} \var{c_y} \var{c_z} 
  axis \var{n_x} \var{n_y} \var{n_z} 
  radius \var{rad} length \var{length} 
  direction \var{direction} 
  type \var{id} 
  
  \variant{4}
  constraint maze nsphere \var{n} 
  dim \var{d} sphrad \var{r_s} cylrad \var{r_c}
  type 10 \var{id} 
\end{essyntax}

\todo{Does it really only work for LJ-potentials?}
A constraint is a surface which interacts with the desired particles
via a Lennard-Jones potential.
\[4 \epsilon \left(\left(\frac{\sigma}{r}\right)^{12} -
  \left(\frac{\sigma}{r}\right)^6 + shift\right)\] with r being the
distance of the center of the particle to the surface. The
corresponding force acts in direction of the normal vector of the
surface. The constraints are identified like a particle for the
lennard-jones force calculation. 

After a type is defined for each constraint one has to define the
interaction of all different particle types with the constraint using
the \codebox{inter} command.

The resulting surface in variant \variant{1} is a plane defined by the
normal vector \var{n_x} \var{n_y} \var{n_z} and the distance
\var{d} from the origin. The resulting surface in variant \variant{2}
is a sphere with center \var{c_x} \var{c_y} \var{c_z} and radius
\var{rad}. The \var{direction} determines the force direction, -1 or
inside for inward and +1 or outside for outward. The resulting surface
in variant \variant{3} is a cylinder with center \var{c_x} \var{c_y}
\var{c_z} and radius \var{rad}. The \var{length} parameter is not the
whole length but a half of the cylinder length. The \var{axis} is a
vector along the cylinder axis, which is normalized in the program.
The \var{direction} is defined the same way as for the spherical
constraint. The resulting surface in variant \variant{4} is \var{n}
spheres of radius \var{r_s} along each dimension, connected by
cylinders of radius \var{r_c}. The spheres have simple cubic
symmetry. The spheres are distributed evenly by dividing the
\var{box_l} by \var{n}.  Dimension of the maze can be controlled by
\var{d}: 0 for one dimensional, 1 for two dimensional and 2 for three
dimensional maze.


\subsection{Deleting a constraint}
\begin{essyntax}
  constraint delete \var{num} 
\end{essyntax}

This will delete the constraint with the number \var{num}.

\subsection{Getting the force on a constraint}
\begin{essyntax}
constraint force \var{n} 
\end{essyntax}
Returns the force acting on the \var{n}th constraint.


\subsection{Getting the currently defined constraints}
\begin{essyntax}
constraint 
\end{essyntax}
Prints out all constraint information.


%%% Local Variables: 
%%% mode: latex
%%% TeX-master: "ug"
%%% End: 
