% Copyright (C) 2010 The ESPResSo project
% Copyright (C) 2002,2003,2004,2005,2006,2007,2008,2009,2010 Max-Planck-Institute for Polymer Research, Theory Group, PO Box 3148, 55021 Mainz, Germany
%  
% This file is part of ESPResSo.
%   
% ESPResSo is free software: you can redistribute it and/or modify it
% under the terms of the GNU General Public License as published by the
% Free Software Foundation, either version 3 of the License, or (at your
% option) any later version.
%  
% ESPResSo is distributed in the hope that it will be useful, but
% WITHOUT ANY WARRANTY; without even the implied warranty of
% MERCHANTABILITY or FITNESS FOR A PARTICULAR PURPOSE.  See the GNU
% General Public License for more details.
%  
% You should have received a copy of the GNU General Public License
% along with this program.  If not, see <http://www.gnu.org/licenses/>.
%
\chapter{Plotting}

\begin{essyntax}
  plotObs \var{file} \{ \var{x1}:\var{y1} \var{x2}:\var{y2} \dots \}
  \opt{titles \{ \var{title1} \var{title2} \dots \}}
  \opt{labels \{ \var{xlabel} \opt{\var{ylabel}} \}} 
  \opt{scale \var{gnuplot-scale}}
  \opt{cmd \var{gnuplot-command}} 
  \opt{out \var{filebase}}
\end{essyntax}
Uses \textsc{gnuplot} to create plots of the data in \var{file} and
writes it to the file \var{filebase}\lit{.ps} (default:
\var{file}\lit{.ps}). The data in \var{file} should be stored
column-wise. $\var{x1}, \var{x2} \dots$ and $\var{y1}, \var{y2} \dots$
denote the columns used for the data of the x- and y-axis,
respectively.

\begin{arguments}
\item[\opt{titles \{ \var{title1} \var{title2} \dots \}}] can be used
  to specify the titles of the different plots
\item[\opt{labels \{ \var{xlabel} \opt{\var{ylabel}} \}}] will define the
  labels of the axis. If \var{ylabel} is omitted, the filename
  \var{file} is used as label for the y-axis.
\item[\opt{scale \var{gnuplot-scale}}] will define the scaling of the
  axis (\eg \lit{scale logscale xy}) (default: \lit{nologscale xy})
\item[\opt{cmd \var{gnuplot-command}}] allows to pass any other
  commands to gnuplot. For example, use
  \codebox{plotObs \dots cmd "set key left"} to adjust the titles on
  the left side.
\item[\opt{out \var{filebase}}] can be used to change the output
  file. By default, the plot will be written to \var{file}\lit{.ps}.
\end{arguments}

\subsection{Joining plots}
\begin{essyntax}
  plotJoin \{ \var{source1} \var{source2} \dots \} \var{final}
\end{essyntax}
Joins the plot files $\var{source1}, \var{source2}, \dots$ into a
single file \var{final}, while placing any two files on one page.
Note that the resulting files may be huge and therefore hard to print!


%%% Local Variables: 
%%% mode: latex
%%% TeX-master: "ug"
%%% End: 
