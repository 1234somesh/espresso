% Copyright (C) 2010,2011,2012,2013,2014,2015,2016 The ESPResSo project
% Copyright (C) 2002,2003,2004,2005,2006,2007,2008,2009,2010 
%   Max-Planck-Institute for Polymer Research, Theory Group
%  
% This file is part of ESPResSo.
%   
% ESPResSo is free software: you can redistribute it and/or modify it
% under the terms of the GNU General Public License as published by the
% Free Software Foundation, either version 3 of the License, or (at your
% option) any later version.
%  
% ESPResSo is distributed in the hope that it will be useful, but
% WITHOUT ANY WARRANTY; without even the implied warranty of
% MERCHANTABILITY or FITNESS FOR A PARTICULAR PURPOSE.  See the GNU
% General Public License for more details.
%  
% You should have received a copy of the GNU General Public License
% along with this program.  If not, see <http://www.gnu.org/licenses/>.
%
\chapter{Setting up the system}
\label{chap:setup}

\section{\texttt{setmd}: Setting global variables in TCL}
\newescommand{setmd}

\begin{essyntax}
\variant{1} setmd \var{variable}
\variant{2} setmd \var{variable} \opt{\var{value}}+
\end{essyntax}

Variant \variant{1} returns the value of the \es global variable
\var{variable}, variant \variant{2} can be used to set the variable
\var{variable} to \var{value}. The '+' in variant \variant{2} means
that for some variables more than one \var{value} can be given
(example: setmd boxl 5 5 5). The following global variables can be
set:

%% List-environment for the description of the global variables
\newenvironment{globvar}{
  \begin{list}{}{
      \setlength{\rightmargin}{1em}
      \setlength{\leftmargin}{2em}
      \setlength{\partopsep}{0pt}
      \setlength{\topsep}{1ex}
      \setlength{\parsep}{0.5ex}
      \setlength{\listparindent}{-1em}
      \setlength{\labelwidth}{0.5em}
      \setlength{\labelsep}{0.5em}
      \renewcommand{\makelabel}[1]{%
        \index{##1@\texttt{##1} (global variable)|mainindex}%
        \index{global variables!\texttt{##1}|mainindex}%
        \texttt{##1}%
      }
    }
  }{
  \end{list}
}
\newcommand{\ro}{\emph{read-only}}

\todo{Better throw some out (\eg switches)?}
\todo{Missing: lattice_switch, dpd_tgamma, n_rigidbonds}
\todo{Which commands can be used to set the \emph{read-only}
  variables?}
\begin{globvar}
\item[box_l] (double[3]) Simulation box lengths of the cuboid box used by Espresso. Note that
  if you change the box length during the simulation, the folded
  particle coordinates will remain the same, i.e., the particle stay
  in the same image box, but at the same relative position in their
  image box. If you want to scale the positions, use the
  \lit{change_volume} command.
\item[cell_grid] (int[3], \ro) Dimension of the inner
  cell grid.
\item[cell_size] (double[3], \ro) Box-length of a cell.
\item[dpd_gamma] (double, \ro) Friction constant for the
  DPD thermostat.
\item[dpd_r_cut] (double, \ro) Cutoff for DPD thermostat.
\item[dpd_ignore_fixed_particles] (int, \ro) Switches fixed particle DPD force calculation ON (0) or OFF (1 default).
\item[gamma] (double \asep double[3], \ro) Friction constant for the
  Langevin thermostat. Feature \feature{PARTICLE_ANISOTROPY} requires 3
  values of the gamma corresponding to diagonal elements of the tensor.
  If one value is specified then other diagonal elements are defined
  equal to it automatically.
\item[gamma_rot] (double \asep double[3], \ro) Rotational friction constant for the
  Langevin thermostat. Feature \feature{ROTATIONAL_INERTIA} requires 3
  values of the gamma_rot corresponding to diagonal elements of the tensor.
  If one is not specified then the translational gamma value is used.
\item[integ_switch] (int, \ro) Internal switch which integrator to
  use.
\item[lb_components] (int, \ro) Number of fluid components.
\item[local_box_l] (int[3], \ro) Local simulation box length of the
  nodes.
\item[max_cut] (double, \ro) Maximal cutoff of real space
  interactions.
\item[max_cut_nonbonded] (double, \ro) Maximal cutoff of nonbonded
  real space interactions.
\item[max_cut_bonded] (double, \ro) Maximal cutoff of bonded
  real space interactions.
\item[max_num_cells] (int) Maximal number of cells for the link cell
  algorithm.  Reasonable values are between 125 and 1000, or for some
  problems (\var{n_total_particles} / \var{n_nodes}).
\item[max_part] (int, \ro) Maximal identity of a particle.
  \emph{This is in general not related to the number of particles!}
\item[max_range] (double, \ro) Maximal range of real space
  interactions: \var{max_cut} + \var{skin}.
\item[max_skin] (double, \ro) Maximal skin to be used for the link
  cell/verlet algorithm. This is the minimum of \var{cell\_size} -
  \var{max\_range}.
\item[min_global_cut] (double) Minimal total cutoff for real space.
  Effectively, this plus the skin is the minimally possible cell size.
  Espresso typically determines this value automatically, but some
  algorithms, \eg{} virtual sites, require you to specify it manually.
\item[min_num_cells] (int) Minimal number of cells for the link cell
  algorithm. Reasonable values range in $10^{-6} N^2$ to $10^{-7}
  N^2$. In general just make sure that the Verlet lists are not
  incredibly large. By default the minimum is 0, but for the automatic
  P3M tuning it may be wise to set larger values for high particle
  numbers.
\item[n_layers] (int, \ro) Number of layers in cell structure LAYERED
  (see section \vref{sec:cell-systems}).
\item[n_nodes] (int, \ro) Number of nodes.
\item[n_part] (int, \ro) Total number of particles.
\item[n_part_types] (int, \ro) Number of particle types that were
  used so far in the \keyword{inter} command (see chapter{tcl:inter}).
\item[node_grid] (int[3]) 3D node grid for real space domain
  decomposition (optional, if unset an optimal set is chosen
  automatically).
\item[nptiso_gamma0] (double, \ro)\todo{Docs missing.}
\item[nptiso_gammav] (double, \ro)\todo{Docs missing.}
\item[npt_p_ext] (double, \ro) Pressure for NPT simulations.
\item[npt_p_inst] (double) Pressure calculated during an
  NPT_isotropic integration.
\item[npt_piston] (double, \ro) Mass off the box when using NPT_isotropic
  integrator.
\item[periodicity] (bool[3]) Specifies periodicity for the three
  directions. If the feature PARTIAL_PERIODIC is set, Espresso can be instructed to treat some dimensions as non-periodic. Per default espresso assumes periodicity in all directions which equals setting this variable to (1,1,1). A dimension is specified as non-periodic via setting the periodicity variable for this dimension to 0. E.g. Periodicity only in z-direction is obtained by (0,0,1). Caveat: Be aware of the fact that making a dimension non-periodic does not hinder particles from leaving the box in this direction. In this case for keeping particles in the simulation box a constraint has to be set.
\item[skin] (double) Skin for the Verlet list.
\item [temperature] (double, \ro) Temperature of the
  simulation.
\item[thermo_switch] (double, \ro) Internal variable which thermostat
  to use. 
\item[time] (double) The simulation time.
\item[time_step] (double) Time step for MD integration.
\item[timings] (int) Number of samples to (time-)average over.
\item[transfer_rate] (int, \ro) Transfer rate for VMD connection. You
  can use this to transfer any integer value to the simulation from
  VMD.
\item[verlet_flag] (bool) Indicates whether the Verlet list will be
  rebuild. The program decides this normally automatically based on
  your actions on the data.
\item[verlet_reuse] (double) Average number of integration steps the
  verlet list has been re-used.
\item[warnings] (int) if non-zero (default), some warnings are printed
  out. Set this to zero if you get annoyed by them.
\item[sd_viscosity] (double) viscosity of the fluid for the Stokesian Dynamics
  simulation.
\item[sd_radius] (double) hydrodynamic radius of the particles used in
  Stokesian Dynamics.
\item[sd_seed] (int[2]) seed of the Stokes Dynamics random number generator.
\item[sd_random_state] (int[2]) offset of the random number
  generator. Together with the seed, the state of the random number generator
  is well defined.
\item[sd_precision_random] (double) precision used for the approximation of
  the square root of the mobility. Sometimes higher accuracy can speedup the
  simulation.
\end{globvar}


\section{Setting global variables in Python}
\label{sec:glob-vars-python}
In analogy to the TCL interface global system variables can be read and 
set in Python simply by accessing the attribute of the corresponding \es 
Python object. Those variables that are already available in the Python 
interface are listed in the following.

\vspace{1em}
Variables of the system class:
\begin{pysyntax}
  \attributes{
    espressomd
  }{
    System()
  }{
    box_l=\arg{list of 3 floats},
    periodicity=\arg{list of 3 ints},
    time=\arg{float},
    time_step=\arg{float},
    timings=\arg{int},
    transfer_rate,
    max_cut_bonded,
    max_cut_nonbonded,
    min_global_cut=\arg{float}
}
\end{pysyntax}

\vspace{1em}
Variables of the cell system module:
\begin{pysyntax}
  \attributes{
    espressomd.System()
  }{
    cell_system
  }{
    max_num_cells=\arg{int},
    min_num_cells=\arg{int},
    node_grid,
    skin=\arg{float}
}
\end{pysyntax}

\vspace{1em}
Some variables like \var{n\_part} or \var{temperature} are no longer directly 
available as attributes. In these cases they can be easily derived from the 
corresponding Python objects like
\begin{pycode}
  n_part = len(espressomd.System().part[:].pos)
\end{pycode}
or by calling the corresponding \texttt{get\_state} methods like
\begin{pycode}
  temperature = espressomd.System().thermostat.get_state()[0]['kT']
  gamma = espressomd.System().thermostat.get_state()[0]['gamma']
  gamma_rot = espressomd.System().thermostat.get_state()[0]['gamma_rotation']
\end{pycode}


\section{\texttt{thermostat}: Setting up the thermostat}
\label{sec:thermostat}
\newescommand{thermostat}

\begin{pysyntax}
  \attributes{
    espressomd.System()
  }{
    thermostat
  }{
    turn_off(),
    get_state()
}
\end{pysyntax}

\begin{essyntax}
  \variant{1} thermostat
  \variant{2} thermostat off
  \variant{3} thermostat \var{parameters}
\end{essyntax}

The \keyword{thermostat} command is used to change settings of the
thermostat. 

The different available thermostats will be described in the following
subsections. Note that for a simulation of the NPT ensemble, you need
to use a standard thermostat for the particle velocities (\eg Langevin
or DPD), and a thermostat for the box geometry (\eg the isotropic NPT
thermostat).

You may combine different thermostats at your own risk by turning them
on one by one. Note that there is only one temperature for all
thermostats, although for some thermostats like the Langevin thermostat,
particles can be assigned individual temperatures.

Since \es{} does not enforce a particular unit system, it cannot know
about the current value of the Boltzmann constant. Therefore, when
specifying the temperature of a thermostat, you actually do not define
the temperature, but the value of the thermal energy $k_B T$ in the
current unit system (see the discussion on units, Section~\ref{sec:units}).

Variant \variant{1} returns the thermostat parameters. A Tcl list is
given containing all the parameters needed to set the specific
thermostat. (exactly the same as the input command line, without the
preceding \texttt{thermostat}).

Variant \variant{2} turns off all thermostats and sets all thermostat 
variables to zero. Setting temperature to zero also affects the way in which 
electrostatics are handled (see also Section~\ref{sec:inter-electrostatics}).

Variant \variant{3} sets up one of the thermostats described below.

Note that their are three different types of noise which can be used
in \es. The one used typically in simulations is flat noise with the
correct variance and it is the default used in \es, though it can be
explicitly specified using the feature \feature{FLATNOISE}.  You can
also employ Gaussian noise which is, in some sense, more realistic.
Notably Gaussian noise (activated using the feature
\feature{GAUSSRANDOM}) does a superior job of reproducing higher order
moments of the Maxwell-Boltzmann distribution.  For typical generic
coarse-grained polymers using FENE bonds the Gaussian noise tends to
break the FENE bonds.  We thus offer a third type of noise, activate
using the feature \feature{GAUSSRANDOMCUT}, which produces Gaussian
random numbers but takes anything which is two standard deviations
($2\sigma$) below or above zero and set it to $-2\sigma$ or $2\sigma$
respectively. In all three cases the distribution is made such that
the second moment of the distribution is the same and thus results in
the same temperature.

\subsection{Langevin thermostat}

\begin{pysyntax}
  \object{
    espressomd.System().thermostat
  }{
    set_langevin
  }{
    kT=\arg{float},
    gamma=\arg{float or array of 3 floats},
    gamma_rotation=\arg{float or array of 3 floats}
}
\end{pysyntax}

\begin{essyntax}
  thermostat langevin \var{temperature} \var{gamma\_trans} \\
  \asep \var{g\_trans\_x} \var{g\_trans\_y} \var{g\_trans\_z} \\
  \opt{\alt{\var{gamma\_rotate} \asep
  \var{g\_rot\_x} \var{g\_rot\_y} \var{g\_rot\_z}}} \opt{on/off} \opt{on/off}
\end{essyntax}

\warning{The behavior of the Langevin Thermostat has changed in this version of \es{}. Before the term \var{gamma} was called the `friction coefficient', but was in fact the inverse relaxation time. The code has been modified such that this value is now a proper friction coefficient.

A warning message has been added to the source (core/thermostat.cpp), which can be commented out if you have verified that your simulation results are not affected by this change.}

The Langevin thermostat consists of a friction and noise term coupled
via the fluctuation-dissipation theorem. The friction term is a
function of the particle velocities. By specifying \var{gamma\_trans} the diffusion coefficient for the particle becomes
\begin{equation}
D = \frac{\var{temperature}}{\var{gamma\_trans}}.
\end{equation}
The relaxation time is given by \var{gamma\_trans}/MASS, with MASS the particle's mass. For a more detailed explanation,
refer to \cite{grest86a}. An anisotropic diffusion coefficient tensor is available to simulate anisotropic colloids (rods, etc.) properly.
It can be enabled by the feature \feature{PARTICLE_ANISOTROPY}.

If the feature \feature{ROTATION} is compiled in, the rotational
degrees of freedom are also coupled to the thermostat. If only the first two arguments are specified then the diffusion coefficient for the rotation is set to the same value as that for the translation.

\warning{This is again a change, since originally it was set to 1/3 of the value of \var{gamma\_trans} for some (unclear) reason. If you want to reproduce this behavior, this can be simply done by setting the value of \var{gamma\_rotate} appropriately.}

A separate rotational diffusion coefficient can be set by inputting \var{gamma\_rotate}. This also allows one to properly match the translational and rotational diffusion coefficients of a sphere.
Feature \feature{ROTATIONAL_INERTIA} enables an anisotropic rotational diffusion coefficient tensor through corresponding friction coefficients \var{g\_rot\_x} \var{g\_rot\_y} \var{g\_rot\_z}.
Finally, the two options \opt{on/off} allow one to switch the translational and rotational thermalization on or off separately, maintaining the frictional behavior. This can be useful, for instance, in high P{\'e}clet number active matter systems, where one only wants to thermalize the rotational degrees of freedom and translational motion is effected by the self-propulsion.

Using the Langevin thermostat, it is posible to set a temperature and
a friction coefficient for every particle individually via the feature
\feature{LANGEVIN_PER_PARTICLE}. Consult the reference of the
\texttt{part} command (chapter \ref{chap:part}) for information on how
to achieve this.

\warning{The behavior of the \feature{LANGEVIN_PER_PARTICLE} thermostat has undergone a similar change.}

\subsection{GHMC thermostat}

\begin{essyntax}
  thermostat ghmc \var{temperature} \var{n\_md} \var{phi} \opt{-no_flip | -flip
| -random_flip} \opt{-no_scale | -scale}
\end{essyntax}

\es implements Generalized Hybrid Monte Carlo (GHMC) as a thermostat. GHMC is a
simulation method for sampling the canonical ensemble \cite{mehlig92}. The
method consists of MC cycles that combine a few constant energy MD steps,
specified by \var{n\_md}, followed by a Metropolis criterion for their
acceptance. Prior to integration, the particles momenta are mixed with momenta
sampled from the appropriate Boltzmann distribution. 

Given the particles momenta $\mathbf{p}^j$ from the last $j^{th}$
GHMC cycle the new momenta are generated
by: $\mathbf{p}^{j+1}=\cos(\phi)\mathbf{p}^j+\sin(\phi)\pmb{\xi}$, where
$\pmb{\xi}$ is a noise vector of random Gaussian variables with zero mean
and variance $1/\var{temperature}$ (see 
\cite{horowitz91} for more details). The
momenta mixing parameter $\cos(\phi)$ corresponds to \var{phi} in the
implementation.

In case the MD step is rejected, the particles momenta may be
flipped. This is specified by setting the \keyword{-no_flip} /
\keyword{-flip} option, for the \keyword{-random_flip} option half of the
rejected MD steps randomly result in momenta flip. The default for momenta
flip is \keyword{-no_flip}. The $\pmb{\xi}$ noise vector's variance van be
tuned to exactly $1/\var{temperature}$ by specifying the \keyword{-scale}
option. The default for temperature scaling is \keyword{-no_scale}.


\subsection{Dissipative Particle Dynamics (DPD) } \label{sec:DPD}
\index{DPD|mainindex}

\es implements Dissipative Particle Dynamics (DPD) either via a global
thermostat, or via a thermostat and a special DPD interaction between
particle types.  The latter allows the user to specify friction
coefficients on a per-interaction basis.

\subsubsection{Thermostat DPD}
\index{label:DPDthermostat}

\begin{pysyntax}
  \object{
    espressomd.System().thermostat
  }{
    set_dpd
  }{
    kT=\arg{float},
    gamma=\arg{float},
    r_cut=\arg{float}
}
\end{pysyntax}

\begin{essyntax}
  thermostat dpd \var{temperature} \var{gamma} \var{r\_cut} \opt{ WF
    \var{wf}  \var{tgamma} \var{tr\_cut} TWF \var{twf}}
  \begin{features}
    \required{DPD} or \required{TRANS_DPD}
  \end{features}
\end{essyntax}

\es's standard DPD thermostat implements the thermostat exactly as
described in \cite{soddeman03a}.  We use the standard
\textit{Velocity-Verlet} integration scheme, \eg DPD only influences
the calculation of the forces. No special measures have been taken to
self-consistently determine the velocities and the dissipative forces
as it is for example described in \cite{Nikunen03}.  DPD adds a
velocity dependent dissipative force and a random force to the usual
conservative pair forces (\eg Lennard-Jones).

\noindent The dissipative force is calculated by
\begin{equation}
  \vec{F}_{ij}^{D} = -\zeta w^D (r_{ij}) (\hat{r}_{ij} \cdot \vec{v}_{ij}) \hat{r}_{ij}
\end{equation}

\noindent The random force by
\begin{equation}
  \vec{F}_{ij}^R = \sigma w^R (r_{ij}) \Theta_{ij} \hat{r}_{ij}
\end{equation}
where $ \Theta_{ij} \in [ -0.5 , 0.5 [ $ is a uniformly distributed
random number.  The connection of $\sigma $ and $\zeta $ is given by
the dissipation fluctuation theorem:
\begin{equation}
  (\sigma w^R (r_{ij})^2=\zeta w^D (r_{ij}) \text{k}_\text{B} T
\end{equation}

The parameters \var{gamma} and \var{r_cut} define the strength of the
friction $\zeta$ and the cutoff radius.

According to the optional parameter WF (can be set to 0 or 1, default
is 0) of the thermostat command the functions $w^D$ and $w^R$ are
chosen in the following way ( $ r_{ij} < \var{r\_cut} $ ) :

\begin{equation}
w^D (r_{ij}) = ( w^R (r_{ij})) ^2 = 
   \left\{
   \begin{array}{clcr} 
             {( 1 - \frac{r_{ij}}{r_c}} )^2 & , \; \var{wf} = 0 \\
             1                      & , \; \var{wf} = 1
   \end{array}
   \right.
 \end{equation}

 For $ r_{ij} \ge \var{r\_cut} $ $w^D$ and $w^R$ are identical to 0 in
 both cases.

The friction (dissipative) and noise (random) term are coupled via the
fluctuation- dissipation theorem. The friction term is a function of
the relative velocity of particle pairs.  The DPD thermostat is better
for dynamics than the Langevin thermostat, since it mimics
hydrodynamics in the system.

When using a Lennard-Jones interaction, $\var{r\_cut} =
2^{\frac{1}{6}} \sigma$ is a good value to choose, so that the
thermostat acts on the relative velocities between nearest neighbor
particles.  Larger cutoffs including next nearest neighbors or even
more are unphysical.

\var{gamma} is basically an inverse timescale on which the system
thermally equilibrates.  Values between $0.1$ and $1$ are o.k, but you
propably want to try this out yourself to get a feeling for how fast
temperature jumps during a simulation are. The dpd thermostat does not
act on the system center of mass motion.  Therefore, before using dpd,
you have to stop the center of mass motion of your system, which you
can achieve by using the command \keyword{galilei_transform} 
\ref{sec:Galilei}. This may be repeated once in a while for long 
runs due to round off errors (check this with the command 
\keyword{system_CMS_velocity}) \ref{sec:Galilei}.

Two restrictions apply for the dpd implementation of \es:

\begin{itemize}
\item As soon as at least one of the two interacting particles is
  fixed (see~\ref{chap:part} on how to fix a particle in space) the
  dissipative and the stochastic force part is set to zero for both
  particles (you should only change this hardcoded behaviour if you
  are sure not to violate the dissipation fluctuation theorem).
\item \texttt{DPD} does not take into account any internal rotational
  degrees of freedom of the particles if \texttt{ROTATION} is switched
  on. Up to the current version DPD only acts on the translatorial
  degrees of freedom.
\end{itemize}

\paragraph{Transverse DPD thermostat}\label{sec:transDPD}
This is an extension of the above standard DPD thermostat
\cite{junghans2008}, which dampens the degrees of freedom
perpendicular on the axis between two particles. To switch it on, the
feature \feature{TRANS_DPD} is required instead of the feature
\texttt{DPD}.

The dissipative force is calculated by
\begin{equation}
  \vec{F}_{ij}^{D} = -\zeta w^D (r_{ij}) (I-\hat{r}_{ij}\otimes\hat{r}_{ij}) \cdot \vec{v}_{ij}
\end{equation}

The random force by
\begin{equation}
  \vec{F}_{ij}^R = \sigma w^R (r_{ij}) (I-\hat{r}_{ij}\otimes\hat{r}_{ij}) \cdot \vec{\Theta}_{ij}
\end{equation}

The parameters \var{tgamma} \var{tr\_cut} define the strength of the
friction and the cutoff in the same way as above.  Note: This
thermostat does \emph{not} conserve angular momentum.

\subsubsection{Interaction DPD}\label{sec:interDPD}
\index{Interaction DPD}

\begin{essyntax}
  thermostat inter_dpd \var{temperature}
  \begin{features}
    \required{INTER_DPD}
  \end{features}
\end{essyntax}

Another way to use DPD is by using the interaction DPD. In this case,
DPD is implemented via a thermostat and corresponding interactions.
The above command will set the global temperature of the system, while
the friction and other parameters have to be set via the command
\texttt{inter inter_dpd} (see \vref{sec:DPDinter}).  This allows to
set the friction on a per-interaction basis.

DPD interactions with fixed particles is switched off by default, because
it is not clear if the results obtained with that method are physically 
correct. If you want activate \texttt{inter_dpd} with fixed particles please 
use:
\begin{essyntax}
  setmd dpd_ignore_fixed_particles 0
  \begin{features}
    \required{INTER_DPD}
  \end{features}
\end{essyntax}
By default the fixed particles are ignored (\texttt{dpd_ignore_fixed_particles} is 1).

\subsubsection{Other DPD extensions}

The features \feature{DPD_MASS_RED} or \feature{DPD_MASS_LIN} make the
friction constant mass dependent:
\begin{equation*}
\zeta \to \zeta M_{ij}
\end{equation*}

\noindent There are two implemented cases. 
\begin{itemize}
\item \feature{DPD_MASS_RED} uses the reduced mass: $M_{ij}=2\frac{m_i
    m_j}{m_i+m_j}$
\item \feature{DPD_MASS_LIN} uses the real mass:
  $M_{ij}=\frac{m_i+m_j}{2}$
\end{itemize}
The prefactors are such that equal masses
result in a factor $1$.

\subsection{Isotropic NPT thermostat}
\label{ssec:NPTthermostat}

\begin{pysyntax}
  \object{
    espressomd.System().thermostat
  }{
    set_npt
  }{
    kT=\arg{float},
    gamma0=\arg{float},
    gammav=\arg{float}
}
\end{pysyntax}


\begin{essyntax}
  thermostat npt_isotropic \var{temperature} \var{gamma0} \var{gammaV}
  \begin{features}
    \required{NPT}
  \end{features}
\end{essyntax}

This theormstat is based on the Anderson thermostat (see
\cite{andersen80a, mann05d}) and will thermalize the box geometry. It
will only do isotropic changes of the box.

Be aware that this feature is neither properly examined for all
systems nor is it maintained regularly. If you use it and notice
strange behaviour, please contribute to solving the problem.
\subsection{CPU thermostat}

\begin{essyntax}
  thermostat cpu \var{temperature}
\end{essyntax}

The CPU thermostat consists of a number crunching and a time wasting
term coupled via the central processing unit's thermal diode. For a
more detailed explanation, refer to German Patent No. DE~10~2015~04F00L.

Using the CPU thermostat, it is posible to keep your central
processing unit operating at a constant temperature.

Be aware that this thermostat requires \var{temperature} to be given
in Kelvin.

\subsection{Stokesian Dynamics thermostat}
\label{ssec:SDthermostat}
\begin{essyntax}
  thermostat sd \var{temperature}
  \begin{features}
    \required{CUDA} and \required{SD}
  \end{features}
\end{essyntax}

This thermostat should be used together with the Stokesian Dynamics
implementation.
No other thermostat is able to thermalize SD correctly.
The precision of the farfield contribution of the thermostat can be tuned
with

\section{\texttt{nemd}: Setting up non-equilibrium MD}
\newescommand{nemd}
\label{sec:NEMD}
\index{NEMD}

\begin{essyntax}
  \variant{1}nemd exchange \var{n\_slabs} \var{n\_exchange}
  \variant{2}nemd shearrate \var{n\_slabs} \var{shearrate}
  \variant{3}nemd off
  \variant{4}nemd
  \variant{5}nemd profile
  \variant{6}nemd viscosity
  \begin{features}
    \required{NEMD}
  \end{features}
\end{essyntax}

Use NEMD (Non Equilibrium Molecular Dynamics) to simulate a system
under shear with help of an unphysical momentum change in two slabs in
the system.

Variants \variant{1} and \variant{2} will initialise NEMD. Two
distinct methods exist.  Both methods divide the simulation box into
\var{n\_slab} slabs that lie parallel to the x-y-plane and apply a
shear in x direction.  The shear is applied in the top and the middle
slabs. Note, that the methods should be used with a DPD thermostat or
in an NVE ensemble.  Furthermore, you should not use other special
features like \keyword{part fix} or \keyword{constraints} inside the
top and middle slabs. For further reference on how NEMD is implemented
into \es see \cite{soddeman01a}.


\index{momentum exchange method}
Variant \variant{1} chooses the momentum exchange method.  In this
method, in each step the \var{n\_exchange} largest positive
x-components of the velocity in the middle slab are selected and
exchanged with the \var{n\_exchange} largest negative x-components of
the velocity in the top slab. 

\index{shear-rate method}
Variant \variant{2} chooses the shear-rate method. In this method, the
targetted x-component of the mean velocity in the top and middle slabs
are given by 
\begin{equation}
  \var{target\_velocity} = \pm \var{shearrate}\,\frac{L_z}{4}
\end{equation}
where $L_z$ is the simulation box size in z-direction. During the
integration, the x-component of the mean velocities of the top and
middle slabs are measured.  Then, the difference between the mean
x-velocities and the target x-velocities are added to the x-component
of the velocities of the particles in the respective slabs. 

Variant \variant{3} will turn off NEMD, variant \variant{4} will
print usage information of the parameters of NEMD. Variant \variant{5} will return the
velocity profile of the system in x-direction (mean velocity per slab).

Variant \variant{6} will return the viscosity of the system, that is
computed via
\begin{equation}
  \eta = \frac{F}{\dot{\gamma} L_x L_y}
\end{equation}
where $F$ is the mean force (momentum transfer per unit time) acting
on the slab, $L_x L_y$ is the area of the slab and $\dot{\gamma}$ is the shearrate. 

NEMD as implemented generates a Pouseille flow, with shear flow rate varying over a finite wavelength determined by the box. For a planar Couette flow (constant shear, infinite wavelength), consider using Lees-Edwards boundary conditions (see \vref{sec:lees-edwards}) to drive the shear.

\section{\texttt{cellsystem}: Setting up the cell system}
\label{sec:cell-systems}
\newescommand{cellsystem}

This section deals with the flexible particle data organization of
\es.  Due to different needs of different algorithms, \es is able to
change the organization of the particles in the computer memory,
according to the needs of the used algorithms. For details on the
internal organization, refer to section
\vref{sec:internal-particle-organization}.

\begin{pysyntax}
  \attributes{
    espressomd.System()
  }{
    cell_system
  }{
    skin=\arg{float}
    max_num_cells=\arg{float}
    min_num_cells=\arg{float}
    get_state()
}
\end{pysyntax}


\subsection{Domain decomposition}
\index{domain decomposition}

\begin{pysyntax}
  \object{
    espressomd.System().cell_system
  }{
    set_domain_decomposition
  }{
    use_verlet_lists=\arg{bool}
}
\end{pysyntax}


\begin{essyntax}
  cellsystem domain_decomposition \opt{-no_verlet_list}
\end{essyntax}
This selects the domain decomposition cell scheme, using Verlet lists
for the calculation of the interactions. If you specify
\keyword{-no_verlet_list}, only the domain decomposition is used, but
not the Verlet lists.

The domain decomposition cellsystem is the default system and suits
most applications with short ranged interactions. The particles are
divided up spatially into small compartments, the cells, such that the
cell size is larger than the maximal interaction range. In this case
interactions only occur between particles in adjacent cells. Since the
interaction range should be much smaller than the total system size,
leaving out all interactions between non-adjacent cells can mean a
tremendous speed-up. Moreover, since for constant interaction range,
the number of particles in a cell depends only on the density. The
number of interactions is therefore of the order N instead of order
$N^2$ if one has to calculate all pair interactions.

\subsection{N-squared}

\begin{pysyntax}
  \object{
    espressomd.System().cell_system
  }{
    set_n_square
  }{
    use_verlet_lists=\arg{bool}
}
\end{pysyntax}


\begin{essyntax}
  cellsystem nsquare 
\end{essyntax}
This selects the very primitive nsquared cellsystem, which calculates
the interactions for all particle pairs. Therefore it loops over all
particles, giving an unfavorable computation time scaling of $N^2$.
However, algorithms like MMM1D or the plain Coulomb interaction in the
cell model require the calculation of all pair interactions.

In a multiple processor environment, the nsquared cellsystem uses a
simple particle balancing scheme to have a nearly equal number of
particles per CPU, \ie $n$ nodes have $m$ particles, and $p-n$ nodes
have $m+1$ particles, such that $n*m+(p-n)*(m+1)=N$, the total number
of particles. Therefore the computational load should be balanced
fairly equal among the nodes, with one exception: This code always
uses one CPU for the interaction between two different nodes. For an
odd number of nodes, this is fine, because the total number of
interactions to calculate is a multiple of the number of nodes, but
for an even number of nodes, for each of the $p-1$ communication
rounds, one processor is idle.

E.g. for 2 processors, there are 3 interactions: 0-0, 1-1, 0-1.
Naturally, 0-0 and 1-1 are treated by processor 0 and 1, respectively.
But the 0-1 interaction is treated by node 1 alone, so the workload
for this node is twice as high. For 3 processors, the interactions are
0-0, 1-1, 2-2, 0-1, 1-2, 0-2. Of these interactions, node 0 treats 0-0
and 0-2, node 1 treats 1-1 and 0-1, and node 2 treats 2-2 and 1-2.

Therefore it is highly recommended that you use nsquared only with an
odd number of nodes, if with multiple processors at all. 

\subsection{Layered cell system}

\begin{pysyntax}
  \object{
    espressomd.System().cell_system
  }{
    set_layered
  }{
    n_layers=\arg{int}
}
\end{pysyntax}


\begin{essyntax}
  cellsystem layered \var{n\_layers}
\end{essyntax}

This selects the layered cell system, which is specifically designed
for the needs of the MMM2D algorithm. Basically it consists of a
nsquared algorithm in x and y, but a domain decomposition along z, i.
e. the system is cut into equally sized layers along the z axis. The
current implementation allows for the cpus to align only along the z
axis, therefore the processor grid has to have the form 1x1xN.
However, each processor may be responsible for several layers, which
is determined by \var{n\_layers}, i. e. the system is split into
N*\var{n\_layers} layers along the z axis. Since in x and y direction
there are no processor boundaries, the implementation is basically
just a stripped down version of the domain decomposition cellsystem.

\section{CUDA}
\label{sec:cuda}
\begin{essyntax}
  \variant{1} cuda list
  \variant{2} cuda setdevice \var{id}
  \variant{3} cuda getdevice
\end{essyntax}

This command can be used to choose the GPU for all subsequent
GPU-computations. Note that due to driver limitations, the GPU cannot
be changed anymore after the first GPU-using command has been issued,
for example \texttt{lbfluid}. If you do not choose the GPU manually
before that, CUDA internally chooses one, which is normally the most
powerful GPU available, but load-independent.

Variant \variant{1} lists the available devices by their ids and brand
names. Variant \variant{2} allows to choose the device by its id,
which can be determined using \texttt{cuda list}, or for example the
\texttt{deviceQuery} example code in the CUDA SDK. Variant \variant{3}
finally gives the id of the currently active GPU.


\section{Creating bonds when particles collide}

\begin{citebox}
  Please cite~\citewbibkey{espresso2} when using dynamic bonding.
\end{citebox}


\begin{essyntax}
\variant{1} on\_collision
\variant{2} on\_collision off
\variant{3} on\_collision \opt{exception} bind_centers \var{d} \var{bond1}
\variant{4} on\_collision \opt{exception} bind_at_point_of_collision \var{d} \var{bond1} \var{bond2} \var{type}
\variant{5} on\_collision \opt{exception} glue_to_surface \var{d} \var{bond1} \var{bond2} \var{type} \var{type2} \var{type3} \var{type4} \var{d2}
\variant{6} on\_collision \opt{exception} bind_three_particles \var{d} \var{bond1} \var{bond2} \var{res}
\end{essyntax}

With the help of the feature \feature{COLLISION_DETECTION}, bonds
between particles can be created automatically during the simulation,
every time two particles collide. This is useful for simulations of
chemical reactions and irreversible adhesion processes.

Two methods of binding are available:
\begin{itemize}
\item \lit{bind_centers} adds a bonded interaction between the
  colliding particles at the first collision. This leads to the
  distance between the particles being fixed, the particles can,
  however still slide around each other.

  The parameters are as follows: \var{d} is the distance at which the
  bond is created. \var{bond1} denotes a pair bond and is the type of
  the bond created between the colliding particles. Particles that are
  already bound by a bond of this type do not get a new bond, in order
  to avoid creating multiple bonds.

\item \lit{bind_at_point_of_collision} prevents sliding of the
  particles at the contact. This is achieved by creating two virtual
  sites at the point of collision. They are rigidly connected to the
  colliding particles, respectively. A bond is then created between
  the virtual sites, or an angular bond between the two real particles
  and the virtual particles. In the latter case, the virtual particles
  are the centers of the angle potentials (particle 2 in the
  description of the angle potential, see \ref{sec:angle}). Due to the
  rigid connection between each of the particles in the collision and
  its respective virtual site, a sliding at the contact point is no
  longer possible.  See the documentation on rigid bodies for
  details. In addition to the bond between the virtual sites, the bond
  between the colliding particles is also created. You can either use
  a real bonded interaction to prevent wobbling around the point of
  contact or you can use a virtual bond to prevent additional force
  contributions, at the expense of RATTLE, see \ref{sec:rattle}.

  The parameters \var{d} and \var{bond1} are the same as for the
  \lit{bind_centers} method. \var{bond2} determines the type of the
  bond created between the virtual sites (if applicable), and can be
  either a pair or a triple (angle) bond. If it is a pair bond, it
  connects the two virtual particles, otherwise it constraints the
  angle between the two real particles around the virtual
  ones. \var{type} denotes the particle type of the virtual sites
  created at the point of collision (if applicable). Be sure not to
  define a short-ranged interaction for this particle type, as two
  particles will be generated in the same place.

\item \lit{glue_to_surface} is used to fix a particle of type
  \var{type2} onto the surface of a particle of type \var{type3}. This
  is achieved by creating a virtual site (particle type \var{type})
  which is rigidly connected to the particle of \var{type3}. A bond of
  type \var{bond2} is then created between the virtual site and the
  particle of \var{type}.  Additionally, a bond of type \var{bond1}
  between the colliding particles is also created.  After the collision,
  the particle of type \var{type3} is changed to type \var{type4}.

\item \lit{bind_three_particles} allows for the creation of agglomerates
  which maintain their shape similarly to those create by the
  \lit{bind_at_point_of_collision} method. The present approach works
  without virtual sites. Instead, for each two-particle collision, the
  surrounding is searched for a third particle. If one is found, angular
  bonds are placed on each of the three particles in addition to the
  distance based bonds between the particle centers. The id of the
  angular bonds is determined from the angle between the particles. Zero
  degrees corresponds to bond id \var{bond2}, whereas 180 degrees
  corresponds to bond id \var{bond2} +\var{res}.  This method das not
  depend on the particles' rotational degrees of freedom being
  integrated. Virtual sites are also not required, and the method is
  implemented to run on more than one cpu core.
\end{itemize}

The code can throw an exception (background error) in case two
particles collide for the first time, if the \lit{exception} keyword
is added to the invocation. In conjunction with the \lit{catch}
command of Tcl, this can be used to intercept the collision:
\begin{tclcode}
if {[catch {integrate 0} err]} {
  foreach exception [lrange $err 2 end] {
    if {[lrange $exception 0 2] == "collision between particles"} {
      set i [lindex $exception 3]
      set j [lindex $exception 5]
      puts "particles $i and $j collided"
    }
  }
}
\end{tclcode}

The following limitations currently apply for the collision detection:
\begin{itemize}
\item The method is currently limited to simulations with a single cpu
\item No distinction is currently made between different particle types
\item The ``bind at point of collision'' approach requires the
  \feature{VIRTUAL_SITES_RELATIVE} feature
\item The ``bind at point of collision'' approach cannot handle
  collisions between virtual sites
\end{itemize}

\section{Catalytic Reactions}
\label{sec:Reactions}

With the help of the feature \feature{CATALYTIC_REACTIONS}, one can
define three particle types to act as reactant (e.g. $H_2O_2$), catalyzer (e.g. platinum), and
products (e.g. $O_2$ and $H_2O$). Using these reaction categories, we model the following
chemical reaction system which is not thermodynamically consistent but rather intended to simulate active swimmers and their propulsion:
\begin{eqnarray}
rt & \rightleftharpoons & pr ; \\
rt & \xrightarrow{ct} & pr.
\end{eqnarray}
The first line indicates that there is a reversible chemical
reaction in the bulk that converts the reactant particles (\var{rt}) into product
(\var{pr}) particles, leading to an equilibrium state. This reaction is intended to artificially recover the reactant (\var{rt}) particles in this model. In the case of $H_2O_2$ this is artificial since it does not spontaneously build up if oxygen is dissolved in water. The second line
indicates that in the vicinity of a catalyst (\var{ct}) the forward
reaction takes place, i.e., conversion of reactants into
products. Of course the decompositon of a reactand into a product also takes place if there is no catalyst (since a catalyst has no effect on the chemical equilibrium) however the reaction is much faster than normally in the presence of a catalyst and the normal decomposition is neglected since it takes place so slowly. This is correct chemistry for waterperoxide since it spontaneously decomposes almost completely and much faster in the presence of a catalyst.

The equilibrium reaction is described by the equilibrium
constant
\begin{equation}
K_{\text{eq}} = \frac{k_{\text{eq,+}}}{k_{\text{eq,-}}} = \frac{[pr]}{[rt]},
\end{equation}
with $[rt]$ and $[pr]$ the reactant and product concentration and
$k_{\text{eq,$\pm$}}$ the forward and backward reaction rate
constants, respectively. The rate constants that specify the change in
concentration for the equilibrium and catalytic reaction are given by
\begin{eqnarray}
\frac{d[rt]}{dt} & = & k_{\text{eq,-}}[pr] - k_{\text{eq,+}}[rt] ; \\
\frac{d[pr]}{dt} & = & k_{\text{eq,+}}[rt] - k_{\text{eq,-}}[pr] ; \\
-\frac{d[rt]}{dt} \;\; = \;\; \frac{d[pt]}{dt} & = & k_{\text{ct}}[rt] ,
\end{eqnarray}
respectively.

In the current \es implementation we assume $k_{\text{eq,+}} =
k_{\text{eq,-}} \equiv k\_eq$ and therefore $K_{\text{eq}}=1$. The
user can specify $k\_eq \ge 0$ and $k\_ct \equiv k_{\text{ct}} >
0$. The former rate constant is applied to all reactant and product
particles in the system, whereas the latter is applied only to the
reactant particles in the vicinity of a catalyst particle. Reactant
particles that have a distance of \var{r} or less to at least one
catalyzer particle are therefore converted into product particles with
rate constant $k\_eq + k\_ct$. The conversion of particles is done
stochastically on the basis of the relevant rate constant (\var{k}
$\ge$ 0):
\begin{equation}
\label{eq:rate} P_{\text{cvt}} = 1 - \exp \left( - k  \Delta t  \right) ,
\end{equation}
with $P_{\text{cvt}}$ the probability of the conversion and $\Delta t$
the integration time step. If the equilibrium rate constant is not
specified it is assumed that \var{k\_eq} = 0.


\begin{pysyntax}
  \object{
    espressomd.reaction
  }{
    Reaction
  }{
    product_type   = \arg{int},
    reactant_type  = \arg{int},
    catalyzer_type = \arg{int},
    ct_range       = \arg{float},
    ct_rate        = \arg{float},
    eq_rate        = \arg{float},
    react_once     = \arg{bool},
    swap           = \arg{bool}
  }
\end{pysyntax}

\begin{essyntax}
\variant{0} reaction reactant_type \var{rt} catalyzer_type \var{ct} 
product_type \var{pt} range \var{r} ct_rate \var{k\_ct}
\opt{eq_rate \var{k\_eq}} \opt{react_once \var{on/off}} \opt{swap \var{on/off}}
\variant{1} reaction off
\variant{2} reaction print
\begin{features}
  \required{CATALYTIC_REACTIONS\footnote{The current implementation
      also requires the use of verlet lists and domain
      decomposition.}}
\end{features}
\end{essyntax}

\begin{itemize}
\item Variant \variant{0} defines a reaction with particles of type
  number \var{rt} as reactant, type \var{ct} as catalyzer and type
  \var{pt} as product\footnote{Only one type of particle can be
    assigned to each of these three reaction species and no particle
    type may be assigned to multiple species. That is, \es currently
    does not support particles of type 1 and 2 both to be reactants,
    nor can particles of type 1 be a reactant as well as a
    catalyst. Moreover, only one of these reactions can be implemented
    in a single Tcl script. If, for instance, there is a reaction
    involving particle types 1, 2, and 4, there cannot be a second
    reaction involving particles of type 5, 6, and 8. It is however
    possible to modify the reaction properties for a given set of
    types during the simulation.}. The catalytic reaction rate
  constant is given by \var{k\_ct}\footnote{Currently only strictly
    positive values of the catalytic conversion rate constant are
    allowed. Setting the value to zero is equivalent to
    \texttt{reaction off}.} and to override the default rate constant
  for the equilibrium reaction (\var{k\_eq} = 0), one can specify it
  by \var{k\_eq}. By default each reactant particle is checked against
  each catalyst particle (\texttt{react\_once} \emph{off}). However,
  when creating smooth surfaces using many catalyst particles, it can
  be desirable to let the reaction rate be independent of the surface
  density of these particles. That is, each particle has a likelihood
  of reacting in the vicinity of the surface (distance is less than
  $r$) as specified by the rate constant, i.e., \emph{not} according
  to $P_{\text{cvt}} = 1 - \exp \left( - n k\Delta t \right)$, with
  $n$ the number of local catalysts. To accomplish this, each reactant
  is considered only once each time step by using the option
  \texttt{react\_once} \emph{on}. The reaction command is set up such
  that the different properties may be influenced individually.
\item Variant \variant{1} disables the reaction. Note that at the
  moment, there can only be one reaction in the simulation.
\item Variant \variant{2} returns the current reaction parameters.
\end{itemize}

The Python interface has some modified capabilities with respect to
the TCL interface.  For example, you can alter parameters using the
\texttt{.setup()} method of the reaction instance.  The reaction
mechanism can be inhibited and restarted using \texttt{.stop()} and
\texttt{.start()}.

In future versions of \es the capabilities of the
\feature{CATALYTIC_REACTIONS} feature may be generalized to handle
multiple reactant, catalyzer, and product types, as well as more
general reaction schemes. Other changes may involve merging the
current implementation with the \feature{COLLISION_DETECTION} feature.

\section{Galilei Transform and Particle Velocity Manipulation}
\label{sec:Galilei}

The following commands may be useful in effecting the velocity of the
system.

\subsection{Particle motion and rotation}

\begin{pysyntax}
  \object{
      espressomd.galilei
  }{
      kill_particle_motion
  }{
      rotation = 0 or 1
  }
\end{pysyntax}
\begin{essyntax}
  kill_particle_motion \require{1}{\opt{rotation}}
  \begin{features}
    \required[1]{ROTATION}
  \end{features}
\end{essyntax}
This command halts all particles in the current simulation, setting
their velocities to zero, as well as their angular momentum if the
option \texttt{rotation} is specified and the feature ROTATION has
been compiled in.

\subsection{Forces and torques acting on the particles}

\begin{pysyntax}
  \object{
      espressomd.galilei
  }{
      kill_particle_forces
  }{
      torques = 0 or 1
  }
\end{pysyntax}
\begin{essyntax}
  kill_particle_forces \require{1}{\opt{torques}}
  \begin{features}
    \required[1]{ROTATION}
  \end{features}
\end{essyntax}
This command sets all forces on the particles to zero, as well as all
torques if the option \texttt{torque} is specified and the feature
ROTATION has been compiled in. 

\subsection{The centre of mass of the system}

\begin{pysyntax}
  \object{espressomd.galilei
  }{
      system_CMS
  }{
  }
\end{pysyntax}
\begin{essyntax}
  system_CMS
\end{essyntax}
Returns the center of mass of the whole system. It currently does not
factor in the density fluctuations of the Lattice-Boltzman fluid.

\subsection{The centre-of-mass velocity}

\begin{pysyntax}
  \object{espressomd.galilei
  }{
      system_CMS_velocity
  }{
  }
\end{pysyntax}
\begin{essyntax}
  system_CMS_velocity
\end{essyntax}
Returns the velocity of the center of mass of the whole system.

\subsection{The Galilei transform}

\begin{pysyntax}
  \object{espressomd.galilei
  }{
      galilei_transform
  }{
  }
\end{pysyntax}
\begin{essyntax}
  galilei_transform
\end{essyntax}
Substracts the velocity of the center of mass of the whole system from
every particle's velocity, thereby performing a Galilei transform into
the reference frame of the center of mass of the system. This
transformation is useful for example in combination with the DPD
thermostat, since there, a drift in the velocity of the whole system
leads to an offset in the reported temperature.

%%% Local Variables: 
%%% mode: latex
%%% TeX-master: "ug"
%%% End: 
