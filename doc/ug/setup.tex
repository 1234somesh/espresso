\chapter{Setting up the system}
\label{chap:setup}

\section{\texttt{part}: Setting up particles}
\label{sec:part}


\section{\texttt{polymer}: Setting up polymer chains}
\label{sec:polymer}

\tclcommand{polymer}
{\var{num\_polymers} 
  \var{monomers\_per\_chain} 
  \var{bond\_length}\\{}
  \opt{start \var{part\_id}}
  \opt{pos \var{x} \var{y} \var{z}}
  \opt{mode < RW | SAW | PSAW > [\var{shield} [\var{max\_try}]]}
  \opt{charge \var{val\_charged\_monomer}}
  \opt{distance \var{dist\_charged\_monomer}}
  \opt{types \var{type\_neutral\_monomer} [\var{type\_charged\_monomer}]}
  \opt{bond \var{type\_bond}}
  \opt{angle \var{phi} [\var{theta} [\var{x} \var{y} \var{z}]]}
}

This command will create \var{num\_polymers} polymer or
polyelectrolyte chains with \var{monomers\_per\_chain} monomers per
chain. The length of the bond between two adjacent monomers will be
set up to be \var{bond\_length}.

\begin{arguments}
\item[\var{num\_polymers}] Sets the number of polymer chains.
\item[\var{monomers\_per\_chain}] Sets the number of monomers per
  chain.
\item[\var{bond\_length}] Sets the initial distance between two
  adjacent monomers.
\item[\opt{start \var{part\_id}}] Sets the particle number of the
  start monomer to be used with the \keyword{part} command. This
  defaults to 0.

\item[\opt{pos \var{x} \var{y} \var{z}}] Sets the position of the
  first monomer in the chain to \var{x}, \var{y}, \var{z} (defaults to
  a randomly chosen value)
  
\item[\opt{mode < RW | SAW | PSAW > [\var{shield} [\var{max\_try}]]}]
  Selects the setup mode: Self avoiding walk (SAW) or plain random
  walk (RW) (defaults to 'SAW').  If 'SAW' was selected, the position
  randomly chosen for the current monomer to be placed is dismissed
  every time it would get closer to another particle's position than
  \var{shield} (defaults to '0.0'); the attempt to find a suitable
  unobstructed random place for the current monomer is then repeated
  for up to \var{max\_try} times (defaults to '30000').
  
\item[\opt{charge \var{val\_charged\_monomer}}] Sets the valency of
  the charged monomers.  If the valency of the charged polymers
  \var{val\_charged\_monomer} is smaller than $10^{-10}$, the charge
  is assumed to be zero, and the types are set to
  \var{type\_charged\_monomer} = \var{type\_neutral\_monomer}. This
  defaults to 0.0.

\item[\opt{distance \var{dist\_charged\_monomer}}] Sets the stride
  between the indices of two charged monomers. This defaults defaults
  to 1, meaning that all monomers in the chain are charged.
  
\item[\opt{types \var{type\_neutral\_monomer}
    \var{type\_charged\_monomer}}] Sets the type numbers of the
  neutral and charged monomer types to be used with the \keyword{part}
  command. If \var{type\_neutral\_monomer} is defined,
  \var{type\_charged\_monomer} defaults to 1. If the option is
  omitted, both monomer types default to 0.
  
\item[\opt{bond \var{type\_bond}}] Sets the type number of the bonded
  interaction to be set between the monomers. This defaults to 0.
  
\item[\opt{angle \var{phi} [\var{theta} [\var{x} \var{y} \var{z}]]}]
  Allows for setting up helices or planar polymers: \var{phi} sets
  the angle $\phi$ and \var{theta} sets the angle $\theta$ between
  adjacent bonds. \var{x}, \var{y} and \var{z} set the position of the
  second monomer of the first chain.
\end{arguments}

\section{\texttt{inter}: Setting up interactions}
\label{sec:inter}

\subsection{Nonbonded interactions}
\label{sec:inter_nb}
%\quickrefheading{Nonbonded interactions}

\tclcommand[LENNARD\_JONES]{inter}{%
  \var{type1 type2} 
  lennard\_jones 
  \var{epsilon sigma cutoff shift offset}
}

Defines a Lennard-Jones interaction between particles of the types
\var{type1} and \var{type2}.
\bigskip

\subsection{Bonded interactions}
\label{sec:inter_bonded}

\index{Bonded interactions} \index{Bonded interaction type id} Bonded
interactions possess an \emph{bonded interaction type id}. On the one
hand, this id is used when particles and bonds between particles are
specified in the command \texttt{part} (see section \vref{sec:part}).
On the other hand, the id is used when the interaction is specified.

\subsection{Coulomb interaction}
\label{sec:inter_electrostatics}

\subsection{Other interaction types}
\label{sec:inter_other}

\subsection{Getting the currently defined interactions}

%\quickrefheading{Getting interactions}
\tclcommand{inter}{ }

%%% Local Variables: 
%%% mode: latex
%%% TeX-master: "ug"
%%% End: 
